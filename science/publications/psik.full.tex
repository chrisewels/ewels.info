%%%%%%%%%%%%%%%%%%%%%%%%%%%%%%%%%%%%%%%%%%%%%%%%%%%%%%%%%%%%%%%%%%%%%
%                                                                   %
%                                             24th November, 1995   %
%                                                                   %
%                                                                   %
%    Dear Networkers,                                               %
%                                                                   %
%                                                                   %
%                                                                   %
%    This is the LaTex file:    newsletter_12.tex                   %
%                               -----------------                   %
%    It contains the LaTeX version of the December issue            %
%    of the Network's Newsletter. This is the sixth and last        %
%    Newsletter of this calendar year. Altogether we have           %
%    published 12 Newsletters.                                      %
%                                                                   %
%                                                                   %
%    At the end of this LaTeX file, after \end{document}, we        %
%    have attached two additional files:                            %
%                                                                   %
%    form.tex: containing the Network Conference suggestion form    %
%    and                                                            %
%    psik.sty: file formatting the form.tex file                    %
%                                                                   %
%                                                                   %
%    You should save all these files separately and process         %
%    form.tex independently from newsletter_12.tex, to create       %
%    the conference suggestion form postscript file, which          %
%    then can be printed, filled and returned by surface mail,      %
%    if you so wish, to the conference organisers. If you prefer    %
%    to send the form by electromic mail you just need to extract   %
%    the form.tex file out of this document, fill in the spaces     %
%    in between the curly brackets and send the form.tex back to    %
%    psik@radix2.mpi-stuttgart.mpg.de.                              %
%                                                                   %
%                                                                   %
%    The command:                                                   %
%                       latex newsletter_12.tex                     %
%                                                                   %
%    will not need any additional files!                            %
%                                                                   %
%                                                                   %
%    The command:                                                   %
%                       latex form.tex                              %
%                                                                   %
%    will call upon psik.sty file!                                  %
%                                                                   %
%                                                                   %
%    The size of the printed Newsletter is 77 A4 pages.             %
%                                          ------------             %
%    The size of the printed suggestion form is 2 A4 pages.         %
%                                               -----------         %
%                                                                   %
%    Let us know of any problems with processing these LaTeX files. %
%    If you would prefer the postscript file of the Newsletter      %
%    please get back to us.                                         %
%                                                                   %
%    This Newsletter is as usually available by anonymous ftp       %
%    and also its postscript version can be downloaded within WWW   %
%    (see this and previous newsletters for details).               %
%                                                                   %
%                                                                   %
%                                                                   %
%                                                                   %
%    Please note, that next newsletter, the February issue, will    %
%    be e-mailed to you at the end of January 1996. Please submit   %
%    all your contributions for it by 29th January. Thanks.         %
%                                                                   %
%                                                                   %
%                                                                   %
%                                      The Editor                   %
%                                                                   %
%                                                                   %
%%%%%%%%%%%%%%%%%%%%%%%%%%%%%%%%%%%%%%%%%%%%%%%%%%%%%%%%%%%%%%%%%%%%%
\documentstyle[12pt]{article}
\parindent 0pt
\topmargin -2cm
\hoffset -1.5cm
\textheight 25cm
\textwidth 16 cm
\renewcommand{\baselinestretch}{1.2}
\def\ni{\noindent}
\begin{document}
\parskip 1ex
%%%%%%%%%%%%%%%%%%%%%%%%%%%%%%%%%%%%%%%%%%%%%%%%%%%%%%%%%%%%%%%%%%%%%%%%%%%%%
%      This is the front page .......................                       %
%%%%%%%%%%%%%%%%%%%%%%%%%%%%%%%%%%%%%%%%%%%%%%%%%%%%%%%%%%%%%%%%%%%%%%%%%%%%%
\vspace{2cm}
\begin{center}
{\Huge HCM Newsletter}
\end{center}
\vspace{3cm}
\rule{16.0cm}{1mm}
\vspace{1cm}
\begin{center}
{\huge \bf $\Psi_{k}$ Network}
\end{center}
\begin{center}
{\Large \bf AB INITIO (FROM ELECTRONIC STRUCTURE) 
CALCULATION OF COMPLEX PROCESSES IN MATERIALS}
\end{center}
\vspace{1cm}
\rule{16.0cm}{1mm}
{Number 12 \hspace{110mm} December 1995\\
\rule{16.0cm}{0.5mm}}
\vspace{8cm} 
%\vspace{10cm} 
\rule{16.0cm}{0.0mm}
\rule{16.0cm}{0.5mm}
\underline{Editor:} Z. (Dzidka) Szotek\\
{\underline{Proposal:} ERB4050PL930589 \hspace{45mm} \underline{Coordinator:} Walter Temmerman }
{\underline{Contract:} ERBCHRXCT930369 \hspace{41mm} \underline{E-mail:} psik-coord@daresbury.ac.uk }
%%%%%%%%%%%%%%%%%%%%%%%%%%%%%%%%%%%%%%%%%%%%%%%%%%%%%%%%%%%%%%%%%%%%%%%%%%%%%
%                                                                           %
%           Here the Editorial starts .................                     %
%                                                                           %
%%%%%%%%%%%%%%%%%%%%%%%%%%%%%%%%%%%%%%%%%%%%%%%%%%%%%%%%%%%%%%%%%%%%%%%%%%%%%
\newpage
\null
\begin{center}
\Large {\bf Editorial \par}
\end{center}

\ni The {\bf Editorial} is followed by the section {\bf News from the Network}. In this section
readers can find information by Volker Heine (the chairman of the Network Management Board) on 
{\it Collaborative and 
Training Projects} established at the recent meeting of the {\it Network Management Board}. 
Moreover, there we also write about the need of receiving all possible information from those of 
you who in any way benefited from the Network in your research.\\

\ni In the section {\bf News from the Working Groups} readers will find reports on recent 
collaborative visits and workshops. These are followed by job announcements. \\ 

\ni The {\bf Abstracts Section} is followed by the section {\bf Presenting Other HCM Projects} 
where we have an article by {\it Chris Ewels} (Exeter) on his research done in CINECA (Bologna) 
using the High Performance Computing Facility within the HCM Programme {\it ICARUS Scheme}. 
In this article readers can also find information on how to apply and make use of this and other
similar facilities supported by the HCM Programme.\\

\ni In the {\bf Highlight of the Month} section we have an article by 
{\it Peter Zahn, I. Mertig, Manuel Richter and Helmut Eschrig} on {\it "Ab-initio
Calculation of Giant Magnetoresistance in Magnetic Multilayers"}.\\

\ni Finally, partly due to their complicated nature, the {\bf Announcement of the Bond Order Potentials Workshop}, 
the {\bf Announcement of the OXYGEN 1996 Workshop}, and the {\bf 2nd Circular of the Network Conference} 
are placed at the very end of the {\it Newsletter}. The {\bf 2nd Circular} refers to a conference suggestion
form {\it form.tex} which calls upon a formatting file {\it psik.sty}. The LaTeX file of the suggestion form,
{\it form.tex}, as well as the {\it psik.sty} file, are attached at the end of the LaTeX file
of this {\it Newsletter}, just after the "\verb|\end{document}|". In order to create a printed version of the 
suggestion form one needs to save all these files separately, and process \verb|form.tex| independently from 
\verb|newsletter_12.tex| file, containing the main body of the {\it Newsletter}. Please note that those of you who 
normally receive only the postscript file of the {\it Newsletter} will additionally receive the conference suggestion 
form as a separate postscript file: \verb|form.ps|.\\

\vspace{0.5cm}
\ni The {\it Network} has a home page on World Wide Web (WWW). Its {\it Uniform Resource
Locator} (URL) is:\\
{\bf http://www.dl.ac.uk/TCSC/HCM/PSIK/main.html}\\

\ni This also contains pointers to several other nodes: {\it O.K. Andersen (Stuttgart)} which 
includes
information on the {\bf Network Conference}, {\it M. Gillan (Keele)}, {\it B.L. Gyorffy (Bristol)}, 
{\it J. N{\o}rskov (Lyngby)} 
with information on CAMP, {\it M. Scheffler (FHI Berlin)}, and {\it A. Walker (UEA Norwich)}. 
If you maintain a home page on your activities we will be happy to include a pointer from
the {\it Network's} home page to your home page.\\

\noindent
{\bf {Please submit all material for the next newsletters to the email address below.}}

%\bigskip
%\newpage
\ni The following email addresses are repeated for your convenience,
and are the easiest way to contact us.\\

\noindent
{\small \bf \begin{tabular} [t] {ll}
 & function \\
psik-coord@daresbury.ac.uk & messages to the coordinator \& newsletter \\
psik-management@daresbury.ac.uk & messages to the NMB \\
psik-network@daresbury.ac.uk & messages to the whole Network \\
\end{tabular} }\\

\bigskip
\ni Dzidka Szotek \& Walter Temmerman\\
\ni e-mail: psik-coord@daresbury.ac.uk\\

%%%%%%%%%%%%%%%%%%%%%%%%%%%%%%%%%%%%%%%%%%%%%%%%%%%%%%%%%%%%%%%%%%%%%%%%%%%%%
%                                                                           %
%    Here the News from the Network Section starts ...........              %
%                                                                           %
%%%%%%%%%%%%%%%%%%%%%%%%%%%%%%%%%%%%%%%%%%%%%%%%%%%%%%%%%%%%%%%%%%%%%%%%%%%%%
\newpage
\null
\begin{center}
\LARGE {\bf News from the Network \par}
\end{center}

\vspace{0.5cm}
\noindent
{\Large{\bf Money}, {\Large{\it Money}}, {\Large{Money, ..., {\it MONEY}}}\\

\normalsize
\ni That has caught your attention: good!\\

\ni The Network has money and would like to see more collaborations. The main
requirement is that first class science results, with publications. The main
boundary conditions are that we can only pay subsistance (now up to 60 ECU
per day) plus travel, not salary or equipment. Also, both collaborators must
be from an organisation that is a node of the Network.

\ni The Network is offering 10 grants for Collaboration and Training Projects
of 2000 ECU each. Together with the availability of email that should be enough 
to get a collaboration going. Once it is up and running, further applications can
be made in the normal way.

\ni There are many different types of collaboration. Perhaps an ideal one would
involve a younger scientist learning a new code to establish a new line of
research at his/her home university: but we are flexible.

\ni Two words of advice to those wishing to start a collaboration. Firstly, what
benefit does the other partner get from the collaboration? Perhaps you can help
further his/her research intrests with modest input in time and effort on his/her
side. Secondly, remember codes are shared in the Network through genuine
collaborations with the originator sharing control of the project and the way his/her
code is used, plus having his/her name on the research papers.

\ni The procedure is to approach the person or group you want to collaborate with, 
formulate the project and submit it as a Collaborative and Training Project with
a budget to the appropriate Working Group Spokesperson (details of the names and
addresses can be found in the Network's Newsletter No. 1) or to myself as Network's 
Chairman, or to Ole K. Andersen (oka@radix2.mpi-stuttgart.mpg.de) as Vice-Chairman, 
or to Walter M. Temmerman (w.m.temmerman@dl.ac.uk) as the Network's Coordinator. \\

\begin{flushright}
Volker Heine\\
fax: +44-1223-337356\\
e-mail: nwvh2@phy.cam.ac.uk\\
\end{flushright}
%%%%%%%%%%%%%%%%%%%%%%%%%%%%%%%%%%%%%%%%%%%%%%%%%%%%%%%%%%%%%%%%%%%%%%%%%%%%%
%     Monitoring the impact of the Network activity on research             %
%%%%%%%%%%%%%%%%%%%%%%%%%%%%%%%%%%%%%%%%%%%%%%%%%%%%%%%%%%%%%%%%%%%%%%%%%%%%%
\newpage
\vspace{2cm}
\begin{center}
\Large{\bf What impact has the Network had on your research? \par}
\end{center}

\noindent
Since the Network is coming up for renewal next year, we would like to appeal 
to those of you who in any way benefited from the Network in your research to
write to us about it. It appears that all the workshops, "hands-on a computer 
code" and others, organised by the Network during the last year and a half, 
emerge as the Network's key deliverable and will impact on the renewal of the
Network and future of science in different universities and countries. \\

\noindent
{\bf Therefore, if you have participated in any of the workshops organised by the Network
we would appreciate if you could contact us and provide  an information what 
impact has the workshop or an acquired code had on your research. Have you used the code or
the knowledge that you have gained at the workshop for your research? Have you 
published any papers whose results benefited from the code or workshop in general, 
and most importantly, have you acknowledged the code and the Network, as the organiser 
of the workshop, in any of your papers? We would also be interested in establishing
the multiplication effect: we started it.} \\

This information is of paramount importance for monitoring the impact that our HCM 
Network has had on research of people in general, and in particular, of those who 
in any way have benefited from the Network. Moreover, this information we need to 
submit to Brussels in the form of an annual report on Network's activity, but is also
going to be used to support our application for the renewal of the Network.  
We are not able to contact each of you individually, in many
cases we do not have your present e-mail addresses etc. So, please do get back to us
if you have anything to tell us on the above matters. Even if you do not use the 
code that you have acquired or knowledge that you have gained at any of the workshops; 
perhaps because you have found out why it cannot be of use to you, we would also 
like to hear it.\\

\noindent
We do also desperately need \underline{all} information on publications that acknowledge 
the {\it Network} explicitly, and which resulted from collaborations or other activities 
within the Network.\\

{\bf If any of you published or submitted for publication such papers
where the {\it Network} had been acknowledged, we would like to hear from you.
Could you please just send us detailed references of such papers, namely the
authors, journal (volume, page, year) and the title of the paper. We would not
like to miss any of such publications.} \\


%%%%%%%%%%%%%%%%%%%%%%%%%%%%%%%%%%%%%%%%%%%%%%%%%%%%%%%%%%%%%%%%%%%%%%%%%%%%%
%    Here the News from the Working Groups Section starts ........          %
%%%%%%%%%%%%%%%%%%%%%%%%%%%%%%%%%%%%%%%%%%%%%%%%%%%%%%%%%%%%%%%%%%%%%%%%%%%%%

\newpage
\vspace{2cm}
\begin{center}
\LARGE {\bf News from the Working Groups \par}
\end{center}
%%%%%%%%%%%%%%%%%%%%%%%%%%%%%%%%%%%%%%%%%%%%%%%%%%%%%%%%%%%%%%%%%%%%%%%%%%%%%
%                                                                           %
%   Report on  the visit of C.M.J. Wijers (Twente) to Aarhus                %
%                                                                           %
%%%%%%%%%%%%%%%%%%%%%%%%%%%%%%%%%%%%%%%%%%%%%%%%%%%%%%%%%%%%%%%%%%%%%%%%%%%%%
\vspace{0.5cm}
\normalsize
\begin{center}
\Large{\bf Report on the collaborative visit of C.M.J. Wijers (Twente)
to N. Christensen (Aarhus)}\\
\large{\bf 2 November, 1995}\\
\end{center}
\vspace{0.5cm}

\noindent
The issues of the visit were twofold: 1) to discuss a common project to be 
performed by a graduate student, Mr. Ivo Wenneker, from Twente University 
2) to give a lecture about the present status of the Twente discrete cellular 
project. The abstract of this lecture follows here below.\\

\bigskip
\noindent
{\large{\bf Discrete Cellular Methods: Maxwell's comeback?}}\\

\noindent
C.M.J. WIJERS,\\
{\it Faculty of Applied Science, Twente University,}\\
{\it P.O. Box 217, 7500 AE Enschede, the Netherlands}\\

\noindent
Large inhomogeneous systems like large molecules or surfaces, display optical 
responses deviating from the results of standard theoretical treatments. Those 
deviations can be amplified by using difference, anisotropic difference, 
SHG or optical scattering experiments. Electrodynamics offers two alternatives 
for the description of such phenomena: continuum or discrete, and both 
approaches have a sizeable record in physics. For solid state optics only the 
continuum description has been used to link the electrodynamics to the 
underlying quantum mechanics of the system. In this lecture arguments will be 
collected that it makes sense to look for the quantum mechanical roots of 
discrete models. If such link can be achieved, one gains immediately the 
ability of discrete models to handle directly a lot of different geometries. 
This pays off directly for large inhomogeneous systems. It turns out however 
that the link to quantum mechanics for discrete models is not a simple analogy 
of the continuum case. A necessary consequence of electrodynamic discreteness, 
is the introduction of cells in quantum mechanics. This means that we have 
to study open quantum mechanical systems. Another consequence is also that 
besides electrodynamical nonlocal interactions also quantum mechanical 
nonlocality needs to be taken into account. The talk will give (and trigger?) 
further discussion how to make discrete cellular methods work.\\

\bigskip
\noindent
As to the first item: Both groups are working on a full potential version of 
their electronic structure packages. The techniques used in the two groups, 
the LMTO-types and the CLOPW-types, have distinct advantages for different 
applications. In calculating NMR-spectra for heavy metals, a CLOPW-type of 
calculation might be supportive for the LMTO-type of calculation. This 
comparative study between the two methods, will be the research topic for 
the graduate student in his 4 month stay at Aarhus.\\

\begin{flushright}
(C.M.J. Wijers)
\end{flushright}

%%%%%%%%%%%%%%%%%%%%%%%%%%%%%%%%%%%%%%%%%%%%%%%%%%%%%%%%%%%%%%%%%%%%%%%%%%%%%
%                                                                           %
%   Report on  the visit of Pietro Ballone (MPI-Stuttgart) to Cambridge     %
%                                                                           %
%%%%%%%%%%%%%%%%%%%%%%%%%%%%%%%%%%%%%%%%%%%%%%%%%%%%%%%%%%%%%%%%%%%%%%%%%%%%%
\vspace{0.5cm}
\normalsize
\begin{center}
\Large{\bf Report on the collaborative visit of Pietro Ballone ((MPI-FKF, Stuttgart)
to Cambridge}\\
\large{\bf 21-25 October, 1995}\\
\end{center}

\noindent
Discussions were held with the group of Mike Payne and Volker Heine in the
Cavendish Laboratory on Monday 23 October on technical aspects of total energy 
calculations with pseudopotentials and plane waves, and their applications.  
Tuesday 24 October was taken up with a Minerals Physics
Workshop in Cambridge, at which a report was given on the
calculation of phase transitions under pressure of magnesium
silicate. Arrival on Saterday to get air fare reduction.
Only part of the cost is claimed from the Network, namely
70 pounds for accomodation in Cambridge. Charge: Cambridge node,
pseudopotential working group.\\


\begin{flushright}
(V. Heine)
\end{flushright}

%%%%%%%%%%%%%%%%%%%%%%%%%%%%%%%%%%%%%%%%%%%%%%%%%%%%%%%%%%%%%%%%%%%%%%%%%%%%%
%                                                                           %
%   Report on  the Central European Workshop n Czech Republic               %
%                                                                           %
%%%%%%%%%%%%%%%%%%%%%%%%%%%%%%%%%%%%%%%%%%%%%%%%%%%%%%%%%%%%%%%%%%%%%%%%%%%%%
\newpage
\null

\begin{center}

\LARGE{\it Report on}\\

\LARGE{\bf Fourth Central European Workshop}\\
\LARGE{\bf on Electronic and Magnetic Properties}\\
\LARGE{\bf of Alloys, Surfaces and Interfaces}\\
{\it 2--5 October, 1995}\\
\end{center}

\vskip3mm
This workshop was organized by the Institute of Physics of Materials, Academy of Sciences
of the Czech Republic, Brno, and the Institute of Physics, Academy of
Sciences of the Czech Republic, Prague, and was financially supported by the Brno Trade 
Fairs and Exhibitions Co.~Ltd., Czechoslovak Trade Bank and Czech Saving Bank. It was a follow 
up of the preceding meetings: in Vienna (P. Weinberger, 1992), in Budapest (J. Koll\'ar, 1993), 
and in the Swiss mountain village Stels (R. Monnier, 1994).  \\

The 1995 workshop took place in the Holiday Center of the Brno Trade Fairs and Exhibitions in
Zub\v{r}\'{\i}, Czech Republic, in the heart of the Czech-Moravian Highlands. It was attended
by 34 scientists from 10 different countries, all of whom were practicioners of Green's function 
techniques and full-potential methods for electronic structure calculations in alloys, surfaces 
and interfaces. There were many interesting talks and plenty of opportunities for participants
to discuss their recent work. The scientific program follows below.\\

The next, fifth, workshop is planned for the autumn of next year in Vienna.\\

\begin{tabular}{lr}
                     \hbox to 48mm{}  & (Mojm\'{\i}r \v Sob, Ilja Turek,
                           Josef Kudrnovsk\'y, V\'aclav Drchal) \\

\end{tabular}

%\begin{center}
{\Large{\bf Scientific Program}}
%\end{center}

\bigskip

{\Large{\bf Magnetic coupling and anisotropy}}

\vskip1mm

{\bf P. H. Dederichs} \\
              Interlayer coupling and quantum well states in Co/Cu
             layers

 {\bf L. Szunyogh}, P. Weinberger and B. \'Ujfalussy \\
               Fully relativistic spin-polarized study of Fe
                multilayers in Au(001)

{\bf B. \'Ujfalussy}, L. Szunyogh and P. Weinberger \\
                Studies of magnetic anisotropy in Fe-Cu over- and
                 interlayers

%\vskip4mm
%\newpage

{\Large{\bf Magnetic coupling and GMR}}

\vskip1mm

 {\bf J. Mathon}\\
                     Quantum well theory of the exchange coupling in
                      magnetic
                     multilayers

 {\bf J. Kudrnovsk\'y}, V. Drchal, I. Turek,
                      M. \v Sob and P. Weinberger \\
                     Interlayer magnetic coupling: effect of disorder
                      in spacer

 {\bf I. Mertig}, P. Zahn, M. Richter and H.
Eschrig\\
                     Ab-initio calculations of giant magnetoresistence

%\newpage

\vskip4mm
{\Large{\bf Alloys }}

\vskip1mm

 A. V. Ruban and {\bf H. L. Skriver}\\
                Calculated site substitution in $\gamma'$-Ni$_3$Al

 {\bf V. Drchal}, J. Kudrnovsk\'y, I. Turek and
                 A. Pasturel \\
                Ab-initio theory of surface segregation in random
                 alloys

  T. Schulthess and {\bf R. Monnier} \\
                Influence of charge correlations on the surface energy,
                work
                 function and surface composition of random binary
                 alloys

{\bf P. Weinberger}\\
                Electrical conductivity in semi-infinite systems

{\bf A. B. Shick}, V. Drchal and J. Kudrnovsk\'y \\
                Relativistic spin-polarized TB-LMTO method: application
                 to
                 magnetic
                properties
                 of random alloys and their
                 surfaces

 {\bf H. Ebert} and G.-Y. Guo \\
                Theoretical investigation of the magnetic X-ray
                dichroism in
                 diluted
                 and concentrated transition
                metal alloys

 {\bf J. Banhart} and H. Ebert \\
                Ab-initio theory of spontaneous galvanomagnetic
                 effects in
                 random
                 alloys

\vskip4mm

{\Large{\bf Full-potential methods }}

\vskip1mm

 L. Vitos, {\bf J. Koll\'ar}
 and H. L. Skriver \\
                Full charge density scheme:
                 application for shear deformations

 O. Genser, A. Biedermann, {\bf J. Redinger} and P.
Varga \\
                Surface electronic structure of p-d bonded systems:
                Some recent results

{\bf C. Blaas}\\
                High resolution Compton scattering in Fermi surface
                 studies

R. Stadler, W. Wolf, G. Kresse, J. Furthm\"uller,
                 {\bf R. Podloucky}\\
                  and J. Hafner \\
                 Electronic structure, elastic constants and phonon
                  dispersion
                   of CoSi$_2$

{\bf F. M\'aca} and M. Scheffler \\
                Electronic structure and STM image of a p(2x2)
               composite double
                layer ordered surface alloy of Na on Al(111)

%\newpage
L. Steinbeck, M. Richter, U. Nitzsche and {\bf H.
Eschrig}\\
                Ab-initio calculation of electronic structure, crystal
               field, and
                intrinsic magnetic properties of Sm$_2$Fe$_{17}$,
               Sm$_2$Fe$_{17}$N$_3$,
                Sm$_2$Fe$_{17}$C$_3$ and Sm$_2$Co$_{17}$

%\newpage

\vskip4mm
{\Large{\bf New methods }}

\vskip1mm

P. Miller and {\bf B. Gyorffy}\\
                On the mechanism of the de Haas--van Alphen
               oscillations in \\
                the superconducting state

 {\bf I. Turek}, J. Kudrnovsk\'y,
                      M. \v Sob and V. Drchal  \\
                Surface magnetism of random binary
                    transition-metal alloys

                {\bf L. Udvardi}, L Szunyogh and R. Kir\'aly \\
                Magnetic anisotropy on overlayers


{\bf B. Velick\'y}, H. Eschrig, K. Koepernik,
                       A. Ernst and R. Hayn\\
                     Towards ab-initio CPA for alloys using mixed basis
                    set

                C. Barreteau and {\bf F. Ducastelle} \\
                Quantitative transmission electron microscopy in
                 disordered alloys

J. Vack\'a\v r, {\bf A. \v Sim\accent23unek} and R.
Podloucky\\
                     Ab-initio pseudopotentials of interacting atoms


%%%%%%%%%%%%%%%%%%%%%%%%%%%%%%%%%%%%%%%%%%%%%%%%%%%%%%%%%%%%%%%%%%%%%%%%%%%%%
%                                                                           %
%      Job Announcements .................                                  %
%                                                                           %
%%%%%%%%%%%%%%%%%%%%%%%%%%%%%%%%%%%%%%%%%%%%%%%%%%%%%%%%%%%%%%%%%%%%%%%%%%%%%
%%%%%%%%%%%%%%%%%%%%%%%%%%%%%%%%%%%%%%%%%%%%%%%%%%%%%%%%%%%%%%%%%%%%%%%%%%%%%
%      Ph. D. in Sheffield
%%%%%%%%%%%%%%%%%%%%%%%%%%%%%%%%%%%%%%%%%%%%%%%%%%%%%%%%%%%%%%%%%%%%%%%%%%%%%
\newpage
\null
\vspace{0.5cm}
\begin{center}
\LARGE{\it  Announcement }\\[1.0cm]
\LARGE{\bf Ph. D. Position}\\[0.3cm]
{\Large {\it University of Sheffield (UK)}}\\
\end{center}
 
\ni A Ph. D. studentship is available as soon as possible to work with Professor
Gillian Gehring (University of Sheffield) and Dr. Walter Temmerman (Daresbury
Laboratory) on the total energy and Fermi surface studies of the heavy fermion compounds.
 
\ni A prospective student will be expected to make a detailed study of $UBe_{13}$. This
is expected to help the interpretation of the planned de Haas van Alphen experiments
in Professor Mike Springford's group (University of Bristol). The theoretical study
will involve both development or generalisation of computer codes and necessary calculations.
 
\ni To apply, please send your CV and letter of application to the addresses below either
by normal or electronic (only postscript or LaTeX files are acceptable) mail.\\
\ni Further information can be obtained from:\\
 
Dr. W.M. Temmerman,\\
Daresbury Laboratory,\\
Daresbury, Warrington, WA4 4AD\\
Cheshire, UK\\
tel.: +44-(0)1925-603227 \\
fax:  +44-(0)1925-603634 \\
e-mail: w.m.temmerman@daresbury.ac.uk \\
 
or\\
 
Professor G. Gehring \\
Physics Department, \\
University of Sheffield \\
Sheffield,S3 7RH, UK \\
tel.: +44-(0)114-282 4299 \\
fax:  +44-(0)114-272 8079 \\
e-mail: g.ghering@sheffield.ac.uk  \\
 
%%%%%%%%%%%%%%%%%%%%%%%%%%%%%%%%%%%%%%%%%%%%%%%%%%%%%%%%%%%%%%%%%%%%%%%%%%%%%
%      Postdoc in National Renewable Energy Laboratory, Golden, Colorado    %
%%%%%%%%%%%%%%%%%%%%%%%%%%%%%%%%%%%%%%%%%%%%%%%%%%%%%%%%%%%%%%%%%%%%%%%%%%%%%
\newpage
\null
\vskip 2em
\begin{center}
{\Large{\it  Announcement}}\\[0.5cm]
\Large{\bf Postdoctoral Research Positions in Electronic Structure Theory
of Solids \par}
\vskip 1.5em
{\large \lineskip .5em \begin{tabular}[t]{c}
Solid State Theory Group\\
Basic Sciences Division\\
National Renewable Energy Laboratory\\
Golden, Colorado 80401, USA\\
\end{tabular}\par}
\end{center}

%\begin{center}
%{\Large{\it  Announcement}}\\[0.5cm]
%{\LARGE{\bf Postdoctoral Research Positions in Electronic Structure Theory 
%of Solids}}\\[1.0cm]
%Solid State Theory Group\\
%Basic Sciences Division\\
%National Renewable Energy Laboratory\\
%Golden, Colorado 80401, USA\\
%\end{center}

\bigskip
%\vspace{1.0cm}
\noindent
The Solid State Theory Group at the National Renewable Energy Laboratory 
invites applications for two postdoctoral research positions, one beginning January, 
1996, and one beginning September, 1996. The positions are for two years, 
renewable upon mutual agreement to a third year. The Solid State Theory Group
currently consists of eight Ph.D.'s in condensed matter theory and interacts 
with a broad range of experimentalists in semiconductor physics at NREL. 
Areas of particular interest include first-principles electronic structure theory of
semiconductor nanostructures and superlattices and the theory of 
phase-stability of metallic alloys. Candidates should send a curriculum vitae,
list of publications (including preprints of unpublished papers, 
if possible) and list of three references to:\\

%\vspace{0.5cm}
\noindent
Dr. Alex Zunger \\
Solid State Theory Group \\
National Renewable Energy Laboratory \\
1617 Cole Boulevard \\
Golden, Colorado, 80401 \\

\vspace{0.5cm}
\noindent
NREL is an equal opportunity/affirmative action employer. Clarifications or
further details can be obtained via e-mail to {\bf alex\_zunger@nrel.gov}. \\


%%%%%%%%%%%%%%%%%%%%%%%%%%%%%%%%%%%%%%%%%%%%%%%%%%%%%%%%%%%%%%%%%%%%%%%%%%%%%%%%%
%  Abstracts                                                                    %
%%%%%%%%%%%%%%%%%%%%%%%%%%%%%%%%%%%%%%%%%%%%%%%%%%%%%%%%%%%%%%%%%%%%%%%%%%%%%%%%%

\newpage
\null
\vskip 2em
\begin{center}
\Large{\bf Noncollinear Magnetic Order and Electronic Properties of
$\rm U_2Pd_2Sn$ and $\rm U_3P_4$ \par}
\vskip 1.5em
{\large \lineskip .5em \begin{tabular}[t]{c}
L.M. Sandratskii and J. K\"ubler\\
{\it Institut f\"ur Festk\"orperphysik, Technische Hochschule,} \\
{\it D-64289 Darmstadt, Germany} \\
\end{tabular}\par}
\end{center}
\begin{abstract}
We report results of calculations that explain in the
itinerant electron picture the noncollinear
magnetic structure observed in the two very different compounds
$ \rm U_2Pd_2Sn$ and $ \rm U_3P_4$.
We use the local approximation to spin-density functional theory
and the ASW method incorporating spin-orbit coupling (SOC),
noncollinear moment arrangements and an effective orbital field
responsible for Hund's second rule.
We show how the relativistic effect of SOC
and the particular symmetry properties of the compounds
cooperate and lead to noncollinear magnetic structures, in the
case of $ \rm U_2Pd_2Sn$ to an antiferromagnetic and in the
case of  $ \rm U_3P_4$ to a ferromagnetic structure the
latter possessing  a weak antiferromagnetic component.
\end{abstract}

\noindent
(submitted to Physica B)\\
Revtex version can be obtained from: dg5m@mad1.fkp.physik.th-darmstadt.de \\
(L. Sandratskii)\\

\noindent
This work has benefited from collaborations within the EU Human
Capital and Mobility Network on
{\it "Ab initio (from electronic structure) calculation of complex
processes in materials"} (contract: \\ ERBCHRXCT930369).\\

\newpage
\null
\vskip 2em 
\begin{center} 
\LARGE 
{\bf Ab initio pseudopotential study of Fe, Co, and Ni 
employing the spin-polarized LAPW approach  %<--TITLE-----
\par} 
\vskip 1.5em 
{\large \lineskip .5em \begin{tabular}[t]{c} \\
Jun-Hyung Cho and Matthias Scheffler  %<--AUTHORS---
\\
{\it Fritz-Haber-Institut der Max-Planck-Gesellschaft,}\\ 
{\it Faradayweg 4-6, D-14195 Berlin-Dahlem, Germany}\\  %<--ADDRESSES-
\end{tabular}\par} 
\end{center} 
\begin{abstract} 
The ground-state properties of Fe, Co, and Ni are studied with the 
linear-augmented-plane-wave (LAPW) method and norm-conserving pseudopotentials.
The calculated lattice constant, bulk modulus, and magnetic moment
with both the local-spin-density approximation (LSDA) and the generalized
gradient approximation (GGA) are in good agreement with those of 
all-electron calculations, respectively.
The GGA results show a substantial improvement over the LSDA results,
i.e., better agreement with experiment.
The accurate treatment of the nonlinear core-valence exchange and correlation
interaction is found to be essential for the determination of the magnetic 
properties of $3d$ transition metals. 
The present study demonstrates the successful application of the
LAPW pseudopotential approach to the calculation 
of ground-state properties of magnetic $3d$ transition metals.
  %<--ABSTRACT--
\end{abstract} 
(Submitted to 
Phys. Rev. B%<--JOURNAL---
) \\ 
REVTEX %<--CHOOSE-VERSION--
version can be obtained from: 
\verb|cho@theo21.RZ-Berlin.MPG.DE| \\

\newpage
\null
\vskip 2em 
\begin{center} 
\LARGE{\bf
Six-Dimensional Quantum Dynamics of Adsorption and Desorption 
       of H$_2$ at Pd(100):\\ Steering and Steric Effects  %<--TITLE-----
\par} 
\vskip 1.5em 
{\large \lineskip .5em \begin{tabular}[t]{c} \\
Axel Gross, Steffen Wilke, and Matthias Scheffler  %<--AUTHORS---
\\
{\it Fritz-Haber-Institut der Max-Planck-Gesellschaft}\\
{\it Faradayweg 4-6, D-14195 Berlin-Dahlem, Germany}\\ %<--ADDRESSES-
\end{tabular}\par} 
\end{center} 
\begin{abstract}
We report the first six-dimensional quantum dynamical calculations of
dissociative adsorption and associative desorption. Using a potential 
energy surface obtained by density functional theory calculations, we show that
the increase of the sticking probability with decreasing kinetic energy 
at low kinetic energies in the system H$_2$/Pd(100), 
which is usually attributed to the existence of
a molecular adsorption state, is due to dynamical steering. 
In addition, we examine the influence of rotational motion and 
orientation of the hydrogen molecule on adsorption and desorption.
  %<--ABSTRACT--
\end{abstract} 
(Submitted to 
Phys. Rev. Lett.%<--JOURNAL---
) \\ 
REVTEX %<--CHOOSE-VERSION--
version can be obtained from: 
\verb|axel@theo22.RZ-Berlin.MPG.DE| \\ %<--EMAIL-ADDRESS--

\newpage
\null
\vskip 2em 
\begin{center} 
\LARGE{\bf
Enhanced electron-phonon coupling at the Mo and W (110) surfaces
induced by adsorbed hydrogen  %<--TITLE-----
\par} 
\vskip 1.5em 
{\large \lineskip .5em \begin{tabular}[t]{c} \\
Bernd Kohler, Paolo Ruggerone, and Matthias Scheffler  %<--AUTHORS---
\\
{\it Fritz-Haber-Institut der Max-Planck-Gesellschaft}\\
{\it Faradayweg 4-6, D-14195 Berlin-Dahlem, Germany}\\
\ \\  %<--ADDRESSES-
Erio Tosatti  %<--AUTHORS---
\\
{\it Instituto Nazionale di Fisica della Materia (INFM)}\\
{\it International School for Advanced Studies (SISSA)}\\ 
{\it and International Centre for Theoretical Physics (ICTP)}\\
{\it Miramare, I-34014 Trieste, Italy}\\  %<--ADDRESSES-
\end{tabular}\par} 
\end{center} 
\begin{abstract} 
The possible occurrence of either a charge-density-wave or a Kohn anomaly
is governed by the 
presence of Fermi-surface nesting and the subtle
interaction of electrons and phonons. 
Recent experimental and theoretical investigations 
suggest such an effect for the 
hydrogen covered Mo and W (110) surfaces. Using density-functional theory
we examine the electronic structure and the electron-phonon coupling of these
systems. 
Besides  good agreement with the experimental phonon frequencies our
study provides a characterization and quantitative analysis
of an unusual scenario determining the electronic, vibrational, and structural
properties of these surfaces.
  %<--ABSTRACT--
\end{abstract} 
(Submitted to 
Phys. Rev. Lett.%<--JOURNAL---
) \\ 
REVTEX %<--CHOOSE-VERSION--
version can be obtained from: 
\verb|kohler@theo26.RZ-Berlin.MPG.DE| \\ %<--EMAIL-ADDRESS--

\newpage
\null
\vskip 2em 
\begin{center} 
\LARGE{\bf
Force calculation and atomic-structure optimization for
the full-potential linearized augmented plane-wave code 
{\tt WIEN}  %<--TITLE-----
\par} 
\vskip 1.5em 
{\large \lineskip .5em \begin{tabular}[t]{c} \\
 Bernd Kohler, Steffen Wilke, and Matthias Scheffler\\  %<--AUTHORS---
{\it Fritz-Haber-Institut der Max-Planck-Gesellschaft}\\
{\it Faradayweg 4-6, D-14195 Berlin-Dahlem, Germany}\\
\ \\
Robert Kouba and Claudia Ambrosch-Draxl\\
{\it Institut f\"ur theoretische Physik, Universit\"at Graz}\\
{\it Universit\"atsplatz 5, A-8010 Graz, Austria} \\  %<--ADDRESSES-
\end{tabular}\par} 
\end{center} 
\begin{abstract} 
Following the approach of Yu, Singh, and Krakauer 
[Phys.\,Rev.\,B\,{\bf43} (1991) 6411] 
we extended the linearized augmented plane wave code 
\verb|WIEN| of Blaha, Schwarz, and coworkers by the 
evaluation of forces. 
In this paper we describe the approach, demonstrate 
the high accuracy of the force calculation,  and use them for 
an efficient geometry optimization of poly-atomic systems.
  %<--ABSTRACT--
\end{abstract} 
(Submitted to 
Comp. Phys. Commun.%<--JOURNAL---
) \\ 
Postscript %<--CHOOSE-VERSION--
version can be obtained from: 
\verb|kohler@theo26.RZ-Berlin.MPG.DE| \\ %<--EMAIL-ADDRESS--

\noindent
This work has benefited from collaborations within the EU Human
Capital and Mobility Network on
{\it "Ab initio (from electronic structure) calculation of complex
processes in materials"} (contract: \\ ERBCHRXCT930369).\\

\newpage
\null
\vskip 2em 
\begin{center} 
\LARGE{\bf 
Theoretical study of O adlayers on Ru\,(0001)  %<--TITLE-----
\par} 
\vskip 1.5em 
{\large \lineskip .5em \begin{tabular}[t]{c} \\
Catherine Stampfl and Matthias Scheffler  %<--AUTHORS---
\\
{\it Fritz-Haber-Institut der Max-Planck-Gesellschaft} \\
{\it Faradayweg 4-6, D-14195 Berlin-Dahlem, Germany} \\  %<--ADDRESSES-
\end{tabular}\par} 
\end{center} 
\begin{abstract} 
Recent experiments performed at high pressures indicate that ruthenium
can support unusually high concentrations of oxygen at the surface.
To investigate the structure and stability of high coverage oxygen 
structures, we performed
density functional theory calculations, within the generalized
gradient approximation, for
O adlayers on Ru\,(0001) from low coverage up to a full monolayer.
We achieve quantitative agreement with previous low energy 
electron diffraction intensity analyses for the
$(2 \times 2)$ and $(2 \times 1)$ phases
and predict that an O adlayer with a $(1 \times 1)$ periodicity 
and coverage $\Theta$=1
can form on Ru\,(0001), where the O adatoms occupy hcp-hollow sites.
  %<--ABSTRACT--
\end{abstract} 
(Submitted to 
Phys. Rev. B%<--JOURNAL---
) \\ 
REVTEX %<--CHOOSE-VERSION--
version can be obtained from: 
\verb|cts@theo21.RZ-Berlin.MPG.DE| \\ %<--EMAIL-ADDRESS--

\newpage
\null
\vskip 2em 
\begin{center} 
\LARGE{\bf
Potential Energy Surface for H$_2$ Dissociation over Pd(100)  %<--TITLE-----
\par} 
\vskip 1.5em 
{\large \lineskip .5em \begin{tabular}[t]{c} \\
Steffen Wilke$^a$ and Matthias Scheffler  %<--AUTHORS---
\\
{\it Fritz-Haber-Institut der Max-Planck-Gesellschaft}\\
{\it Faradayweg 4-6, D-14195 Berlin-Dahlem, Germany}\\
{\it $^a$ Exxon Res. \& Eng. Comp., Annandale, NJ 08801, USA}\\  %<--ADDRESSES-
\end{tabular}\par} 
\end{center} 
\begin{abstract} 
The potential energy surface (PES) of dissociative adsorption of H$_2$ on
Pd(100) is investigated using density functional theory and the full-potential
linear augmented plane wave (FP-LAPW) method.  Several dissociation pathways
are identified which have a vanishing energy barrier. A pronounced dependence
of the potential energy on ``cartwheel'' rotations of the molecular axis is
found. The calculated PES shows no indication of the presence of a precursor
state in front of the surface. Both results indicate that steering effects
determine the observed decrease of the sticking coefficient at low energies of
the H$_2$ molecules. We show that the topology of the PES is related
to the dependence of the covalent H($s$)-Pd($d$) interactions on the
orientation of the H$_2$ molecule.
  %<--ABSTRACT--
\end{abstract} 
(Submitted to 
Phys. Rev. B%<--JOURNAL---
) \\ 
REVTEX %<--CHOOSE-VERSION--
version can be obtained from: 
\verb|swilke@vnatolisgi.erenj.com| \\ %<--EMAIL-ADDRESS--

\newpage
\null
\vskip 2em 
\begin{center} 
\LARGE{\bf 
Scattering of rare-gas atoms at a metal surface:
evidence of anticorrugation of the helium-atom
potential-energy surface and the surface electron density  %<--TITLE-----
\par} 
\vskip 1.5em 
{\large \lineskip .5em \begin{tabular}[t]{c} \\
Max Petersen, Paolo Ruggerone, Bernd Kohler,\\
 and Matthias Scheffler  %<--AUTHORS---
\\
{\it Fritz-Haber-Institut der Max-Planck-Gesellschaft} \\
{\it Faradayweg 4-6, D-14195 Berlin-Dahlem, Germany}\\  %<--ADDRESSES-
\ \\
Steffen Wilke\\
{\it Exxon Res. \& Eng. Comp., Annandale, NJ 08801, USA} \\
\end{tabular}\par} 
\end{center} 
\begin{abstract} 
Recent measurements of the scattering of He and Ne atoms at Rh\,(110)
suggest that these two rare-gas atoms measure a {\em qualitatively} different
surface corrugation: While Ne atom scattering seemingly reflects
the electron-density undulation of the substrate surface, the scattering
potential of He atoms appears to be anticorrugated. An understanding of this
perplexing result is lacking. In  this paper we present density functional
theory calculations of the interaction potentials of He and Ne with  Rh\,(110).
We find that, and explain why, the nature of the interaction of the two
probe particles is qualitatively different, which implies that
the topographies of their scattering potentials are indeed anticorrugated.
  %<--ABSTRACT--
\end{abstract} 
(Submitted to 
Phys. Rev. Lett.%<--JOURNAL---
) \\ 
REVTEX %<--CHOOSE-VERSION--
version can be obtained from: 
\verb|paolo@theo24.RZ-Berlin.MPG.DE| \\ %<--EMAIL-ADDRESS--

\newpage
\null
\vskip 2em 
\begin{center} 
\LARGE{\bf
Unusually Large Thermal Expansion of Ag(111)  %<--TITLE-----
\par} 
\vskip 1.5em 
{\large \lineskip .5em \begin{tabular}[t]{c} \\
Shobhana Narasimhan and Matthias Scheffler  %<--AUTHORS---
\\
{\it Fritz-Haber-Institut der Max-Planck-Gesellschaft} \\ 
{\it Faradayweg 4-6, D-14\,195 Berlin-Dahlem, Germany} \\ %<--ADDRESSES-
\end{tabular}\par} 
\end{center} 
\begin{abstract} 
We investigate the thermal behavior of the (111) surface of silver, 
using  phonon frequencies
obtained from {\it ab initio} total energy calculations, and anharmonic 
effects treated
within a quasiharmonic approximation.
Our results reproduce the experimental
observation of a large and anomalous increase in the surface thermal 
expansion at high
temperatures~[P.~Statiris {\em et al.}, 
Phys.~Rev.~Lett. {\bf 72}, 3574 (1994)]. Surprisingly, we find
that this increase can be attributed to a rapid softening
of the {\it in-plane}
phonon frequencies, rather than due to the anharmonicity of the 
out-of-plane surface phonon modes. This
provides evidence for a new mechanism for the enhancement of surface 
anharmonicity. A comparison with Al(111)
shows that the two surfaces behave quite differently, with no 
evidence for such anomalous behavior on Al(111).
  %<--ABSTRACT--
\end{abstract} 
(Submitted to 
Phys. Rev. Lett.%<--JOURNAL---
) \\ 
REVTEX %<--CHOOSE-VERSION--
version can be obtained from: 
\verb|shobhana@theo26.RZ-Berlin.MPG.DE| \\ %<--EMAIL-ADDRESS--


\newpage
\null
\vskip 2em
\begin{center}
\Large{\bf  Electronic Structure of 
PrBa$_{2}$Cu$_{3}$O$_{7}$:\\
A local-spin-density approximation\\
with on-site Coulomb interaction calculation \par}
\vskip 1.5em
{\large \lineskip .5em \begin{tabular}[t]{c}
M. Biagini, C. Calandra and Stefano Ossicini\\
{\it Istituto Nazionale Fisica della 
Materia and}\\
{\it Dip. di Fisica, Universit\`a di Modena,}\\
{\it via Campi 213/A, I-41100 Modena, Italy} \\
\end{tabular}\par}
\end{center}
\begin{abstract}
Electronic structure calculations based on the local-spin-density
approximation fail to reproduce the antiferromagnetic ground state
of PrBa$_{2}$Cu$_{3}$O$_{7}$ (PBCO).
We have performed LMTO-ASA
calculations, based on the local-spin-density 
approximation with on-site Coulomb correlation applied to Cu(1) 
and Cu(2) $3d$ states. 
We have found that inclusion of the on-site Coulomb interaction
modifies qualitatively the electronic structure of PBCO with respect to the 
LSDA results, and gives Cu spin moments 
in good agreement with the experimental values.
The Cu(2) upper Hubbard band lies about 1 eV above the
Fermi energy, indicating a Cu$^{II}$ oxidation state. 
On the other hand,
the Cu(1) upper Hubbard band  is located  across the Fermi level, 
which implies
an intermediate oxidation state for the Cu(1) ion,
between Cu$^{I}$ and Cu$^{II}$.
The metallic character of the 
CuO chains is preserved, 
in agreement with optical reflectivity
[K. Takenaka {\em et al.}
Phys. Rev. B {\bf 46}, 5833 (1992)]
and positron annihilation experiments
[L. Hoffmann {\em et al.} Phys. Rev. Lett. {\bf 71}, 4047 (1993)].
These results support the view of an
extrinsic origin of the insulating character of PrBa$_{2}$Cu$_{3}$O$_{7}$.
\end{abstract}

\noindent
(Submitted to Phys. Rev. B)\\
Manuscripts available from: biagini@imoax1.unimo.it \\


\newpage
\null
\vskip 2em
\begin{center}
\Large{\bf High Resolution Compton Scattering in Fermi Surface Studies:
       Application to FeAl \par}
\vskip 1.5em
{\large \lineskip .5em \begin{tabular}[t]{c}
C.~Blaas and J.~Redinger \\
{\it Institut f\"ur Technische Elektrochemie,}\\
{\it Technische Universit\"at Wien,}\\
{\it Getreidemarkt 9/158, A-1060 Wien, Austria}\\
S.~Manninen, V.~Honkim\"aki, and K.~H\"am\"al\"ainen \\
{\it Department of Physics, University of Helsinki,}\\
{\it P.O. Box 9, FIN-00014 Finland} \\
P.~Suortti \\
{\it European Synchrotron Radiation Facility,}\\
{\it B.P. 220, F-38043 Grenoble, France} \\
\end{tabular}\par}
\end{center}
\begin{abstract}
We present a novel theoretical approach to identify Fermi surface
contributions in high resolution Compton scattering data. For FeAl
Compton profiles we show that the Fermi-surface-related features
(occupation of states) can be separated from those originating from
the spatial extent of the wave functions. The first high resolution
Compton scattering experiment done at the European Synchrotron
Radiation Facility confirms these findings. This technique
opens up new possibilities especially in Fermiology studies of high
temperature superconductors.
\end{abstract}

\noindent
(Phys. Rev. Lett. {\bf 75}, 1984 (1995))\\
Reprints available from: cb\verb|@|eecws6.tuwien.ac.at \\\\


\newpage
\null
\vskip 2em
\begin{center}
\Large{\bf Fully relativistic spin--polarized description of interface exchange
coupling: Fe multilayers in Au(100) \par}
\vskip 1.5em
{\large \lineskip .5em \begin{tabular}[t]{c}
L. Szunyogh$^{a,b}$, B. \'Ujfalussy$^a$, P. Weinberger$^a$ and C. Sommers$^c$ \\
{\it $^a$Institut f\"ur Technische Elektrochemie,}\\
{\it Technische Universit\"at Wien,}\\
{\it Getreidemarkt 9/158, A-1060, Wien, Austria }\\
{\it $^b$Institute for Physics, Technical University Budapest,}\\
{\it Budafoki, \'ut 8,H-1111, Budapest, Hungary }\\
{\it $^c$Laboratoire de Physique des Solides, }\\
{\it Bat. 510, Campus d'Orsay, 91 405 Orsay, France}\\
\end{tabular}\par}
\end{center}
\begin{abstract}
The spin-polarized fully relativistic Screened Korringa-Kohn-Rostocker
method is applied to calculate magnetic anisotropy energies for Fe
multilayers in Au(100), separated by up to 16 spacer layers of Au. With
respect to the antiparallel as well as to a perpendicular relative
orientation of the magnetization in the Fe slabs these anisotropy energies
as calculated (a) in terms of total energies, (b) within the force theorem
approximation and (c) within the frozen potential approximation are
discussed using the concept of layer-resolved quantities. In particular the
comparison between the total energy and the force theorem approach is
presented in some detail because of possible implications for an ab-initio
description of transport properties in multilayer systems.
\end{abstract}

\noindent
(submitted to Phys. Rev. B)\\
Manuscripts available from: pw@eecws1.tuwien.ac.at \\

\newpage
\null
\vskip 2em
\begin{center}
\LARGE{\bf Acid-base catalysis in zeolites from first principles \par} 
\vskip 1.5em
{\large \lineskip .5em \begin{tabular}[t]{c}
Rajiv Shah$^*$, J. D. Gale$\dagger$ and M. C. Payne$^*$ \\
{\it $^*$ Cavendish Laboratory (TCM), Cambridge University, U.K} \\
{\it $\dagger$ Chemistry Department, Imperial College, London, U.K}\\
\end{tabular}\par}
\end{center}
\begin{abstract}
 Zeolite materials are microporous aluminosilicates with various uses,
including acting as important catalysts in many processes. One such
process is the methanol to gasoline reaction, used widely in industry.
This reaction is known to be associated with Br\o nsted acid sites in the
zeolite, formed when Si is substituted by Al in the framework, with an
associated H$^{+}$ being bound nearby to maintain charge neutrality. 
However it is not clear exactly what role the proton plays in this
reaction. 

Because of the large unit cell (generally 50-300 atoms, depending on the
particular zeolite) of such structures, most ab initio calculations of
these materials have focussed on studying small clusters representing just
a portion of the framework.  However, by choosing the chabazite zeolite
structure, which has only 36 atoms in the primitive unit cell, we have
been able to perform a full periodic ab initio calculation. This has used
density functional theory with a generalised gradient approximation for
the exchange-correlation energy, a plane-wave basis set, and
norm-conserving optimised pseudopotentials. Using these methods we have
examined the geometry and electronic structure of a zeolite acid site, and
considered one way in which a methanol molecule may bind to such a site.
\end{abstract} 

\noindent
(Intl. J. Quant. Chem., in press) \\ 
Preprint available on request from Rajiv Shah: rs133@cam.ac.uk \\

\newpage
\null 
\vskip 2em 
\begin{center} 
\LARGE{\bf The mechanism of tritium diffusion in lithium oxide \par} 
\vskip 1.5em 
{\large \lineskip .5em
\begin{tabular}[t]{c} 
R. Shah, A. De Vita, V. Heine and M. C. Payne \\
{\it Cavendish Laboratory (TCM), University of Cambridge, U.K.}\\
\end{tabular}\par} 
\end{center} 
\begin{abstract} 

Lithium oxide is a possible candidate for a breeder blanket material in
fusion reactors. Tritium is generated in the material, which can be
extracted and fed into the fusion reactor to help sustain the fusion
reaction. Experimental studies have shown the extraction rate is
controlled by diffusion of tritium in the bulk, but the exact mechanism is
not clear. 

Here we present ab initio density functional calculations of the various
diffusion pathways which have been suggested, including the diffusion of
tritium as an interstitial and various vacancy assisted mechanisms. The
activation energy has been calculated for each pathway, and by comparison
with experimental results we have deduced which mechanism is most likely.
This is shown to be a simple two-stage swapping of a lithium and tritium
ion. 
\end{abstract} 

\noindent
(submitted to Phys. Rev. B) \\ 
Preprint available on request from Rajiv Shah: rs133@cam.ac.uk \\

\newpage
\null
\vskip 2em
\begin{center}
\LARGE{\bf Time-dependent density functional theory beyond linear response:\ an
exchange-correlation potential with memory \par}
\vskip 1.5em 
{\large \lineskip .5em \begin{tabular}[t]{c}
M. B\"unner$^a$, John. F. Dobson$^b$, and E.K.U. Gross$^{c,b}$\\
{\it $^a$Physikalisches Institut der Universit\"at Bayreuth,}\\
{\it D-95440 Bayreuth, Germany}\\
{\it $^b$School of Science, Griffith University,}\\
{\it Nathan, Queensland 4111, Australia}\\
{\it $^c$Institut f\"ur Theoretische Physik der Universit\"at W\"urzburg,} \\
{\it Am Hubland, D-97974 W\"urzburg, Germany}\\
\end{tabular}\par}
\end{center}
\begin{abstract}
We propose a memory form of exchange-correlation (xc) potential $v_{xc}({\bf %
r},t)$ for time-dependent interacting many-particle systems. Unlike previous
memory-xc potentials, our $v_{xc}$ is not limited to the linear response
regime. The proposed form of $v_{xc}$ has the maximum degree of spatial
locality allowable in view of constraints imposed by Galilean invariance and
by the Harmonic Potential Theorem. For the case of the inhomogeneous
electron gas, we give an explicit prescription for $v_{xc}$ based solely on
an existing parametrization of the linear xc response kernel $f_{xc}^{\hom
}(n,\omega )$ of the uniform gas.
\end{abstract}

\noindent
(submitted to Physical Review Letters)\\
Latex version can be obtained from: gross@physik.uni-wuerzburg.de\\

\newpage
\null
\vskip 2em
\begin{center}
\LARGE{\bf Conventional Quantum Chemical Correlation Energy 
versus Density-Functional Correlation Energy \par}
\vskip 1.5em
{\large \lineskip .5em \begin{tabular}[t]{c}
E.K.U. Gross, M. Petersilka and T. Grabo\\
{\it Institut f\"ur Theoretische Physik, Universit\"at W\"urzburg}\\
{\it Am Hubland, D-97074 W\"urzburg, Germany}\\
\end{tabular}\par}
\end{center}
\vspace*{1cm}
\begin{abstract}
We analyze the difference between the correlation energy as defined
within the conventional quantum chemistry framework and its namesake in
density-functional theory. Both quantities are rigorously defined concepts;
one finds that $E_c^{QC} \geq E_c^{DFT}$. We give numerical and analytical
arguments suggesting that the numerical difference between the two rigorous
quantities is small. Finally, approximate density functional correlation
energies resulting from some popular correlation energy functionals are
compared with the conventional quantum chemistry values.
\end{abstract}

\noindent
(To appear in: \\
``Density Functional Methods in Chemistry'', ACS series, 1996)\\
Latex version can be obtained from: ptrslka@physik.uni-wuerzburg.de \\

\newpage
\null
\vskip 2em
\begin{center}
\LARGE{\bf Scaling and virial theorems in current-density-functional theory \par}
\vskip 1.5em
{\large \lineskip .5em \begin{tabular}[t]{c}
    S. Erhard and E. K. U. Gross \\
{\it Institut f\"ur Theoretische Physik, Universit\"at W\"urzburg,}\\
{\it Am Hubland, D-97074 W\"urzburg, Germany} \\
\end{tabular}\par}
\end{center}
\begin{abstract}
%\renewcommand{\r}{{\bf r}}
\newcommand{\n}{{\bf \nabla}}
\renewcommand{\j}{{\bf j}}
Starting from a constrained-search formulation of 
current-density-func\-tio\-nal 
theory, we obtain rigorous scaling and virial relations for the kinetic,
exchange and correlation energy functionals of electronic systems in strong
magnetic fields. With the scaled density
$n_\lambda({\bf r}) \equiv \lambda^3n(\lambda{\bf r})$  and the scaled {\em paramagnetic}
current density $\j_\lambda({\bf r}) \equiv \lambda^4\j(\lambda{\bf r})$
the exchange energy functional $E_{\rm x}[n,\j]$ and the non-interacting 
kinetic energy functional 
$T_s[n,\j]$ satisfy $E_{\rm x}[n_\lambda ,\j_\lambda ] 
= \lambda  E_{\rm x}[n,\j]  $ and $T_s[n_\lambda ,\j_\lambda ] 
= \lambda ^2 T_s[n,\j]$.
The correlation energy functional satisfies the inequalities
$E_{\rm c}[n_\lambda ,\j_\lambda ] \, {<\atop>} \, \lambda  E_{\rm c}[n,\j]$
for $\lambda \, {<\atop>} \, 1$. 
More\-over, any homogeneously scaling functional 
$Q[n_\lambda,\j_\lambda]= \lambda^k Q[n,\j]$ obeys the rela\-tion:\\
$kQ + \int\!{\rm d}^3r\,n({\bf r}) ({\bf r}\n) [\delta Q/\delta n({\bf r})] +
\int\!{\rm d}^3r\,{\j}{({\bf r})} ({\bf r}{\n-1}) [{\delta} Q/{\delta} {\j}{({\bf r})}]  = 0$.\\
This relation, combined with the appropriate virial theorem in the presence
of an external magnetic field, yields the following central theorems:\\
$E_{\rm x} + \int\!{\rm d}^3r\,n({\bf r}) ({\bf r}\n) [\delta E_{\rm x}/\delta n({\bf r})] +
\int\!{\rm d}^3r\,\j({\bf r}) ({\bf r}\n-1) [\delta E_{\rm x}/\delta \j({\bf r})] = 0$ \\
and \\
$E_{\rm c} + \int\!{\rm d}^3r\,n({\bf r}) ({\bf r}\n) [\delta E_{\rm c}/\delta n({\bf r})] +
\int\!{\rm d}^3r\,\j({\bf r}) ({\bf r}\n-1) [\delta E_{\rm c}/\delta \j({\bf r})] = 
-T_{\rm xc}$, \\
where $T_{\rm xc}[n,\j]$ is the difference of the interacting and the 
non-inter\-act\-ing kinetic energy functionals.
\end{abstract}

\noindent
(accepted by Physical Review A Rapid Communication)\\
Latex version can be obtained from: erhard@physik.uni-wuerzburg.de\\

\newpage
\null
\vskip 2em
\begin{center}
\LARGE{\bf Density-Functional Theory for Triplet Superconductors \par}
\vskip 1.5em
{\large \lineskip .5em \begin{tabular}[t]{c}
K. Capelle and E.K.U. Gross \\
{\it Institut f\"ur Theoretische Physik, Universit\"at W\"urzburg} \\
{\it Am Hubland, D-97074 W\"urzburg, Germany} \\
\end{tabular}\par}
\end{center}
\vspace*{1cm}
\begin{abstract}
The density-functional theory of superconductivity is extended to triplet superconductors and 
superfluid helium 3. We prove a Hohenberg-Kohn-type theorem for these systems and derive 
effective single-particle equations.  The latter include exchange and correlations in a formally 
exact way and allow the treatment of both electronic and phonon-induced superconductivity.
The relation of this approach to the Bogolu\-bov-de Gennes mean-field theory and to phenomenological 
theories based on Ginzburg-Landau functionals is discussed.
\end{abstract}

\noindent
(submitted to International Journal of Quantum Chemistry)\\
Latex version can be obtained from: capelle@physik.uni-wuerzburg.de \\

\newpage
\null
\vskip 2em
\begin{center}
\Large{\bf A First Principles Investigation of Vacancy Oxygen defects in Si \par}
\vskip 1.5em
{\large \lineskip .5em \begin{tabular}[t]{c}
C. P. Ewels, R. Jones \\
{\it Department of Physics, University of Exeter,} \\
{\it Exeter, EX4 4QL, UK} \\
S. \"Oberg\\
{\it Department of Mathematics, University of Lule\aa,} \\
{\it Lule\aa, S95187, Sweden}\\
\end{tabular}\par}
\end{center}
\begin{abstract}
{\it Ab initio} techniques are used to study vacancy oxygen complexes
in Si.  Substitutional oxygen is found to be an off-site defect in
agreement with experiment.  It possesses one high frequency O-related
LVM at 788 cm$^{-1}$. The VO$_2$ defect has $D_{2d}$ symmetry and only
one O-related high frequency IR active mode at 807 cm$^{-1}$.  The
VO$_3$ defect has three high frequency IR active modes.  The V$_2$O
defect has one such LVM. These results provide strong support for the
assignment of the 889 cm$^{-1}$ (300 K) local vibrational mode to
VO$_2$.
\end{abstract}

\noindent
{\bf Keywords:} {\it ab initio} theory, silicon, oxygen, vacancy,
defect.\\

\noindent
(Paper presented this year at the ICDS '95 International Defects Conference 
in Sendai, Japan)\\
\ni Paper is available by e-mail from: \verb+ jones@excc.ex.ac.uk + \\
\ni or alternatively from the World Wide Web at: \\
\ni \verb+ http://newton.ex.ac.uk/jones/papers/sendai.html +\\


\newpage
\null
\vskip 2em
\begin{center}
\Large{\bf A First Principles Study Of Ni Defects In Synthetic Diamond \par}
\vskip 1.5em
{\large \lineskip .5em \begin{tabular}[t]{c}
J. Goss, R. Jones\\
{\it Department of Physics, University of Exeter,} \\
{\it Exeter, EX4 4QL, UK} \\
A. Resende\\
{\it Departamento de Fisica, Universidade de Aveiro,} \\
{\it  3800 Aveiro, Portugal}\\
S. \"Oberg\\
{\it Department of Mathematics, University of Lule\aa,} \\
{\it  Lule\aa, S95187, Sweden}\\
P. R. Briddon\\
{\it Department of Physics, University of Newcastle,} \\
{\it Newcastle, NE1 7RU, UK}\\
\end{tabular}\par}
\end{center}
\begin{abstract}
We present the results of {\it ab initio} local density functional
cluster calculations, using the AIMPRO program, performed on
substitutional and interstitial Ni defects, and complexes of Ni with N
and B.

Interstitial Ni$^+$ (S=1/2) remains on the $T_d$ site in contrast with
results of previous calculations [2,3].  Ni$^{++}$ (S=1) should be
present in boron rich diamonds but as it has not been observed, we
speculate that only substitutional Ni defects are found in diamond.

Substitutional Ni$^-$ is found to be more stable, by around 1 eV, in
an S=3/2 configuration than that with S=1/2, in agreement with
experiment.  The defect gives rise to a number of gap levels and could
give rise to an optical transition around 1.4 and 2.6 eV.

Substitutional Ni$^+$ (S=1/2) is found to have $T_d$ symmetry at
finite temperatures.  A Jahn-Teller effect is expected for this
defect, which could be trigonal, but is likely to be only observable
at cryogenic temperatures.  This defect is proposed for the NIRIM-1
EPR centre [5] that is found to have $T_d$ symmetry at 25 K, but
$C_3v$ at 4 K, and observed in samples with low N concentration.

The trigonal Ni$_s^+$B$_s^-$ (S=1/2) complex is found to be stable.
It gives rise to a number of states deep in the band-gap and an
optical $e-a_1$ transition around 1.1 eV.  No hyperfine splitting due
to B is expected.  We propose this defect as responsible for the
NIRIM-2 and the 1.4 eV optical centres.

The trigonal Ni$_s^-$N$_s^+$ (S=1/2) complex is stable and gives rise
to an optical transition around 1.4 eV and may explain the 1.693 eV
optical line seen in annealed N rich diamonds containing Ni [4].
\end{abstract}

\noindent
1) Gippius, A A,Collins, A T, and Kazarian, S A, Mater.  Sci. Forum
{\bf 143-147}, 41 (1994).

2) Paslovsky, L, and Lowther, J E, J. Phys.: Cond. Matter 4, 775
(1992).

3) Yang Jinlong, Zhang Manhong, and Wang Kelin, Phys. Rev. B {\bf 49},
15525 (1994).

4) Lawson, S C, and Kanda, H, J.\ App. Phys. {\bf 73}, 3967 (1993).

5) Isoya, J, Kanda, H, and Uchida, Y, Phys. Rev. B {\bf 42}, 9843 (1990). \\

\noindent
{\bf Keywords:} {\it ab initio} theory, diamond, TM, nickel \\

\noindent
(Paper presented this year at the ICDS '95 International Defects Conference 
in Sendai, Japan)\\
\ni Paper is available by e-mail from: \verb+ jones@excc.ex.ac.uk + \\
\ni or alternatively from the World Wide Web at: \\
\ni \verb+ http://newton.ex.ac.uk/jones/papers/sendai.html +\\

\newpage
\null
\vskip 2em
\begin{center}
\Large{\bf Interstitial H and the dissociation of C-H defects in GaAs \par}
\vskip 1.5em
{\large \lineskip .5em \begin{tabular}[t]{c}
S. J. Breuer, R. Jones \\
{\it Department of Physics, University of Exeter,} \\
{\it Exeter, EX4 4QL, UK} \\
S. \"Oberg \\
{\it Department of Mathematics, University of Lule\aa,} \\
{\it Lule\aa, S95187, Sweden} \\
P. R. Briddon \\
{\it Department of Physics, University of Newcastle,} \\
{\it Newcastle, NE1 7RU, UK}\\
\end{tabular}\par}
\end{center}
\begin{abstract}
Local--density--functional calculations using a real--space cluster
approach are used to model interstitial hydrogen in GaAs and the
dissociation of the C--H and C--H$^-$ complexes.  The equilibrium site
is found to be on a Ga--As bond axis for H$^0$ and H$^+$ and at a Ga
anti--bonding site for H$^-$.  It is also shown that a H$_2$ molecule
is stable in interstitial space and has a lower energy than the two
possible H$_2^*$ defects and than widely separated single interstitial
hydrogen atoms.  The study of hydrogen in pure GaAs also yields the
result that interstitial hydrogen is a negative--U defect.  It is
found that the energy barrier to the dissociation of the C--H complex
is 1.8 eV, but that this is reduced to 0.9 eV for C--H$^-$.
Comparison is made with recent experimental results and implications
for current containing devices are discussed.
\end{abstract}

\noindent
{\bf Keywords:} C-H, dissocation, interstitial hydrogen, GaAs, {\it ab
initio} \\

\noindent
(Paper presented this year at the ICDS '95 International Defects Conference 
in Sendai, Japan)\\
\ni Paper is available by e-mail from: \verb+ jones@excc.ex.ac.uk + \\
\ni or alternatively from the World Wide Web at: \\
\ni \verb+ http://newton.ex.ac.uk/jones/papers/sendai.html +\\

\newpage
\null
\vskip 2em
\begin{center}
\Large{\bf Peculiarities of interstitial carbon and di-carbon defects in Si \par}
\vskip 1.5em
{\large \lineskip .5em \begin{tabular}[t]{c}
R. Jones, P. Leary \\
{\it Department of Physics, University of Exeter,} \\
{\it Exeter, EX4 4QL, UK} \\
S. \"Oberg \\
{\it Department of Mathematics, University of Lule\aa,} \\
{\it Lule\aa, S95187, Sweden} \\
V. Torres \\
{\it Departamento de Fisica, Universidade de Aveiro,} \\
{\it 3800 Aveiro, Portugal} \\
\end{tabular}\par}
\end{center}
\begin{abstract}
The C$_i$ and C$_s$-C$_i$ defects in Si exhibit several unexplained
properties. In the neutral charge state, the C$_i$ defect possesses
two almost degenerate vibrational modes suggesting a trigonal defect
in disagreement with the $C_{2v}$ symmetry deduced from several
experiments.  The B-form of the second defect is believed to consist
of a Si interstitial, Si$_i$, located near a BC site between two C$_s$
atoms, in apparent conflict with the results of PL experiments which
show that the C-related vibrational modes are decoupled.  The
structure and vibrational modes of both defects are analysed using LDF
cluster theory.  The degeneracy of the modes of C$_i$ is attributed to
an almost $D_{3h}$ structure, with a 3-fold axis along [01\=1]. The
modes of the di-carbon interstitial lead to a resolution of the long
standing problem concerning the almost zero-shifts due to mixed
isotopes in the 580 and 543 cm$^{-1}$ local modes observed in PL
studies.
\end{abstract}

\noindent
{\bf Keywords:} {\it ab initio} theory, carbon interstitials, silicon,
vibrational modes\\

\noindent
(Paper presented this year at the ICDS '95 International Defects Conference 
in Sendai, Japan)\\
\ni Paper is available by e-mail from: \verb+ jones@excc.ex.ac.uk + \\
\ni or alternatively from the World Wide Web at: \\
\ni \verb+ http://newton.ex.ac.uk/jones/papers/sendai.html +\\

\newpage
\null
\vskip 2em
\begin{center}
\Large{\bf Theory of the NiH$_2$ Complex in Si and the CuH$_2$ Complex in GaAs \par}
\vskip 1.5em
{\large \lineskip .5em \begin{tabular}[t]{c}
R. Jones, J. Goss \\
{\it Department of Physics, University of Exeter,} \\
{\it Exeter, EX4 4QL, UK} \\
S. \"Oberg \\
{\it Department of Mathematics, University of Lule\aa,} \\
{\it Lule\aa, S95187, Sweden} \\
P. R. Briddon \\
{\it Department of Physics, University of Newcastle,} \\
{\it Newcastle, NE1 7RU, UK}\\
\end{tabular}\par}
\end{center}
\begin{abstract}
Spin-polarised local density functional cluster calculations are
carried out on substitutional Ni in Si and Cu in GaAs along with
TM-H$_2$ complexes. The Jahn-Teller distortion for Ni in Si leads to a
slight displacement along $\langle$100$\rangle$ in agreement with EPR
experiments.  Several models of NiH$_2$ are investigated and it is
shown that one, with H located at anti-bonding sites to two of the Si
neighbours of Ni, has the lowest energy and possesses H related local
vibrational modes close to those reported for Pt-H$_2$.  A similar
structure is found for CuH$_2$ in GaAs.  The electronic properties of
the complexes are described in terms of the vacancy model of TM
impurities.
\end{abstract}

\noindent
(Paper presented this year at the ICDS '95 International Defects Conference 
in Sendai, Japan)\\
\ni Paper is available by e-mail from: \verb+ jones@excc.ex.ac.uk + \\
\ni or alternatively from the World Wide Web at: \\
\ni \verb+ http://newton.ex.ac.uk/jones/papers/sendai.html +\\

\newpage
\null
\vskip 2em
\begin{center}
\Large{\bf Theory of Si delta-doped GaAs \par}
\vskip 1.5em
{\large \lineskip .5em \begin{tabular}[t]{c}
R. Jones \\
{\it Department of Physics, University of Exeter,} \\
{\it Exeter, EX4 4QL, UK} \\
S. \"Oberg \\
{\it Department of Mathematics, University of Lule\aa,} \\
{\it Lule\aa, S95187, Sweden} \\
\end{tabular}\par}
\end{center}
\begin{abstract}
The insulating property of approx.  0.5 ML Si delta-doped GaAs is
attributed to extended defects of self-compensated Si$_{Ga}$-Si$_{As}$
pairs.  The longitudinal and transverse optical modes of a chain and
bilayer made from these pairs are calculated using first principles
methods.  The chain possesses a Raman active longitudinal mode at 470
cm$^{-1}$ which drops to 465 cm$^{-1}$, when the chain is depleted of
Si, and increases to about 500 cm$^{-1}$ for Si rafts, ie (110) chains
cross-linked by adjacent (1\=10) chains. These results are in good
agreement with recent Raman scattering studies which show a band
increasing from 470 to 490 cm$^{-1}$ when [Si] is increased from 0.5
ML to 2 ML.  The theory also predicts three transverse branches, as
well as a longitudinal resonant band, none of which have been reported
so far.
\end{abstract}

\noindent
{\bf Keywords:} Si-delta doping, Si-Si chains, Raman scattering, {\it
ab initio} theory \\

\noindent
(Paper presented this year at the ICDS '95 International Defects Conference 
in Sendai, Japan)\\
\ni Paper is available by e-mail from: \verb+ jones@excc.ex.ac.uk + \\
\ni or alternatively from the World Wide Web at: \\
\ni \verb+ http://newton.ex.ac.uk/jones/papers/sendai.html +\\

\newpage
\null
\vskip 2em
\begin{center}
\Large{\bf Vacancy- and acceptor- H complexes in InP \par}
\vskip 1.5em
{\large \lineskip .5em \begin{tabular}[t]{c}
C. Ewels, R. Jones \\
{\it Department of Physics, University of Exeter,} \\
{\it Exeter, EX4 4QL, UK} \\
S. \"Oberg \\
{\it Department of Mathematics, University of Lule\aa,} \\
{\it Lule\aa, S95187, Sweden} \\
B. Pajot \\
{\it Groupe de Physique des Solides, Universite Paris VII,}\\
{\it 2 place Jussieu, 75251 PARIS, Cedex 05, France} \\
P. R. Briddon \\
{\it Department of Physics, University of Newcastle,} \\
{\it Newcastle, NE1 7RU, UK}\\
\end{tabular}\par}
\end{center}
\begin{abstract}
It has been suggested that iron in InP is compensated by a donor,
related to the 2316 cm$^{-1}$ local vibrational mode and previously
assigned to VH$_4$. Using {\it ab initio} local density functional
cluster calculations, we find that the fully hydrogenated indium
vacancy, VH$_4$, acts as a single shallow donor, and has a triplet
vibrational mode at around this value, consistent with this
assignment. We also analyse the other hydrogenated vacancies VH$_n,
n=1,3$ and determine their structure, vibrational modes, and charge
states. Substitutional Group II impurities also act as acceptors in
InP, but can be passivated by hydrogen. We investigate the passivation
of beryllium by hydrogen and find that the hydrogen sits in a bond
centred site and is bonded to its phosphorus neighbour. Its calculated
vibrational modes are in good agreement with experiment.
\end{abstract}

\noindent
{\bf Keywords:} {\it ab initio} theory, InP, donor, vacancy, hydrogen,
defect, beryllium

\noindent
(Paper presented this year at the ICDS '95 International Defects Conference 
in Sendai, Japan)\\
\ni Paper is available by e-mail from: \verb+ jones@excc.ex.ac.uk + \\
\ni or alternatively from the World Wide Web at: \\
\ni \verb+ http://newton.ex.ac.uk/jones/papers/sendai.html +\\

\newpage
\null
\vskip 2em
\begin{center}
\Large{\bf {\it Ab initio} full charge-density study of the atomic volume of \\
\mbox{{\boldmath $\alpha$}}-phase Fr, Ra, Ac, Th, Pa, U, Np, and Pu \par}
\vskip 1.5em
{\large \lineskip .5em \begin{tabular}[t]{c}
L. Vitos$^*$, J. Koll\'ar$^*$, and H. L. Skriver$^+$ \\
{\it $^*$Research Institute for Solid State Physics,}\\
{\it H-1525 Budapest, P.O.Box 49, Hungary}\\
{\it $^+$Center for Atomic-scale Materials Physics and Physics Department,} \\
{\it Technical University of Denmark, DK-2800 Lyngby, Denmark} \\
\end{tabular}\par}
\end{center}
\begin{abstract}
We have used a full charge density technique based on the linear muffin-tin
orbitals method in first-principles calculations of the atomic volumes of
the light actinides including Fr, Ra, and Ac in their low temperature
crystallographic phases. The small deviations between the theoretical and
experimental values along the series support the picture of itinerant $5f$
electronic states in Th to Pu. The increased deviation between theory and
experiment found in Np and Pu may be an indication of correlation effects
not included in present day local density approximations.
\end{abstract}

\noindent
(submitted to Phys. Rev. B)\\
Manuscripts available from: jk@power.szfki.kfki.hu \\

\newpage
\null
\vskip 2em
\begin{center}
%--------------------------------------------------------
\Large{\bf
Interlayer magnetic coupling: the torque method \par}
\vskip1.5em
{\large \lineskip .5em \begin{tabular}[t]{c}
V. Drchal$^{\; a,b}$, J. Kudrnovsk\'y$^{\; a,b}$,
I. Turek$^{\; c}$, and P. Weinberger$^{\; b}$ \\

$^a$ {\it Institute of Physics, Academy of Sciences} \\
{\it of the Czech Republic, CZ-180 40 Praha 8, Czech Republic}\\
%\\
$^b$ {\it Institute for Technical Electrochemistry,} \\
{\it Technical University, A-1060 Vienna, Austria } \\
$^c$ {\it Institute of Physics of Materials,
Academy of Sciences}\\
{\it of the Czech Republic, CZ-616 62 Brno, Czech Republic }
\end{tabular}\par}
\end{center}
\begin{abstract}
We present ab-initio calculations of the interlayer exchange coupling
between two, in general non-collinearly aligned  magnetic slabs
embedded in a non-magnetic spacer.
Based on a surface Green's function formalism, two equivalent but
formally and physically different approaches are examined and
discussed.
For the Co/Cu/Co(001) system we demonstrate the usefulness of the
concept of infinitesimal rotations in order to calculate the
coupling for a finite relative angle ${\theta}$, in particular for
${\theta=\pi}$, between the corresponding spin directions in the
magnetic slabs.
The temperature and layer dependence of the interlayer exchange
coupling is examined and the possibility for non-collinear
coupling is investigated.
\end{abstract}

\noindent
(submitted to Phys. Rev. B) \\
Preprint can be obtained from V. Drchal (vd@eecws7.tuwien.ac.at)\\

\newpage
\null
\vskip 2em
\begin{center}
%--------------------------------------------------------
\Large{\bf
Interlayer magnetic coupling: effect of disorder in spacer \par}
\vskip1.5em
{\large \lineskip .5em \begin{tabular}[t]{c}
 J. Kudrnovsk\'y$^{\; a,c}$, V. Drchal$^{\; a,c}$,
I. Turek$^{\; b}$, M. \v Sob$^{\; b}$ and P. Weinberger$^{\; c}$ \\

$^a$ {\it Institute of Physics,
Academy of Sciences} \\{\it of the Czech Republic,
CZ-180 40 Praha 8, Czech Republic} \\
$^b$ {\it Institute of Physics of Materials,
Academy of Sciences}\\{\it of the Czech Republic,
CZ-616 62 Brno, Czech Republic } \\
$^c$ {\it Institute for Technical Electrochemistry,}  \\
{\it Technical University, A-1060 Vienna, Austria }\\
\end{tabular}\par}
\end{center}
\begin{abstract}
 The influence of spacer randomness  on the
periods and the amplitudes of the oscillations of exchange coupling in
magnetic multilayers is studied from first principles. The effect of
disorder is treated within the coherent potential approximation. As a
case
study, results obtained for trilayers Co/Cu$_{1-x}$M$_x$/Co(001),
where
M=Ni, Pd, and Zn, are presented and discussed.
\end{abstract}
 
\noindent
(submitted to JMMM) \\
Preprint can be obtained from J. Kudrnovsk\'y (jk@eecws7.tuwien.ac.at)\\

\newpage
\null
\vskip 2em
\begin{center}
\Large{\bf 3d semicore-states in ZnSe, GaAs and Ge \par}
\vskip 1.5em
{\large \lineskip .5em \begin{tabular}[t]{c}
F. Aryasetiawan and O. Gunnarsson \\
{\it Max-Planck-Institut f\"ur Festk\"orperforschung,}\\
{\it D-70569 Stuttgart, FRG}\\
\end{tabular}\par}
\end{center}
\begin{abstract}
We present the self-energy corrections for the semi-core states in ZnSe, GaAs
and Ge within the GW approximation. Good agreement with experiment is
found.  To study the error trend in the local density eigenvalues which
increases from ZnSe to Ge, we have also performed Slater transition
state calculations.  This increasing error can be understood in terms
of the change of the occupation numbers of the orbitals and the Coulomb
energies.  We derived a simple formula to describe this increasing error.
\end{abstract}

\noindent
(submitted to Phys. Rev. B)\\
Manuscripts available from: gunnar@radix3.mpi-stuttgart.mpg.de \\


\newpage
\null
\vskip 2em
\begin{center}
\Large{\bf On the idea of a new quantum number in the cluster problem \par}
\vskip 1.5em
{\large \lineskip .5em \begin{tabular}[t]{c}
Erik Koch \\
{\it Max-Planck-Institut f\"ur Festk\"orperforschung,}\\
{\it D-70569 Stuttgart, FRG}\\
\end{tabular}\par}
\end{center}
\begin{abstract}
It has recently been suggested that an exactly solvable problem characterized
by a new quantum number may underlie the electronic shell structure
observed in the mass spectra of medium-sized sodium clusters. We
investigate whether the conjectured quantum number $3n+l$ bears a
similarity to the quantum numbers $n+l$ and $2n+l$, which characterize 
the hydrogen problem and the isotropic harmonic oscillator in three 
dimensions.
\end{abstract}

\noindent
(submitted to Phys. Rev. B)\\
Manuscripts available from: koch@radix5.mpi-stuttgart.mpg.de \\


\newpage
\null
\vskip 2em
\begin{center}
\Large{\bf Supershells in metal clusters:\\
       Self-consistent calculations and their semiclassical interpretation \par}
\vskip 1.5em
{\large \lineskip .5em \begin{tabular}[t]{c}
Erik Koch \\
{\it Max-Planck-Institut f\"ur Festk\"orperforschung,}\\
{\it D-70569 Stuttgart, FRG}\\
\end{tabular}\par}
\end{center}
\begin{abstract}
To understand the electronic shell- and supershell-structure in large
metal clusters we have performed self-consistent calculations in the
homogeneous, spherical jellium model for a variety of different
materials.  A scaling analysis of the results reveals a surprisingly
simple dependence of the supershells on the jellium density.
It is shown how this can be understood in the framework of a
periodic-orbit-expansion by analytically extending the well-known
semiclassical treatment of a spherical cavity to more realistic
potentials.  
\end{abstract}

\noindent
(submitted to Phys. Rev. Lett.)\\
Manuscripts available from: koch@radix5.mpi-stuttgart.mpg.de \\

\newpage
\null
\vskip 2em
\begin{center}
\Large{\bf Band Theory for Electronic and Magnetic Properties of
$\alpha$-$ \bf Fe_2O_3$ \par}
\vskip 1.5em
{\large \lineskip .5em \begin{tabular}[t]{c}
L. M. Sandratskii, M. Uhl and J. K\"ubler\\
{\it Institut f\"ur Festk\"orperphysik, Technische Hochschule} \\ 
{\it D-64289 Darmstadt, Germany} \\
\end{tabular}\par}
\end{center}
\begin{abstract}
We report results of calculations that explain in the
itinerant-electron picture magnetic and electronic
properties of hematite,  $\alpha$-$ \rm Fe_2O_3$. For this
we use the local approximation to spin-density functional theory
and the ASW method incorporating spin-orbit coupling and
noncollinear moment arrangements.
The insulating character of the compound
is obtained correctly and features in the density of
states connected with Fe - O hybridization correlate well with
experimental features seen in direct and inverse photoemission
intensities. The total energy correctly predicts the experimentally
observed magnetic order of the ground state, and, using total energies
of different magnetic configurations, we can give a rough estimate
of the N\'eel temperature. We also obtain a state showing weak
ferromagnetism . The rate of change is calculated for the decrease
of the insulating gap when an external magnetic field is applied.
\end{abstract}

\noindent
(submitted to J. Physics Cond. Matter)\\
Revtex version can be obtained from: dg5m@mad1.fkp.physik.th-darmstadt.de (L. Sandratskii)\\
 
\noindent
This work has benefited from collaborations within the EU Human
Capital and Mobility Network on
{\it "Ab initio (from electronic structure) calculation of complex
processes in materials"} (contract: \\ ERBCHRXCT930369).\\

\newpage
\null
\vskip 2em
\begin{center}
\Large{\bf The Fermi Surface of  $\bf UPd_2Al_3$ \par}
\vskip 1.5em
{\large \lineskip .5em \begin{tabular}[t]{c}
K. Kn\"opfle, A. Mavromaras, L. M.  Sandratskii and J.  K\"ubler\\
{\it Institut f\"ur Festk\"orperphysik, Technische Hochschule} \\
{\it D-64289 Darmstadt, Germany} \\
\end{tabular}\par}
\end{center}
\begin{abstract}
The de Haas-van Alphen spectrum of $ \rm UPd_2Al_3$
is calculated and compared with the experimental spectrum
for continuously varying directions of the magnetic field.
The local approximation to spin-density functional
theory is used for the self-consistent calculations treating
the U 5f electrons as itinerant and including spin-orbit coupling.
The amount of f-angular-momentum is obtained and visualized graphically
for each sheet of the Fermi surface. The band-decomposed spin
susceptibility, $\chi_0$, is calculated for the states at
the Fermi surface and the anisotropy of $\chi_0$ is discussed.
\end{abstract}

\noindent
(submitted to J. Physics Cond. Matter)\\
Revtex version can be obtained from:
 dg5m@mad1.fkp.physik.th-darmstadt.de (L. Sandratskii)\\

\noindent
This work has benefited from collaborations within the EU Human
Capital and Mobility Network on
{\it "Ab initio (from electronic structure) calculation of complex
processes in materials"} (contract: \\ ERBCHRXCT930369).\\


\newpage
\null
\vskip 2em
\begin{center}
\Large{\bf Spin and orbital polarized relativistic multiple scattering theory \par}
\vskip 1.5em
{\large \lineskip .5em \begin{tabular}[t]{c}
H. Ebert and Marco Battocletti \\
{\it Institute for Physical Chemistry, University of Munich,} \\
{\it Theresienstr. 37, D-80333 M\"{u}nchen, Germany}\\
\end{tabular}\par}
\end{center}
\begin{abstract}
A scheme is presented that allows to incorporate the orbital polarization (OP)
mechanism into the Dirac equation for spin-polarized systems.
This allows in
a straightforward way to extend any band structure method accordingly
-- including
those based on multiple scattering theory.
The corresponding spin and orbital polarized relativistic version
of the Korringa-Kohn-Rostoker-Coherent Potential Approximation
(KKR-CPA) method has been implemented and applied to disordered
bcc-\mbox{Fe$_{x}$Co$_{1-x}$} alloys.
Results for the spin and orbital magnetic moments as well as the
spin-orbit coupling induced hyperfine fields are presented and discussed.
\end{abstract}

\noindent
(submitted to Phys. Rev. B)\\
A postscript version of the manuscripts available upon request from: \\
H.\ Ebert (he@gaia.phys.chemie.uni-muenchen.de) \\


\newpage
\null
\vskip 2em
\begin{center}
\Large{\bf Crucial role of the Lattice Distortion in
the Magnetism of LaMnO$_3$ \par}
\vskip 1.5em
{\large \lineskip .5em \begin{tabular}[t]{c}
Igor Solovyev$^1$, Noriaki Hamada$^1$ and Kiyoyuki Terakura$^2$ \\
{\it $^1$ Joint Research Center for Atom Technology,}\\
{\it Angstrom Technology Partnership,}\\
{\it 1-1-4 Higashi, Tsukuba, Ibaraki 305, Japan}\\
{\it $^2$ Joint Research Center for Atom Technology,}\\
{\it National Institute for Advanced Interdisciplinary Research,}\\
{\it 1-1-4 Higashi, Tsukuba, Ibaraki 305, Japan}\\
\end{tabular}\par}
\end{center}
\begin{abstract}
The stability of the A-type antiferromagnetic order
and canted magnetic structure
of LaMnO$_3$ perovskite
is explained in the itinerant-electron picture based on the
local-spin-density approximation.
The lattice distortion plays a crucial role and determines
behavior of both single-ion anisotropy, anisotropic ${\it and}$
isotropic exchange interactions in this compound.
\end{abstract}

\noindent
(submitted to Phys. Rev. B)\\
Manuscripts available from: igor@jrcat.or.jp \\


\newpage
\null
%\vskip 2em
\begin{center}
\Large{\bf A first-principles study of the magnetic hyperfine field in Fe and Co
multilayers \par}
\vskip 0.5em
{\large \lineskip .5em \begin{tabular}[t]{c}
G. Y. Guo$^{a}$ and H. Ebert$^{b}$, \\
{\it $^{a}$Daresbury Laboratory, Daresbury, Warrington WA4 4AD, UK} \\
{\it $^{b}$Institute for Physical Chemistry, University of Muenchen,} \\
{\it Theresienstrasse 37, D-80333 Muenchen, FRG} \\
\end{tabular}\par}
\end{center}
\begin{abstract}
We present {\it ab initio} calculations of
the magnetic hyperfine field and magnetic moments in several Fe and Co
multilayers [$Fe(Co)_{2}/Cu_{6} fcc-(001), FeCu(Ag)_{5}/fcc-(001),
bcc-Fe/fcc-Ag_{5} (001), bcc-Fe_{n}/fcc-Au_{5} (001) (n = 1, 3, 7),
Co_{k}/Pd_{l} fcc-(111) (k (l) = 1 (5), 2 (4), 3 (3)) and
Co_{2}/Pt_{m} fcc-(111) (m = 1, 4, 7)$] as well as in bcc
Fe and fcc (hcp, bcc) Co. 
The first-principles spin-polarized, relativistic linear muffin-tin orbital
(SPRLMTO) method is used. Therefore, both the orbital and magnetic
dipole contributions as well as the conventional Fermi contact term are
calculated. Calculations have been performed for both in-plane and
perpendicular magnetizations. The calculated hyperfine field and its
variation with crystalline structure and magnetization direction in both
Fe and Co are in reasonable agreement (within 10 \%) with experiments.
The hyperfine field of Fe (Co) in the interface monolayers in the
magnetic multilayers is found to be substantially reduced compared with
that in the corresponding bulk metal, in strong contrast to the highly
enhanced magnetic moments in the same monolayers. It is argued that the
magnetic dipole and orbital contributions to the hyperfine field are
approximately proportional to the so-called magnetic dipole moment and
the orbital moment, respectively. These linear relations are then
demonstrated to hold rather well by using the calculated non-s electron
hyperfine fields, orbital and magnetic dipole moments. Unlike in the
bulk metals and alloys, the magnetic dipole moment in the multilayers is
predicted to be comparable to the orbital moment and as a result, the
magnetic dipole contribution to the hyperfine field is large. The
anisotropy in the hyperfine field is found to be very pronounced and to
be strongly connected with the large anisotropy in the orbital moment
and magnetic dipole moment. The induced magnetic moments and hyperfine
fields in the nonmagnetic spacer layers are also calculated. The results
for the multilayers are compared with available experiments and previous
nonrelativistic calculations.
\end{abstract}

\noindent
(Phys. Rev. B (in press)); Preprints available from: g.y.guo@dl.ac.uk \\
\noindent
This work has benefited from collaborations within the EU Human Capital and
Mobility Network on "{\it Ab initio (from electronic structure) calculation of
complex processes in materials}" (contract: ERBCHRXCT930369).

\newpage
\null
\vskip 2em
\begin{center}
\Large{\bf Non-collinear magnetism in Al-Mn topologically disordered
systems \par}
\vskip 1.5em
{\large \lineskip .5em \begin{tabular}[t]{c}
A. V. Smirnov \\
        {\it Institut f\"ur Festk\"orperphysik, Technische Hochschule, }\\
        {\it D-64289 Darmstadt, Germany }\\
A. M. Bratkovsky \\
        {\it
        Oxford University, Department of Materials, }\\
        {\it Oxford OX1 3PH, England }\\
\end{tabular}\par}
\end{center}

\begin{abstract}
We have performed the first ab-initio calculations of a possible complex
non-collinear magnetic structure in aluminium-rich Al-Mn liquids
within the real-space tight-binding LMTO method.
In our previous work we have predicted  the existence of large magnetic 
moments in Al-Mn liquids [A.M. Bratkovsky, A.V. Smirnov, D. N. Manh,
and A. Pasturel, Phys. Rev. B {\bf 52}, 3056 (1995)] which has been very recently
confirmed experimentally. Our present calculations show that
there is a strong tendency for the moments on Mn to have a
non-collinear (random) order retaining their large value of about
3~$\mu_B$. The d-electrons on Mn demonstrate a pronounced
non-rigid band behaviour which cannot be reproduced within 
a simple Stoner picture.
The origin of the magnetism in these systems is a
topological disorder which drives the moments formation and 
frustrates their directions in the liquid phase.
\end{abstract}

\noindent
(submitted to Europhys. Lett.) \\
The manuscripts are available from: alex.bratkovsky@materials.oxford.ac.uk \\

\newpage
\null
\vskip 2em 
\begin{center} 
\Large{\bf 
Adlayer core-level shifts of admetal monolayers on
transition metal substrates and their relation to the
surface chemical reactivity \par} 
\vskip 1.5em 
{\large \lineskip .5em \begin{tabular}[t]{c}
Dieter Hennig\\
{\it Humboldt-Universit\"at zu Berlin, Institut f\"ur Physik}\\
{\it Unter der Linden 6, D-10$\,$099 Berlin, Germany} \\
Maria Veronica Ganduglia-Pirovano and Matthias Scheffler\\
{\it Fritz-Haber-Institut der Max-Planck-Gesellschaft} \\
{\it Faradayweg 4-6, D-14195 Berlin-Dahlem, Germany} \\
\end{tabular}\par} 
\end{center} 
\begin{abstract} 
Using density-functional-theory we study the electronic and structural
properties of a monolayer of Cu on the fcc (100) and (111) surfaces of the
late $4d$ transition metals, as well as a monolayer of Pd on Mo bcc(110).
We calculate the ground states of these systems,
as well as the difference of the ionization energies of an adlayer
core electron and a core electron of the clean surface of the adlayer
metal.
The theoretical results are compared to available
experimental data and discussed in a simple physical picture; it is shown why
and how adlayer core-level binding energy shifts
can be used to deduce information on the
adlayer's chemical reactivity. 
\end{abstract} 

\noindent
(Submitted to Phys. Rev. B) \\ 
REVTEX version can be obtained from: \verb|vero@theo23.RZ-Berlin.MPG.DE| \\

\newpage
\null
\vskip 2em 
\begin{center} 
\Large{\bf 
Role of Self-Interaction Effects in the
Geometry Optimization of Small Metal Clusters \par} 
\vskip 1.5em 
{\large \lineskip .5em \begin{tabular}[t]{c}
J. M. Pacheco\\
{\it Departamento de Fisica da Universidade} \\
{\it 3000 Coimbra, Portugal} \\ 
W. Ekardt and W.-D. Sch\"one\\
{\it Fritz-Haber-Institut der Max-Planck-Gesellschaft} \\
{\it Faradayweg 4-6, 14195 Berlin, Germany} \\
\end{tabular}\par} 
\end{center} 
\begin{abstract} 
By combining the Self-Interaction Correction (SIC) with
pseudopotential perturbation theory,
the role of self-interaction errors inherent to the Local Density
Approximation (LDA) to Density Functional Theory is
estimated in the determination of ground state and low energy
isomeric structures of small metallic clusters.
Its application to
neutral sodium clusters with 8 and 20 atoms shows
that the SIC provides
sizeable effects in $Na_8$, leading to a different ordering of the low lying
isomeric states compared with {\it ab-initio} LDA predictions, whereas
for $Na_{20}$, the SIC effects are less pronounced, such that
a quantitative agreement is achieved between the present
method and {\it ab-initio} LDA calculations.
\end{abstract} 

\noindent
(Submitted to Europhys. Lett.) \\ 
Copy can be obtained from: \verb|ekardt@fhi-berlin.mpg.de| \\
\newpage
\null
\vskip 2em 
\begin{center} 
\Large{\bf
The Effective Particle-Hole Interaction and the
       Optical Response of Simple Metal Clusters \par} 
\vskip 1.5em 
{\large \lineskip .5em \begin{tabular}[t]{c} 
W. Ekardt \\
{\it Fritz-Haber-Institut der Max-Planck-Gesellschaft}\\
{\it Faradayweg 4--6, 14\,195 Berlin, Germany}\\
J. M. Pacheco\\
{\it Departamento de Fisica da Universidade}\\
{\it 3000 Coimbra, Portugal} \\
\end{tabular}\par} 
\end{center} 
\begin{abstract} 
Following
Sham and Rice [L. J. Sham, T. M. Rice, Phys. Rev. {\bf 144} (1966) 708]
the correlated motion of particle-hole pairs is studied, starting from
the general two-particle Greens function. In this way we derive a matrix
equation for eigenvalues and wave functions, respectively, of the general
type of collective excitation of a N-particle system. The interplay between
excitons and plasmons is fully described by this new set of equations.  As a
by-product  we obtain - at least {\it a-posteriori} - a justification
for the use of the TDLDA for simple-metal clusters.
\end{abstract} 

\noindent
(Submitted to 
Phys. Rev. B) \\ 
Copy can be obtained from: \verb|ekardt@fhi-berlin.mpg.de| \\

\newpage
\null

\begin{center}
{\Large \bf Preprints from Center for Atomic-scale Materials Physics (CAMP)}
\end{center}
\begin{itemize}
   
\item
I. Stensgaard, E. L{\ae}gsgaard, and F. Besenbacher: \\
{\em The reaction of carbon dioxide with an oxygen-precovered Ag(110) surface} 

\item
P.T. Sprunger, Y. Okawa, F. Besenbacher, I. Stensgaard, and K. Tanaka: \\
{\em STM investigation of the coadsorption and reaction of oxygen and hydrogen on Ni(110)} 

\item
A.R.H. Clarke, J.B. Pethica, J.A. Nieminen, F. Besenbacher, E. L{\ae}gsgaard, and I. Stensgaard:\\
{\em Quantitative STM at atomic resolution: influence of forces and tip configuration} 

\item
P.W. Murray, I Stensgaard, E. L{\ae}gsgaard, and F. Besenbacher: \\
{\em Mechanisms of initial alloy formation for Pd on Cu(100) studied by STM} 

\item
P.M. Holmblad, J.H. Larsen, I. Chorkendorff, L.P. Nielsen, F. Besenbacher, I. Stensgaard, E. L{\ae}gsgaard, P. Kratzer, B. Hammer, and J.K. N{\o}rskov: \\
{\em Designing surface alloys with specific chemical properties} 

\item
B. Hammer, Y. Morikawa, and J.K. N{\o}rskov: \\
{\em CO chemisorption over metal surfaces and overlayers} 

\item
H. Brune, K. Bromann, K. Kern, J. Jacobsen, P. Stoltze, K.W. Jacobsen, and J.K. N{\o}rskov:\\
{\em Atomistic porcesses in diffusion limited metal aggregation} 

\item
J. Jacobsen, K.W. Jacobsen, and J.K. N{\o}rskov:\\
{\em Island shapes in homoepitaxial growth of Pt(111)} 
 
\item
P. Kratzer, B. Hammer, and J.K. N{\o}rskov:\\
{\em Geometric and electronic factors determining the difference in reactivity of H$_2$ on Cu(110) and Cu(111)} 

\end{itemize}


\bigskip
\noindent
{\large More information can be obtained from helle@fysik.dtu.dk}

\bigskip
\noindent
Helle Wellejus\\
CAMP and Physics Department, Building 307, Technical University of Denmark\\
DK - 2800 Lyngby, Denmark

%\newpage
%\null
%\vskip 2em
%\begin{center}
%\Large{\bf  \par}
%\vskip 1.5em
%{\large \lineskip .5em \begin{tabular}[t]{c}
%{\it }\\
%{\it }\\
%\end{tabular}\par}
%\end{center}
%\begin{abstract}
%\end{abstract}

%\noindent
%(submitted to)\\
%Manuscripts available from: \\


%%%%%%%%%%%%%%%%%%%%%%%%%%%%%%%%%%%%%%%%%%%%%%%%%%%%%%%%%%%%%%%%%%%%%%%%%%%%%
%                                                                           %
%   Presenting other HCM Projects .........                                 %
%                                                                           %
%%%%%%%%%%%%%%%%%%%%%%%%%%%%%%%%%%%%%%%%%%%%%%%%%%%%%%%%%%%%%%%%%%%%%%%%%%%%%
\newpage
\null

\begin{center}
\LARGE \bf
Presenting Other Human Capital and Mobility Projects\\

\bigskip
\large
Report on research performed at Cineca, Bologne, Italy\\
supported by the EU Human Capital and Mobility Programme\\
{\it ICARUS Scheme}
\end{center}

\bigskip

{\Large \bf Modelling of oxygen defects in silicon}\\
{\large \it by Chris Ewels\\
\normalsize \it 
Department of Physics, University of Exeter, Devon. EX1 2HR. UK. \\
E-mail: ewels@excc.ex.ac.uk}
\bigskip

The European Union `Human Capital and Mobility Programme' encompasses
a range of schemes that share the common restriction that applicants
should spend time working in a country other than their own.  One such
scheme is {\it ICARUS} [1], organised by CINECA, the Italian
Universities' Central High Performance Computing (HPC) Facility [2].

{\it ICARUS} ($I$ntensive $C$omputing for $A$dvanced $I$nterdisciplinary
$R$esearch of E$U$ropean $S$cientists) started in January 1994 and
will finish in December 1996.  It is designed to provide visitors with
the training and resources necessary for research that requires
supercomputing facilities.  There are also many indirect benefits from
such a scheme; the constant flux of researchers provides a fascinating
insight into the way that other groups work, and allows comparison of
many different software packages and applications.  Finally the chance
to stay in one of Italy's most central and cosmopolitan cities should
not be missed.

CINECA is located in Casalecchio sul Reno [3], an industrial suburb of
Bologne [4] in central Italy.  It was originally set up in the early
seventies by a small group of Italian universities, in recognition of
the fact that high performance computing was too expensive to tackle
individually.  There are now 13 Universities in the consortium and
CINECA acts as a central service for computing support, training and
resources for the Italian Universities.

The resources at CINECA include a Cray C92/2128, Cray T3D (64 nodes;
to be replaced with a larger T3E in 1996), IBM S/390-9672, VAX
6000-510, as well as many SGI and RS6000 machines.  There is also a
scientific visualisation laboratory and excellent network facilities
to allow continued access from home sites after returning.  All of the
facilities are available to {\it ICARUS} visitors, along with
training, support and documentation where required.

Details of the {\it ICARUS} scheme can be found on the CINECA World
Wide Web site, including the necessary application forms and details
of how to apply [1].  There are two other similar schemes at other
supercomputing centres; the TRACS scheme at the Edinburgh parallel
computing centre [5] and a scheme at CESCA/CEPBA in Barcelona [6]. All
of these are designed to cover a broad spread of academic disciplines;
in addition there are a number of single discipline centres including
the European Molecular Biology Lab (EMBL) in Heidelberg, GEOMAR
(Forschungszentrum f\"ur Marine Geowissenschaften), and the European
Climate Computer Network including the Hadley Centre in Bracknell
and the DKRZ in Hamburg [7].

I can thoroughly recommend the {\it ICARUS} scheme, as it was
successful both in providing me with the resources I needed for my
calculations, and broadening my knowledge of effective ways of
utilising HPC resources. My previous HPC experience was primarily on
the Cray at Edinburgh, and it was interesting to see the differences
in approach between these two centres.  The Cray T3D at CINECA has
less nodes than Edinburgh and has a smaller workload, and due to its
size and immanent replacement is used primarily for developmental
work.  My work on the Edinburgh Cray has been performed remotely, so
there has been little opportunity for interaction with other users.
At CINECA, however, discussion of various parallelisation strategies
and different problem solving techniques proved extremely useful. I
spent six weeks at Cineca during July and August, in order to use
their Cray T3D to tackle the complex problem of oxygen diffusion in
silicon.

Diffusion is an extremely important process in solids, affecting both
their atomic structure and stability.  However it is difficult to
handle theoretically for several reasons.  Firstly there are many
possible channels for diffusion, including simple atomic exchange,
interstitial or vacancy controlled diffusion, and diffusion catalysed
by other impurities.  Secondly, diffusion involves non-equilibrium
structures, and requires the accurate determination of {\it saddle
point structures} (structures possessing the highest energy along a
particular diffusion path).  This means that a lot of large
calculations must be performed to examine even a relatively simple
diffusion path, and workstations are prohibitively slow for such work.

The calculations were performed using the {\it Aimpro} code, a local
density functional scheme applied to atomic clusters [8].  {\it
Aimpro} has been developed at Exeter and Newcastle for over ten years
[9], with successful application to a wide variety of molecular and
semiconductor problems.  Recent development of the code includes
parallelisation and incorporation of the block algorithm SCALAPACK
routines [10]. These routines take full advantage of massively parallel
architectures.  The code can run with comparable efficiency across a
variety of platforms including parallel machines (such as the Cray T3D
and IBM SP2), workstation clusters, and single workstations (eg SGI,
IBM, and HP machines).

Oxygen in silicon has been an active area of research for over forty
years.  The main problem is to identify the structure and properties
of oxygen aggregates, and determine the mechanism of interstitial
creation.  Oxygen is the most common impurity in commercial
Czochralski grown silicon, and forms many electrically active defects.
The most important of these are {\it Thermal Donors}, that form during
low temperature annealing from 300-500~$^\circ$C [11].  In spite of
intensive work, there has been no consensus about their structure.

Crucial to an understanding of how these defects form is a thorough
investigation of how the defect atoms diffuse through the silicon
lattice.  Many of the oxygen related defects start to form at
temperatures below 450~$^\circ$C, however at these temperatures
isolated interstitial oxygen does not diffuse.  It has recently been
suggested that oxygen may be able to diffuse much faster when
travelling in pairs [12], which could diffuse at lower
temperatures. If this was the case it would account for much of the
unusual formation kinetics of oxygen defects.

Fast diffusing oxygen pairs would also have implications for low
temperature formation of oxygen-vacancy complexes (such as VO$_2$) and
would certainly affect our understanding of the oxygen diffusion
mechanisms responsible for defect formation.  Therefore I modelled the
diffusion of a single oxygen interstitial and an interstitial oxygen
pair.

Preliminary calculations using small clusters on the Cray gave a
calculated energy barrier for single interstitial oxygen of 3.42 eV,
and 2.14 eV for an oxygen interstitial pair.  This supports the idea
that an oxygen pair can diffuse at lower temperatures than a single
oxygen interstitial.  However the calculated single interstitial
barrier is much higher than the experimental value of 2.54 eV, since
the diffusion path requires a long range relaxation of lattice silicon
atoms, which is not possible in the small cluster being used.  Using
the Cray T3D I was able to test this by examining different sizes of
cluster (see Table 1).  As the cluster size increases, the barrier
drops gradually towards the experimental value.  Work performed on 132
atom clusters since this project was completed is giving a calculated
diffusion barrier for single interstitial oxygen of around 2.5 eV.
Similar cluster size calculations are underway for the oxygen pair.

\bigskip 

\begin{center}
{\bf Table 1 :} Effect of Cluster size on diffusion barrier \\ 
for interstitial oxygen diffusion in silicon\\
\begin{tabular}{ccc} \hline
Cluster & Number of Atoms & Energy Barrier (eV) \\ \hline
Si$_{11}$H$_{24}$O & 36 & 3.424 \\
Si$_{14}$H$_{30}$O & 45 & 3.158 \\
Si$_{35}$H$_{36}$O & 72 & 2.969 \\ \hline
\end{tabular}
\end{center}

\bigskip
{\large \it References:}
\smallskip

1.  http://www.cineca.it/icarus/              \\
2.  http://www.cineca.it/                     \\
3.  http://www.nettuno.it/casalecchio/        \\
4.  http://www.nettuno.it/bologna/            \\
5.  http://www.epcc.ed.ac.uk/epcc/link/tracs/ \\
6.  http://www.cesca.es/		      \\
7.  http://www.dkrz.de/dkrz/eccn-eng.html     \\
8.  http://newton.ex.ac.uk/jones/             \\
9.  P. R Briddon and R. Jones, Phys Rev Lett, {\bf 64}, 2535, (1990). \\
10. http://www.netlib.org/scalapack/scalapack\_home.html \\
11. W. Kaiser, H. L. Frisch, H. Reiss, Phys Rev {\bf 112}, 1546, (1958). \\
12. C. A. Londos, M. J. Binns, A. R. Brown, S. A. McQuaid,
     R. C. Newman, Appl. Phys. Lett., {\bf 62} (13), 1525 (1993). \\


%%%%%%%%%%%%%%%%%%%%%%%%%%%%%%%%%%%%%%%%%%%%%%%%%%%%%%%%%%%%%%%%%%%%%%%%%%%%%
%%%%%%%%%%%%%%%%%%%%%%%%%%%%%%%%%%%%%%%%%%%%%%%%%%%%%%%%%%%%%%%%%%%%%%%
%                                                                     %
%     Here the text of the Highlight of the month is included         %
%                                                                     %
%%%%%%%%%%%%%%%%%%%%%%%%%%%%%%%%%%%%%%%%%%%%%%%%%%%%%%%%%%%%%%%%%%%%%%%
\newpage
\null
\begin{center}
\Large {\bf HIGHLIGHT OF THE MONTH}
\end{center}
\vspace{1cm}
\rule{16.5cm}{1mm}
\vspace{1cm}
%\begin{center}
%\Large {\bf title}\\
%
%\Large{ (subtitle)}\\
%
%\large{\it author}\\
%\large{\it address}\\
%\end{center}
%\vspace{1cm}
%\rule{16.5cm}{1mm}
%\vskip 0.5cm


%\documentstyle[12pt]{article}
%%%%%%%%%%%%%%%%%%%%%%%%%%%%%%%%%%%%%%%%%%%%%%%%%%%%%%%%%%%%%%%%%%%%%%%%%%%%%%%
% Allgemeine Definitionen %%%%%%%%%%%%%%%%%%%%%%%%%%%%%%%%%%%%%%%%%%%%%%%%%%%%%
%%%%%%%%%%%%%%%%%%%%%%%%%%%%%%%%%%%%%%%%%%%%%%%%%%%%%%%%%%%%%%%%%%%%%%%%%%%%%%%
%%%%% Seiten-Layout
%------------------
\pagestyle{plain}
\renewcommand{\baselinestretch}{1.1}
\hoffset-1in
\oddsidemargin25,7mm
\voffset-1in
\topmargin25mm
\textwidth156mm
\headheight0pt \headsep0pt \topskip1\baselineskip
\textheight240mm
\footskip15mm
%\parindent0pt
%
\renewcommand{\topfraction}{1.0}
\renewcommand{\textfraction}{0.0}
\renewcommand{\bottomfraction}{1.0}
\setcounter{topnumber}{3}
\setcounter{bottomnumber}{3}
\setcounter{totalnumber}{6}
%
%%%%%%%%%%%%%%%%%%%%%%%%%%%%%%%%%%%%%%%%%%%%%%%%%%%%%%%%%%%%%%%%%%%%%%%%%%%%%%%
\edef\PSIKRestoreAt{\catcode`@=\number\catcode`@\relax}%
\catcode`\@=11\relax
%%%%%%%%%%%%%%%%%%%%%%%%%%%%%%%%%%%%%%%%%%%%%%%%%%%%%%%%%%%%%%%%%%%%%%%%%%%%%%%
% bezier.sty %%%%%%%%%%%%%%%%%%%%%%%%%%%%%%%%%%%%%%%%%%%%%%%%%%%%%%%%%%%%%%%%%%
% BEZIER DOCUMENT-STYLE OPTION - released 17 December 1985
%    for LaTeX version 2.09
% Copyright (C) 1985 by Leslie Lamport
\newcounter{@sc}
\newcounter{@scp}
\newcounter{@t}
\newlength{\@x}
\newlength{\@xa}
\newlength{\@xb}
\newlength{\@y}
\newlength{\@ya}
\newlength{\@yb}
\newsavebox{\@pt}

\def\bezier#1(#2,#3)(#4,#5)(#6,#7){\c@@sc#1\relax
  \c@@scp\c@@sc \advance\c@@scp\@ne
  \@xb #4\unitlength \advance\@xb -#2\unitlength \multiply\@xb \tw@
  \@xa #6\unitlength \advance\@xa -#2\unitlength
      \advance\@xa -\@xb \divide\@xa\c@@sc
  \@yb #5\unitlength \advance\@yb -#3\unitlength \multiply\@yb \tw@
  \@ya #7\unitlength \advance\@ya -#3\unitlength
      \advance\@ya -\@yb \divide\@ya\c@@sc
  \setbox\@pt\hbox{\vrule height\@halfwidth  depth\@halfwidth
   width\@wholewidth}\c@@t\z@
   \put(#2,#3){\@whilenum{\c@@t<\c@@scp}\do
      {\@x\c@@t\@xa \advance\@x\@xb \divide\@x\c@@sc \multiply\@x\c@@t
       \@y\c@@t\@ya \advance\@y\@yb \divide\@y\c@@sc \multiply\@y\c@@t
       \raise \@y \hbox to \z@{\hskip \@x\unhcopy\@pt\hss}%
       \advance\c@@t\@ne}}}
%%%%%%%%%%%%%%%%%%%%%%%%%%%%%%%%%%%%%%%%%%%%%%%%%%%%%%%%%%%%%%%%%%%%%%%%%%%%%%%
\PSIKRestoreAt
%%%%%%%%%%%%%%%%%%%%%%%%%%%%%%%%%%%%%%%%%%%%%%%%%%%%%%%%%%%%%%%%%%%%%%%%%%%%%%%
%
%%%%%%%%%%%%%%%%%%%%%%%%%%%%%%%%%%%%%%%%%%%%%%%%%%%%%%%%%%%%%%%%%%%%%%%%%%%%%%%
%
%\begin{document}
%
%%%%%%%%%%%%%%%%%%%%%%%%%%%%%%%%%%%%%%%%%%%%%%%%%%%%%%%%%%%%%%%%%%%%%%%%%%%%%%%
% GNUPLOT: LaTeX picture with Postscript %%%%%%%%%%%%%%%%%%%%%%%%%%%%%%%%%%%%%%
%%%%%%%%%%%%%%%%%%%%%%%%%%%%%%%%%%%%%%%%%%%%%%%%%%%%%%%%%%%%%%%%%%%%%%%%%%%%%%%
\special{!
%!PS-Adobe-2.0
%%Creator: gnuplot
%%DocumentFonts: Helvetica
%%BoundingBox: 50 50 770 520
%%Pages: (atend)
%%EndComments
/gnudict 40 dict def
gnudict begin
/Color false def
/Solid false def
/gnulinewidth 5.000 def
/vshift -33 def
/dl {10 mul} def
/hpt 31.5 def
/vpt 31.5 def
/M {moveto} bind def
/L {lineto} bind def
/R {rmoveto} bind def
/V {rlineto} bind def
/vpt2 vpt 2 mul def
/hpt2 hpt 2 mul def
/Lshow { currentpoint stroke M
  0 vshift R show } def
/Rshow { currentpoint stroke M
  dup stringwidth pop neg vshift R show } def
/Cshow { currentpoint stroke M
  dup stringwidth pop -2 div vshift R show } def
/DL { Color {setrgbcolor Solid {pop []} if 0 setdash }
 {pop pop pop Solid {pop []} if 0 setdash} ifelse } def
/BL { stroke gnulinewidth 2 mul setlinewidth } def
/AL { stroke gnulinewidth 2 div setlinewidth } def
/PL { stroke gnulinewidth setlinewidth } def
/LTb { BL [] 0 0 0 DL } def
/LTa { AL [1 dl 2 dl] 0 setdash 0 0 0 setrgbcolor } def
/LT0 { PL [] 0 1 0 DL } def
/LT1 { PL [4 dl 2 dl] 0 0 1 DL } def
/LT2 { PL [2 dl 3 dl] 1 0 0 DL } def
/LT3 { PL [1 dl 1.5 dl] 1 0 1 DL } def
/LT4 { PL [5 dl 2 dl 1 dl 2 dl] 0 1 1 DL } def
/LT5 { PL [4 dl 3 dl 1 dl 3 dl] 1 1 0 DL } def
/LT6 { PL [2 dl 2 dl 2 dl 4 dl] 0 0 0 DL } def
/LT7 { PL [2 dl 2 dl 2 dl 2 dl 2 dl 4 dl] 1 0.3 0 DL } def
/LT8 { PL [2 dl 2 dl 2 dl 2 dl 2 dl 2 dl 2 dl 4 dl] 0.5 0.5 0.5 DL } def
/P { stroke [] 0 setdash
  currentlinewidth 2 div sub M
  0 currentlinewidth V stroke } def
/D { stroke [] 0 setdash 2 copy vpt add M
  hpt neg vpt neg V hpt vpt neg V
  hpt vpt V hpt neg vpt V closepath stroke
  P } def
/DD { stroke [] 0 setdash 2 copy vpt add M
  hpt neg vpt neg V hpt vpt neg V
  hpt vpt V hpt neg vpt V closepath fill
  P } def
/A { stroke [] 0 setdash vpt sub M 0 vpt2 V
  currentpoint stroke M
  hpt neg vpt neg R hpt2 0 V stroke
  } def
/B { stroke [] 0 setdash 2 copy exch hpt sub exch vpt add M
  0 vpt2 neg V hpt2 0 V 0 vpt2 V
  hpt2 neg 0 V closepath stroke
  P } def
/BB { stroke [] 0 setdash 2 copy exch hpt sub exch vpt add M
  0 vpt2 neg V hpt2 0 V 0 vpt2 V
  hpt2 neg 0 V closepath fill
  P } def
/C { stroke [] 0 setdash exch hpt sub exch vpt add M
  hpt2 vpt2 neg V currentpoint stroke M
  hpt2 neg 0 R hpt2 vpt2 V stroke } def
/T { stroke [] 0 setdash 2 copy vpt 1.12 mul add M
  hpt neg vpt -1.62 mul V
  hpt 2 mul 0 V
  hpt neg vpt 1.62 mul V closepath stroke
  P  } def
/TT { stroke [] 0 setdash 2 copy vpt 1.12 mul add M
  hpt neg vpt -1.62 mul V
  hpt 2 mul 0 V
  hpt neg vpt 1.62 mul V closepath fill
  P  } def
/S { 2 copy A C} def
end
%%EndProlog
}
%%%%%%%%%%%%%%%%%%%%%%%%%%%%%%%%%%%%%%%%%%%%%%%%%%%%%%%%%%%%%%%%%%%%%%%%%%%%%%%
%%%%%%%%%%%%%%%%%%%%%%%%%%%%%%%%%%%%%%%%%%%%%%%%%%%%%%%%%%%%%%%%%%%%%%%%%%%%%%%
\section*{Ab-initio calculation of Giant Magnetoresistance in magnetic 
multilayers}
\vspace{2\baselineskip}
%
\hspace*{1.69cm}\begin{minipage}[t]{5.125in}
Peter Zahn,$^1$ Ingrid Mertig,$^1$ Manuel Richter,$^2$ and Helmut Eschrig$^2$
\end{minipage}\\[\baselineskip]
%
\hspace*{1in}\begin{minipage}[t]{5in}
Technische Universit\"at Dresden\\
$^1$Institut f\"ur Theoretische Physik\\
$^2$MPG-Arbeitsgruppe "Elektronensysteme"\\
D-01062 Dresden, Germany
\end{minipage}
\vspace{2\baselineskip}
%
%%%
%
\section*{Introduction}
Since the discovery of giant magnetoresistance (GMR) in magnetic
multilayer systems (like $Fe/Cr$ or $Co/Cu$) \cite{baibich88,binash89}
several experimental
and theoretical studies have been carried out to elucidate the microscopic
origin of the phenomenon.
Here, {\it ab initio} calculations of the electronic structure and of the
electron scattering are of great interest.
\par
%
%%%%%%%%%%%%%%%%%%%%%%%%%%%%%%%%%%%%%%%%%%%%%%%%%%%%%%%%%%%%%%%%%%%%%%%%%%%%%%%
%%%   Abb. 1                                                                %%%
%%%%%%%%%%%%%%%%%%%%%%%%%%%%%%%%%%%%%%%%%%%%%%%%%%%%%%%%%%%%%%%%%%%%%%%%%%%%%%%
%
\begin{figure}[hbt]\begin{center}
{\Large\unitlength=1mm
\linethickness{0.4pt}}
\begin{picture}(133.00,70.00)(0,0)
\put(0.00,10.00){\vector(1,0){125.00}}
\put(60.00,10.00){\vector(0,1){55.00}}
\put(63.00,62.00){\makebox(0,0)[lc]{$\frac{\rho(H)-\rho(H_S)}{\rho(H_S)}$}}
\bezier{144}(60.00,50.00)(72.00,50.00)(85.00,30.00)
\bezier{144}(60.00,50.00)(48.00,50.00)(35.00,30.00)
\bezier{164}(35.00,30.00)(22.00,10.00)(5.00,10.00)
\bezier{164}(115.00,10.00)(98.00,10.00)(85.00,30.00)
\put(129.00,10.00){\makebox(0,0)[lc]{H}}
\put(115.00,8.00){\makebox(0,0)[ct]{$H_S$}}
\put(115.00,10.00){\circle*{1.00}}
\end{picture}
\caption{Giant magnetoresistance. The resistivity is maximum when the
magnetic moments of successive ferromagnetic layers are antiparallel.
It drops off as the applied field aligns the magnetic moments $H_S$,
denoted by the right and left diagrams for positive and negative fields,
respectively.\label{fig12}}
\end{center}
\end{figure}
%
%%%%%%%%%%%%%%%%%%%%%%%%%%%%%%%%%%%%%%%%%%%%%%%%%%%%%%%%%%%%%%%%%%%%%%%%%%%%%%%
%
The GMR effect occurs when successive ferromagnetic layers exhibit 
anti-parallel magnetization. The application of an external
magnetic field brings the magnetization of the ferromagnetic layers into
alignment and causes a decrease of resistivity (Fig. \ref{fig12})
for both current-in-plane (CIP) and current-perpendicular-to-plane (CPP) 
geometries (see Fig. \ref{fig13}).
\par
%
%%%%%%%%%%%%%%%%%%%%%%%%%%%%%%%%%%%%%%%%%%%%%%%%%%%%%%%%%%%%%%%%%%%%%%%%%%%%%%%
%%%   Abb. 2                                                                %%%
%%%%%%%%%%%%%%%%%%%%%%%%%%%%%%%%%%%%%%%%%%%%%%%%%%%%%%%%%%%%%%%%%%%%%%%%%%%%%%%
%
\begin{figure}[hbt]\begin{center}
{\large\unitlength=1.00mm\linethickness{0.4pt}
\begin{picture}(90.00,55.00)(0,10)
\put(0.00,15.00){\framebox(20.00,20.00)[cc]{}}
\put(7.50,15.00){\framebox(5.00,20.00)[cc]{}}
\put(20.00,15.00){\line(1,1){10.00}}
\put(30.00,25.00){\line(0,1){20.00}}
\put(30.00,45.00){\line(-1,-1){10.00}}
\put(30.00,45.00){\line(-1,0){7.50}}
\put(22.50,45.00){\line(-1,-1){10.00}}
\put(22.50,45.00){\line(-1,0){5.00}}
\put(17.50,45.00){\line(-1,-1){10.00}}
\put(17.50,45.00){\line(-1,0){7.50}}
\put(10.00,45.00){\line(-1,-1){10.00}}
\put(50.00,15.00){\framebox(20.00,20.00)[cc]{}}
\put(57.50,15.00){\framebox(5.00,20.00)[cc]{}}
\put(70.00,15.00){\line(1,1){10.00}}
\put(80.00,25.00){\line(0,1){20.00}}
\put(80.00,45.00){\line(-1,-1){10.00}}
\put(80.00,45.00){\line(-1,0){7.50}}
\put(72.50,45.00){\line(-1,-1){10.00}}
\put(72.50,45.00){\line(-1,0){5.00}}
\put(67.50,45.00){\line(-1,-1){10.00}}
\put(67.50,45.00){\line(-1,0){7.50}}
\put(60.00,45.00){\line(-1,-1){10.00}}
\put(4.00,20.00){\vector(0,1){10.00}}
\put(15.00,30.00){\vector(0,-1){10.00}}
\put(17.00,20.00){\vector(0,1){10.00}}
\put(54.00,20.00){\vector(0,1){10.00}}
\put(65.00,30.00){\vector(0,-1){10.00}}
\put(67.00,20.00){\vector(0,1){10.00}}
\put(12.00,37.00){\vector(0,1){10.00}}
\put(15.00,40.00){\vector(0,1){10.00}}
\put(18.00,43.00){\vector(0,1){10.00}}
\put(11.00,51.00){\makebox(0,0)[cc]{$\vec{j}$}}
\put(65.00,40.00){\vector(0,1){10.00}}
\put(15.00,55.00){\makebox(0,0)[cc]{$\vec{H}$}}
\put(19.00,59.00){\makebox(0,0)[cc]{$\vec{E}$}}
\put(65.00,55.00){\makebox(0,0)[cc]{$\vec{H}$}}
\put(75.00,27.00){\vector(1,0){10.00}}
\put(75.00,35.00){\vector(1,0){10.00}}
\put(90.00,35.00){\makebox(0,0)[cc]{$\vec{E}$}}
\put(90.00,27.00){\makebox(0,0)[cc]{$\vec{j}$}}
%\put(60.00,5.00){\makebox(0,0)[cc]{{\LARGE\bf CPP}}}
%\put(10.00,5.00){\makebox(0,0)[cc]{{\LARGE\bf CIP}}}
\end{picture}}
\caption{Geometrical arrangements of current density $\bf j$, electrical
field $\bf E$ and magnetic field $\bf H$ in CIP and CPP measurement of GMR.%
\label{fig13}}
\end{center}
\end{figure}
%
%%%%%%%%%%%%%%%%%%%%%%%%%%%%%%%%%%%%%%%%%%%%%%%%%%%%%%%%%%%%%%%%%%%%%%%%%%%%%%%
%
Most theories \cite{camley89,levy90,inoue91,hood92,valet93,levy94}
try to explain the GMR by spin-dependent scattering at interfaces or
bulk defects but neglect the electronic structure of the multilayer
system.
Attempts to include the electronic structure have been made
by several authors \cite{oguchi93,butler93,schep95,zahn95}.
All calculations predict a strong influence of the electronic structure
of the multilayer on the GMR.\par\noindent
The GMR in magnetic multilayers is defined as
\begin{equation}\label{eq63}
{\rm GMR}=\frac{\sigma^{P}}{\sigma^{AP}} - 1 ,
\end{equation}
where $\sigma^{P}$ and $\sigma^{AP}$ are the conductivities
of the multilayer for parallel and anti-parallel alignment
of successive ferromagnetic layer magnetization, respectively.
Experimentally, one can be sure that parallel alignment is achieved,
in view of the marked transition from strong to weak magnetic field
dependence of the conductivity at the saturation field (see Fig. \ref{fig12}).
%
%
\section*{Electronic structure of the multilayer}
Our considerations focus on
(100) oriented $Fe_mCr_n$ layer sequences ($m$ monolayers of $Fe$ followed
by $n$ monolayers of $Cr$), where the $Fe$ layers are intrinsically
ferromagnetic and the $Cr$ layers are intrinsically antiferromagnetic.
{\it Ab initio} electronic structure calculations have been performed
using spin-density functional theory for initially parallel and antiparallel
spin configurations of subsequent $Fe$ layers \cite{zahn95}.\par
Starting from the bandstructure $E_k^\sigma$ of the situation,
that we have just addressed, potential
scattering causes transitions from a state $k$ into a state $k'$
of the multilayer system.
The scattering can be caused by impurity atoms in the ferromagnetic or in the
nonmagnetic spacer layer. The scattering is described by spin-dependent
relaxation times $\tau^\sigma$ which are different for majority and
minority electrons depending on the type of scatterer.
The spin anisotropy ratio
$\beta=\tau^\uparrow/\tau^\downarrow$ is a direct measure. $\beta<1$
means strong scattering of the majority electrons and for $\beta>1$ the
minority electrons are strongly scattered.
%
%
\section*{Transport theory}
Transport is described within the quasiclassical theory solving the
Boltzmann equation in relaxation time approximation.
Consequently, the conductivity tensor becomes
\begin{equation}\label{eq64}
{\hat \sigma}^{\sigma}= e^2
\tau^\sigma \sum _{k} \delta (E _{k}^\sigma -
E_F)\: {\bf v}^{\sigma}_k \: {\bf v}^{\sigma}_k .
\end{equation}
The integration is performed over
the Fermi surface of the parallel configuration of the layered system.
Summation of the two currents yields the total
conductivity
\begin{equation}\label{eq65}
{\hat \sigma}^{P} = {\hat \sigma} ^{\uparrow} + {\hat \sigma} ^{\downarrow}.
\end{equation}
In the anti-parallel configuration the electronic states are spin
degenerate, and the conductivity becomes
\begin{equation}\label{eq66}
{\hat \sigma}^{AP} = 2 \: e^2
\tau^{AP} \sum _{k} \delta (E _{k}^{AP} -
E_F)\: {\bf v}^{AP}_k \: {\bf v}^{AP}_k.
\end{equation}
Assuming that the layered system is grown in the $z$ direction,
CIP corresponds
to the $xx$ or $yy$ component of the conductivity tensor and CPP to the
$zz$ component in Eqs. \ref{eq64} and \ref{eq66}.
$\tau^{AP}$, the relaxation time in the anti-parallel configuration,
can be obtained by summation of the scattering operators.
For equal concentration of defects in adjacent ferromagnetic layers,
the relaxation time is then given by
\begin{equation}\label{eq67}
\frac{1}{\tau^{AP}}=\frac{1}{2}\;(\;\frac{1}{\tau^\uparrow}\;+\;\frac{1}
{\tau^\downarrow}\;).
\end{equation}
%
%
\section*{Results}
The simplest situation occurs, when all the scattering centers are localized
in the antiferromagnetic $Cr$ layers (for example $Fe$ impurities).
\par
%
%%%%%%%%%%%%%%%%%%%%%%%%%%%%%%%%%%%%%%%%%%%%%%%%%%%%%%%%%%%%%%%%%%%%%%%%%%%%%%%
%%%   Abb. 3                                                                %%%
%%%%%%%%%%%%%%%%%%%%%%%%%%%%%%%%%%%%%%%%%%%%%%%%%%%%%%%%%%%%%%%%%%%%%%%%%%%%%%%
%
\begin{figure}\begin{center}
\setlength{\unitlength}{0.1bp}
\begin{picture}(3600,2014)(0,-100)
\special{"
%%Page: 1 1
gnudict begin
gsave
50 50 translate
0.100 0.100 scale
0 setgray
/Helvetica findfont 100 scalefont setfont
newpath
-500.000000 -500.000000 translate
LTa
600 251 M
2817 0 V
600 251 M
0 1712 V
LTb
LTa
600 251 M
2817 0 V
LTb
600 251 M
63 0 V
2754 0 R
-63 0 V
LTa
600 422 M
2817 0 V
LTb
600 422 M
63 0 V
2754 0 R
-63 0 V
LTa
600 593 M
2817 0 V
LTb
600 593 M
63 0 V
2754 0 R
-63 0 V
LTa
600 765 M
2817 0 V
LTb
600 765 M
63 0 V
2754 0 R
-63 0 V
LTa
600 936 M
2817 0 V
LTb
600 936 M
63 0 V
2754 0 R
-63 0 V
LTa
600 1107 M
2817 0 V
LTb
600 1107 M
63 0 V
2754 0 R
-63 0 V
LTa
600 1278 M
2817 0 V
LTb
600 1278 M
63 0 V
2754 0 R
-63 0 V
LTa
600 1449 M
2817 0 V
LTb
600 1449 M
63 0 V
2754 0 R
-63 0 V
LTa
600 1621 M
2817 0 V
LTb
600 1621 M
63 0 V
2754 0 R
-63 0 V
LTa
600 1792 M
2817 0 V
LTb
600 1792 M
63 0 V
2754 0 R
-63 0 V
LTa
600 1963 M
2817 0 V
LTb
600 1963 M
63 0 V
2754 0 R
-63 0 V
LTa
600 251 M
0 1712 V
LTb
600 251 M
0 63 V
0 1649 R
0 -63 V
LTa
801 251 M
0 1712 V
LTb
801 251 M
0 63 V
0 1649 R
0 -63 V
LTa
1002 251 M
0 1712 V
LTb
1002 251 M
0 63 V
0 1649 R
0 -63 V
LTa
1204 251 M
0 1712 V
LTb
1204 251 M
0 63 V
0 1649 R
0 -63 V
LTa
1405 251 M
0 1712 V
LTb
1405 251 M
0 63 V
0 1649 R
0 -63 V
LTa
1606 251 M
0 1712 V
LTb
1606 251 M
0 63 V
0 1649 R
0 -63 V
LTa
1807 251 M
0 1712 V
LTb
1807 251 M
0 63 V
0 1649 R
0 -63 V
LTa
2009 251 M
0 1712 V
LTb
2009 251 M
0 63 V
0 1649 R
0 -63 V
LTa
2210 251 M
0 1712 V
LTb
2210 251 M
0 63 V
0 1649 R
0 -63 V
LTa
2411 251 M
0 1712 V
LTb
2411 251 M
0 63 V
0 1649 R
0 -63 V
LTa
2612 251 M
0 1712 V
LTb
2612 251 M
0 63 V
0 1649 R
0 -63 V
LTa
2813 251 M
0 1712 V
LTb
2813 251 M
0 63 V
0 1649 R
0 -63 V
LTa
3015 251 M
0 1712 V
LTb
3015 251 M
0 63 V
0 1649 R
0 -63 V
LTa
3216 251 M
0 1712 V
LTb
3216 251 M
0 63 V
0 1649 R
0 -63 V
LTa
3417 251 M
0 1712 V
LTb
3417 251 M
0 63 V
0 1649 R
0 -63 V
600 251 M
2817 0 V
0 1712 V
-2817 0 V
600 251 L
LT0
801 1148 M
201 368 V
202 -430 V
201 262 V
201 514 V
201 -368 V
202 341 V
201 -541 V
201 181 V
201 -298 V
201 182 V
202 -298 V
201 233 V
801 1148 BB
1002 1516 BB
1204 1086 BB
1405 1348 BB
1606 1862 BB
1807 1494 BB
2009 1835 BB
2210 1294 BB
2411 1475 BB
2612 1177 BB
2813 1359 BB
3015 1061 BB
3216 1294 BB
LT0
801 453 M
201 82 V
1204 410 L
201 14 V
201 89 V
201 -81 V
202 40 V
2210 369 L
201 22 V
201 -63 V
201 53 V
202 -70 V
201 55 V
801 453 TT
1002 535 TT
1204 410 TT
1405 424 TT
1606 513 TT
1807 432 TT
2009 472 TT
2210 369 TT
2411 391 TT
2612 328 TT
2813 381 TT
3015 311 TT
3216 366 TT
LT0
801 453 M
201 82 V
1204 410 L
201 14 V
201 89 V
201 -81 V
202 40 V
2210 369 L
201 22 V
201 -63 V
201 53 V
202 -70 V
201 55 V
801 624 M
201 86 V
1204 583 L
201 24 V
201 106 V
201 -89 V
202 60 V
2210 561 L
201 32 V
201 -70 V
201 53 V
202 -75 V
201 58 V
801 624 S
1002 710 S
1204 583 S
1405 607 S
1606 713 S
1807 624 S
2009 684 S
2210 561 S
2411 593 S
2612 523 S
2813 576 S
3015 501 S
3216 559 S
LT0
801 624 M
201 86 V
1204 583 L
201 24 V
201 106 V
201 -89 V
202 60 V
2210 561 L
201 32 V
201 -70 V
201 53 V
202 -75 V
201 58 V
LT1
801 1116 M
20 23 V
20 23 V
21 21 V
20 21 V
20 20 V
20 20 V
20 18 V
20 19 V
20 17 V
20 17 V
21 16 V
20 16 V
20 15 V
20 14 V
20 14 V
20 13 V
20 13 V
20 12 V
21 11 V
20 11 V
20 11 V
20 9 V
20 10 V
20 8 V
20 9 V
20 7 V
20 8 V
21 6 V
20 7 V
20 6 V
20 5 V
20 5 V
20 4 V
20 4 V
20 4 V
21 3 V
20 3 V
20 2 V
20 2 V
20 2 V
20 1 V
20 0 V
20 1 V
21 0 V
20 0 V
20 -1 V
20 -1 V
20 -1 V
20 -2 V
20 -1 V
20 -3 V
21 -2 V
20 -3 V
20 -3 V
20 -3 V
20 -3 V
20 -4 V
20 -4 V
20 -4 V
21 -5 V
20 -4 V
20 -5 V
20 -5 V
20 -5 V
20 -6 V
20 -5 V
20 -6 V
20 -6 V
21 -6 V
20 -6 V
20 -6 V
20 -7 V
20 -6 V
20 -7 V
20 -6 V
20 -7 V
21 -7 V
20 -7 V
20 -7 V
20 -7 V
20 -7 V
20 -7 V
20 -7 V
20 -7 V
21 -7 V
20 -7 V
20 -7 V
20 -7 V
20 -7 V
20 -7 V
20 -7 V
20 -7 V
21 -7 V
20 -7 V
20 -7 V
20 -6 V
20 -7 V
20 -6 V
20 -7 V
20 -6 V
20 -6 V
21 -6 V
20 -5 V
20 -6 V
20 -6 V
20 -5 V
20 -5 V
20 -5 V
20 -5 V
21 -4 V
20 -4 V
20 -4 V
20 -4 V
20 -4 V
20 -3 V
20 -3 V
20 -3 V
21 -3 V
20 -2 V
20 -2 V
801 461 M
20 2 V
20 1 V
21 2 V
20 1 V
20 2 V
20 1 V
20 1 V
20 1 V
20 1 V
20 1 V
21 1 V
20 0 V
20 1 V
20 0 V
20 0 V
20 1 V
20 0 V
20 0 V
21 0 V
20 0 V
20 -1 V
20 0 V
20 0 V
20 -1 V
20 0 V
20 -1 V
20 -1 V
21 -1 V
20 0 V
20 -1 V
20 -1 V
20 -1 V
20 -2 V
20 -1 V
20 -1 V
21 -1 V
20 -2 V
20 -1 V
20 -1 V
20 -2 V
20 -1 V
20 -2 V
20 -2 V
21 -1 V
20 -2 V
20 -2 V
20 -2 V
20 -1 V
20 -2 V
20 -2 V
20 -2 V
21 -2 V
20 -2 V
20 -2 V
20 -2 V
20 -2 V
20 -2 V
20 -2 V
20 -2 V
21 -2 V
20 -2 V
20 -2 V
20 -3 V
20 -2 V
20 -2 V
20 -2 V
20 -2 V
20 -2 V
21 -2 V
20 -2 V
20 -2 V
20 -2 V
20 -2 V
20 -2 V
20 -2 V
20 -2 V
21 -2 V
20 -2 V
20 -2 V
20 -2 V
20 -2 V
20 -2 V
20 -2 V
20 -2 V
21 -1 V
20 -2 V
20 -2 V
20 -1 V
20 -2 V
20 -1 V
20 -2 V
20 -1 V
21 -2 V
20 -1 V
20 -1 V
20 -2 V
20 -1 V
20 -1 V
20 -1 V
20 -1 V
20 -1 V
21 -1 V
20 0 V
20 -1 V
20 -1 V
20 0 V
20 -1 V
20 0 V
20 0 V
21 -1 V
20 0 V
20 0 V
20 0 V
20 0 V
20 1 V
20 0 V
20 0 V
21 1 V
20 1 V
20 0 V
801 628 M
20 3 V
20 2 V
21 2 V
20 3 V
20 2 V
20 2 V
20 2 V
20 2 V
20 1 V
20 2 V
21 1 V
20 2 V
20 1 V
20 1 V
20 1 V
20 1 V
20 1 V
20 0 V
21 1 V
20 0 V
20 1 V
20 0 V
20 0 V
20 0 V
20 0 V
20 0 V
20 0 V
21 0 V
20 0 V
20 -1 V
20 0 V
20 -1 V
20 -1 V
20 0 V
20 -1 V
21 -1 V
20 -1 V
20 -1 V
20 -1 V
20 -1 V
20 -1 V
20 -1 V
20 -1 V
21 -2 V
20 -1 V
20 -1 V
20 -2 V
20 -1 V
20 -2 V
20 -2 V
20 -1 V
21 -2 V
20 -2 V
20 -1 V
20 -2 V
20 -2 V
20 -2 V
20 -1 V
20 -2 V
21 -2 V
20 -2 V
20 -2 V
20 -2 V
20 -2 V
20 -2 V
20 -2 V
20 -2 V
20 -2 V
21 -2 V
20 -2 V
20 -2 V
20 -2 V
20 -2 V
20 -2 V
20 -2 V
20 -2 V
21 -2 V
20 -2 V
20 -2 V
20 -2 V
20 -2 V
20 -2 V
20 -1 V
20 -2 V
21 -2 V
20 -2 V
20 -2 V
20 -1 V
20 -2 V
20 -2 V
20 -1 V
20 -2 V
21 -2 V
20 -1 V
20 -2 V
20 -1 V
20 -1 V
20 -2 V
20 -1 V
20 -1 V
20 -1 V
21 -1 V
20 -1 V
20 -1 V
20 -1 V
20 -1 V
20 -1 V
20 -1 V
20 0 V
21 -1 V
20 0 V
20 -1 V
20 0 V
20 0 V
20 -1 V
20 0 V
20 0 V
21 0 V
20 0 V
20 1 V
stroke
grestore
end
showpage
}
\put(2008,-49){\makebox(0,0){{\large Cr Thickness (ML)}}}
\put(200,1107){%
\special{ps: gsave currentpoint currentpoint translate
270 rotate neg exch neg exch translate}%
\makebox(0,0)[b]{\shortstack{{\large Giant Magnetoresistance}}}%
\special{ps: currentpoint grestore moveto}%
}
\put(3417,151){\makebox(0,0){14}}
\put(3015,151){\makebox(0,0){12}}
\put(2612,151){\makebox(0,0){10}}
\put(2210,151){\makebox(0,0){8}}
\put(1807,151){\makebox(0,0){6}}
\put(1405,151){\makebox(0,0){4}}
\put(1002,151){\makebox(0,0){2}}
\put(600,151){\makebox(0,0){0}}
\put(540,1963){\makebox(0,0)[r]{10.0}}
\put(540,1621){\makebox(0,0)[r]{8.0}}
\put(540,1278){\makebox(0,0)[r]{6.0}}
\put(540,936){\makebox(0,0)[r]{4.0}}
\put(540,593){\makebox(0,0)[r]{2.0}}
\put(540,251){\makebox(0,0)[r]{0.0}}
\end{picture}
\caption{Calculated CIP-GMR for $Fe_3Cr_n$ versus $Cr$ layer thickness 
$n=1,13$. The results assuming $Cr$ defects in the $Fe$ layers are indicated 
by squares ($\beta=0.11$). Stars mark the results for spin-independent 
relaxation times $\beta=1$ and triangles for $\beta=\beta_{min}$.\label{fig14}}
\end{center}
\end{figure}
%
%%%%%%%%%%%%%%%%%%%%%%%%%%%%%%%%%%%%%%%%%%%%%%%%%%%%%%%%%%%%%%%%%%%%%%%%%%%%%%%
%
%
%%%%%%%%%%%%%%%%%%%%%%%%%%%%%%%%%%%%%%%%%%%%%%%%%%%%%%%%%%%%%%%%%%%%%%%%%%%%%%%
%%%   Abb. 4                                                                %%%
%%%%%%%%%%%%%%%%%%%%%%%%%%%%%%%%%%%%%%%%%%%%%%%%%%%%%%%%%%%%%%%%%%%%%%%%%%%%%%%
%
\begin{figure}\begin{center}
\setlength{\unitlength}{0.1bp}
\begin{picture}(3600,2014)(0,-100)
\special{"
%%Page: 1 1
gnudict begin
gsave
50 50 translate
0.100 0.100 scale
0 setgray
/Helvetica findfont 100 scalefont setfont
newpath
-500.000000 -500.000000 translate
LTa
600 251 M
2817 0 V
600 251 M
0 1712 V
LTb
LTa
600 251 M
2817 0 V
LTb
600 251 M
63 0 V
2754 0 R
-63 0 V
LTa
600 496 M
2817 0 V
LTb
600 496 M
63 0 V
2754 0 R
-63 0 V
LTa
600 740 M
2817 0 V
LTb
600 740 M
63 0 V
2754 0 R
-63 0 V
LTa
600 985 M
2817 0 V
LTb
600 985 M
63 0 V
2754 0 R
-63 0 V
LTa
600 1229 M
2817 0 V
LTb
600 1229 M
63 0 V
2754 0 R
-63 0 V
LTa
600 1474 M
2817 0 V
LTb
600 1474 M
63 0 V
2754 0 R
-63 0 V
LTa
600 1718 M
2817 0 V
LTb
600 1718 M
63 0 V
2754 0 R
-63 0 V
LTa
600 1963 M
2817 0 V
LTb
600 1963 M
63 0 V
2754 0 R
-63 0 V
LTa
600 251 M
0 1712 V
LTb
600 251 M
0 63 V
0 1649 R
0 -63 V
LTa
801 251 M
0 1712 V
LTb
801 251 M
0 63 V
0 1649 R
0 -63 V
LTa
1002 251 M
0 1712 V
LTb
1002 251 M
0 63 V
0 1649 R
0 -63 V
LTa
1204 251 M
0 1712 V
LTb
1204 251 M
0 63 V
0 1649 R
0 -63 V
LTa
1405 251 M
0 1712 V
LTb
1405 251 M
0 63 V
0 1649 R
0 -63 V
LTa
1606 251 M
0 1712 V
LTb
1606 251 M
0 63 V
0 1649 R
0 -63 V
LTa
1807 251 M
0 1712 V
LTb
1807 251 M
0 63 V
0 1649 R
0 -63 V
LTa
2009 251 M
0 1712 V
LTb
2009 251 M
0 63 V
0 1649 R
0 -63 V
LTa
2210 251 M
0 1712 V
LTb
2210 251 M
0 63 V
0 1649 R
0 -63 V
LTa
2411 251 M
0 1712 V
LTb
2411 251 M
0 63 V
0 1649 R
0 -63 V
LTa
2612 251 M
0 1712 V
LTb
2612 251 M
0 63 V
0 1649 R
0 -63 V
LTa
2813 251 M
0 1712 V
LTb
2813 251 M
0 63 V
0 1649 R
0 -63 V
LTa
3015 251 M
0 1712 V
LTb
3015 251 M
0 63 V
0 1649 R
0 -63 V
LTa
3216 251 M
0 1712 V
LTb
3216 251 M
0 63 V
0 1649 R
0 -63 V
LTa
3417 251 M
0 1712 V
LTb
3417 251 M
0 63 V
0 1649 R
0 -63 V
600 251 M
2817 0 V
0 1712 V
-2817 0 V
600 251 L
LT0
801 1332 M
201 514 V
202 -719 V
201 464 V
201 98 V
201 -504 V
202 411 V
201 34 V
201 -176 V
201 10 V
201 -406 V
202 215 V
201 294 V
801 1332 BB
1002 1846 BB
1204 1127 BB
1405 1591 BB
1606 1689 BB
1807 1185 BB
2009 1596 BB
2210 1630 BB
2411 1454 BB
2612 1464 BB
2813 1058 BB
3015 1273 BB
3216 1567 BB
LT0
801 536 M
201 86 V
1204 418 L
201 84 V
201 -42 V
201 -65 V
202 52 V
201 11 V
201 -56 V
201 7 V
201 -52 V
202 25 V
201 69 V
801 536 TT
1002 622 TT
1204 418 TT
1405 502 TT
1606 460 TT
1807 395 TT
2009 447 TT
2210 458 TT
2411 402 TT
2612 409 TT
2813 357 TT
3015 382 TT
3216 451 TT
LT0
801 590 M
201 102 V
1204 485 L
201 102 V
201 -11 V
201 -99 V
202 80 V
201 10 V
201 -49 V
201 4 V
201 -81 V
202 42 V
201 72 V
801 590 S
1002 692 S
1204 485 S
1405 587 S
1606 576 S
1807 477 S
2009 557 S
2210 567 S
2411 518 S
2612 522 S
2813 441 S
3015 483 S
3216 555 S
stroke
grestore
end
showpage
}
\put(2008,-49){\makebox(0,0){{\large Cr Thickness (ML)}}}
\put(100,1107){%
\special{ps: gsave currentpoint currentpoint translate
270 rotate neg exch neg exch translate}%
\makebox(0,0)[b]{\shortstack{{\large Giant Magnetoresistance}}}%
\special{ps: currentpoint grestore moveto}%
}
\put(3417,151){\makebox(0,0){14}}
\put(3015,151){\makebox(0,0){12}}
\put(2612,151){\makebox(0,0){10}}
\put(2210,151){\makebox(0,0){8}}
\put(1807,151){\makebox(0,0){6}}
\put(1405,151){\makebox(0,0){4}}
\put(1002,151){\makebox(0,0){2}}
\put(600,151){\makebox(0,0){0}}
\put(540,1963){\makebox(0,0)[r]{{\large $35.0$}}}
\put(540,1718){\makebox(0,0)[r]{{\large $30.0$}}}
\put(540,1474){\makebox(0,0)[r]{{\large $25.0$}}}
\put(540,1229){\makebox(0,0)[r]{{\large $20.0$}}}
\put(540,985){\makebox(0,0)[r]{{\large $15.0$}}}
\put(540,740){\makebox(0,0)[r]{{\large $10.0$}}}
\put(540,496){\makebox(0,0)[r]{{\large $5.0$}}}
\put(540,251){\makebox(0,0)[r]{{\large $0.0$}}}
\end{picture}
\caption{Calculated CPP-GMR for $Fe_3Cr_n$ versus $Cr$ layer thickness 
$n=1,13$. The results assuming $Cr$ defects in the $Fe$ layers are indicated 
by squares ($\beta=0.11$). Stars mark the results for spin-independent 
relaxation times $\beta=1$ and triangles for $\beta=\beta_{min}$.\label{fig15}}
\end{center}\end{figure}
%
%%%%%%%%%%%%%%%%%%%%%%%%%%%%%%%%%%%%%%%%%%%%%%%%%%%%%%%%%%%%%%%%%%%%%%%%%%%%%%%
%
In this case the lifetimes $\tau^\sigma$ for both spin directions must be
equal since the spin up wavefunction amplitudes of one half
of the $Cr$
atoms is equal to the spin down wavefunction amplitudes of the other half.
This corresponds to $\beta=1$.
Eq. \ref{eq67} now implies $\tau^{AP}=\tau^
{\uparrow}=\tau^\downarrow$.
Hence the lifetime completely
disappears from the GMR expression
\begin{equation}\label{eq68}
{\rm GMR}=\frac
{\sum_{\sigma}\sum_{k}\delta (E _{k}^\sigma -E_F)
\: {v^{\sigma} _{k}}_\mu \: {v^{\sigma} _{k}}_\mu}
{\;2\;\;\sum_{k} \delta (E_k^{AP}-E_F) {v_{k}}_\mu^{AP} \: {v_{k}}_\mu^{AP}} -1 .
\end{equation}
Here, the GMR is fully determined by the Fermi surface and Fermi
velocities as functions of the magnetization configuration.
Hence it is a pure bandstructure effect.
The obtained GMR values are about $200 \%$
in CIP and $700 \%$ in CPP (stars in Fig. \ref{fig14} and \ref{fig15}),
which is in agreement with the experimentally obtained maximum CIP-GMR
($220 \%$) for ultrathin $Fe$ layers \cite{schad94}.\par
Simplifying Eq. \ref{eq68} for further discussion of the origin of GMR
leads to
\begin{equation}\label{eq69}
GMR=\frac{\sum_{\sigma}N^\sigma(E_F) {v_\mu^{\sigma}}^2}
         {N_{AP}(E_F)\;\; {{{v}_\mu^{AP}}^2}} -1
\end{equation}
with the density of states of the superlattice for parallel alignment
\begin{equation}\label{eq70}
N_P(E_F)=\sum_{\sigma}\;N^\sigma(E_F)=\sum_{\sigma}\sum_{k}
\delta (E_{k}^\sigma - E_F)
\end{equation}
and for anti-parallel alignment
\begin{equation}\label{eq71}
N_{AP}(E_F)=2\;\sum _{k} \delta (E_{k} -E_F).
\end{equation}
${v_\mu^\sigma}$ and ${v_\mu^{AP}}$ are Fermi surface averages
of the Cartesian components of the velocity for parallel
\begin{equation}\label{eq72}
{v_\mu^\sigma}\;=\;\sqrt{\frac
{\sum _{k} \delta (E _{k}^\sigma - E_F)\:{v^\sigma_{k_\mu}}^2}
{\sum _{k} \delta (E _{k}^\sigma - E_F)}}
\end{equation}
and anti-parallel alignment
\begin{equation}\label{eq73}
{v_\mu^{AP}}\;=\;\sqrt{\frac
{\sum _{k} \delta (E _{k}^{AP}- E_F)\:{v^{AP}_{k_\mu}}^2}
{\sum _{k} \delta (E _{k}^{AP}- E_F)}}.
\end{equation}\par
%
%%%%%%%%%%%%%%%%%%%%%%%%%%%%%%%%%%%%%%%%%%%%%%%%%%%%%%%%%%%%%%%%%%%%%%%%%%%%%%%
%%%   Abb. 5                                                                %%%
%%%%%%%%%%%%%%%%%%%%%%%%%%%%%%%%%%%%%%%%%%%%%%%%%%%%%%%%%%%%%%%%%%%%%%%%%%%%%%%
%
\begin{figure}\begin{center}
\setlength{\unitlength}{0.1bp}
\begin{picture}(3600,2014)(0,-100)
\special{"
%%Page: 1 1
gnudict begin
gsave
50 50 translate
0.100 0.100 scale
0 setgray
/Helvetica findfont 100 scalefont setfont
newpath
-500.000000 -500.000000 translate
LTa
600 251 M
2817 0 V
600 251 M
0 1712 V
LTb
LTa
600 251 M
2817 0 V
LTb
600 251 M
63 0 V
2754 0 R
-63 0 V
LTa
600 394 M
2817 0 V
LTb
600 394 M
63 0 V
2754 0 R
-63 0 V
LTa
600 536 M
2817 0 V
LTb
600 536 M
63 0 V
2754 0 R
-63 0 V
LTa
600 679 M
2817 0 V
LTb
600 679 M
63 0 V
2754 0 R
-63 0 V
LTa
600 822 M
2817 0 V
LTb
600 822 M
63 0 V
2754 0 R
-63 0 V
LTa
600 964 M
2817 0 V
LTb
600 964 M
63 0 V
2754 0 R
-63 0 V
LTa
600 1107 M
2817 0 V
LTb
600 1107 M
63 0 V
2754 0 R
-63 0 V
LTa
600 1250 M
2817 0 V
LTb
600 1250 M
63 0 V
2754 0 R
-63 0 V
LTa
600 1392 M
2817 0 V
LTb
600 1392 M
63 0 V
2754 0 R
-63 0 V
LTa
600 1535 M
2817 0 V
LTb
600 1535 M
63 0 V
2754 0 R
-63 0 V
LTa
600 1678 M
2817 0 V
LTb
600 1678 M
63 0 V
2754 0 R
-63 0 V
LTa
600 1820 M
2817 0 V
LTb
600 1820 M
63 0 V
2754 0 R
-63 0 V
LTa
600 1963 M
2817 0 V
LTb
600 1963 M
63 0 V
2754 0 R
-63 0 V
LTa
600 251 M
0 1712 V
LTb
600 251 M
0 63 V
0 1649 R
0 -63 V
LTa
801 251 M
0 1712 V
LTb
801 251 M
0 63 V
0 1649 R
0 -63 V
LTa
1002 251 M
0 1712 V
LTb
1002 251 M
0 63 V
0 1649 R
0 -63 V
LTa
1204 251 M
0 1712 V
LTb
1204 251 M
0 63 V
0 1649 R
0 -63 V
LTa
1405 251 M
0 1712 V
LTb
1405 251 M
0 63 V
0 1649 R
0 -63 V
LTa
1606 251 M
0 1712 V
LTb
1606 251 M
0 63 V
0 1649 R
0 -63 V
LTa
1807 251 M
0 1712 V
LTb
1807 251 M
0 63 V
0 1649 R
0 -63 V
LTa
2009 251 M
0 1712 V
LTb
2009 251 M
0 63 V
0 1649 R
0 -63 V
LTa
2210 251 M
0 1712 V
LTb
2210 251 M
0 63 V
0 1649 R
0 -63 V
LTa
2411 251 M
0 1712 V
LTb
2411 251 M
0 63 V
0 1649 R
0 -63 V
LTa
2612 251 M
0 1712 V
LTb
2612 251 M
0 63 V
0 1649 R
0 -63 V
LTa
2813 251 M
0 1712 V
LTb
2813 251 M
0 63 V
0 1649 R
0 -63 V
LTa
3015 251 M
0 1712 V
LTb
3015 251 M
0 63 V
0 1649 R
0 -63 V
LTa
3216 251 M
0 1712 V
LTb
3216 251 M
0 63 V
0 1649 R
0 -63 V
LTa
3417 251 M
0 1712 V
LTb
3417 251 M
0 63 V
0 1649 R
0 -63 V
600 251 M
2817 0 V
0 1712 V
-2817 0 V
600 251 L
LT0
801 1578 M
201 117 V
202 -420 V
201 60 V
201 -225 V
201 465 V
202 -217 V
201 89 V
201 -115 V
201 57 V
201 -19 V
202 -52 V
201 -208 V
801 1578 BB
1002 1695 BB
1204 1275 BB
1405 1335 BB
1606 1110 BB
1807 1575 BB
2009 1358 BB
2210 1447 BB
2411 1332 BB
2612 1389 BB
2813 1370 BB
3015 1318 BB
3216 1110 BB
LT0
801 1833 M
201 -258 V
202 -215 V
201 282 V
201 -352 V
201 248 V
202 -384 V
201 360 V
201 -327 V
201 153 V
201 -179 V
202 47 V
201 -33 V
801 1833 C
1002 1575 C
1204 1360 C
1405 1642 C
1606 1290 C
1807 1538 C
2009 1154 C
2210 1514 C
2411 1187 C
2612 1340 C
2813 1161 C
3015 1208 C
3216 1175 C
LT0
801 1300 M
201 -192 V
1204 826 L
201 359 V
1606 812 L
201 262 V
2009 695 L
201 363 V
2411 745 L
201 98 V
2813 743 L
202 20 V
201 -6 V
801 1300 DD
1002 1108 DD
1204 826 DD
1405 1185 DD
1606 812 DD
1807 1074 DD
2009 695 DD
2210 1058 DD
2411 745 DD
2612 843 DD
2813 743 DD
3015 763 DD
3216 757 DD
LT0
801 785 M
201 -67 V
202 67 V
201 -77 V
201 21 V
201 -14 V
202 -5 V
201 -4 V
201 -13 V
201 54 V
201 -78 V
202 27 V
201 -27 V
801 785 TT
1002 718 TT
1204 785 TT
1405 708 TT
1606 729 TT
1807 715 TT
2009 710 TT
2210 706 TT
2411 693 TT
2612 747 TT
2813 669 TT
3015 696 TT
3216 669 TT
stroke
grestore
end
showpage
}
\put(2008,-49){\makebox(0,0){{\large Cr Thickness (ML)}}}
\put(100,1107){%
\special{ps: gsave currentpoint currentpoint translate
270 rotate neg exch neg exch translate}%
\makebox(0,0)[b]{\shortstack{{\large DOS at Fermi level}}}%
\special{ps: currentpoint grestore moveto}%
}
\put(3417,151){\makebox(0,0){14}}
\put(3015,151){\makebox(0,0){12}}
\put(2612,151){\makebox(0,0){10}}
\put(2210,151){\makebox(0,0){8}}
\put(1807,151){\makebox(0,0){6}}
\put(1405,151){\makebox(0,0){4}}
\put(1002,151){\makebox(0,0){2}}
\put(600,151){\makebox(0,0){0}}
\put(540,1963){\makebox(0,0)[r]{120.0}}
\put(540,1678){\makebox(0,0)[r]{100.0}}
\put(540,1392){\makebox(0,0)[r]{80.0}}
\put(540,1107){\makebox(0,0)[r]{60.0}}
\put(540,822){\makebox(0,0)[r]{40.0}}
\put(540,536){\makebox(0,0)[r]{20.0}}
\put(540,251){\makebox(0,0)[r]{0.0}}
\end{picture}
\caption{$N_{P}(E_F)$ (squares) and $N_{AP}(E_F)$ (times)
of $Fe_3Cr_n$ superlattices versus $Cr$ layer thickness in arbitrary units.
The spin-projected densities of states of $N_P(E_F)$ are marked as follows: 
$N^\uparrow(E_F)$ (diamonds) and $N^\downarrow(E_F)$ (triangles) 
\protect\cite{mertig95}.\label{fig16}}
\end{center}\end{figure}
%
%%%%%%%%%%%%%%%%%%%%%%%%%%%%%%%%%%%%%%%%%%%%%%%%%%%%%%%%%%%%%%%%%%%%%%%%%%%%%%%
%
\par
%
%%%%%%%%%%%%%%%%%%%%%%%%%%%%%%%%%%%%%%%%%%%%%%%%%%%%%%%%%%%%%%%%%%%%%%%%%%%%%%%
%%%   Abb. 6                                                                %%%
%%%%%%%%%%%%%%%%%%%%%%%%%%%%%%%%%%%%%%%%%%%%%%%%%%%%%%%%%%%%%%%%%%%%%%%%%%%%%%%
%
\begin{figure}\begin{center}
\setlength{\unitlength}{0.1bp}
\begin{picture}(3600,2014)(0,-100)
\special{"
%%Page: 1 1
gnudict begin
gsave
50 50 translate
0.100 0.100 scale
0 setgray
/Helvetica findfont 100 scalefont setfont
newpath
-500.000000 -500.000000 translate
LTa
600 251 M
2817 0 V
600 251 M
0 1712 V
LTb
LTa
600 251 M
2817 0 V
LTb
600 251 M
63 0 V
2754 0 R
-63 0 V
LTa
600 394 M
2817 0 V
LTb
600 394 M
63 0 V
2754 0 R
-63 0 V
LTa
600 536 M
2817 0 V
LTb
600 536 M
63 0 V
2754 0 R
-63 0 V
LTa
600 679 M
2817 0 V
LTb
600 679 M
63 0 V
2754 0 R
-63 0 V
LTa
600 822 M
2817 0 V
LTb
600 822 M
63 0 V
2754 0 R
-63 0 V
LTa
600 964 M
2817 0 V
LTb
600 964 M
63 0 V
2754 0 R
-63 0 V
LTa
600 1107 M
2817 0 V
LTb
600 1107 M
63 0 V
2754 0 R
-63 0 V
LTa
600 1250 M
2817 0 V
LTb
600 1250 M
63 0 V
2754 0 R
-63 0 V
LTa
600 1392 M
2817 0 V
LTb
600 1392 M
63 0 V
2754 0 R
-63 0 V
LTa
600 1535 M
2817 0 V
LTb
600 1535 M
63 0 V
2754 0 R
-63 0 V
LTa
600 1678 M
2817 0 V
LTb
600 1678 M
63 0 V
2754 0 R
-63 0 V
LTa
600 1820 M
2817 0 V
LTb
600 1820 M
63 0 V
2754 0 R
-63 0 V
LTa
600 1963 M
2817 0 V
LTb
600 1963 M
63 0 V
2754 0 R
-63 0 V
LTa
600 251 M
0 1712 V
LTb
600 251 M
0 63 V
0 1649 R
0 -63 V
LTa
801 251 M
0 1712 V
LTb
801 251 M
0 63 V
0 1649 R
0 -63 V
LTa
1002 251 M
0 1712 V
LTb
1002 251 M
0 63 V
0 1649 R
0 -63 V
LTa
1204 251 M
0 1712 V
LTb
1204 251 M
0 63 V
0 1649 R
0 -63 V
LTa
1405 251 M
0 1712 V
LTb
1405 251 M
0 63 V
0 1649 R
0 -63 V
LTa
1606 251 M
0 1712 V
LTb
1606 251 M
0 63 V
0 1649 R
0 -63 V
LTa
1807 251 M
0 1712 V
LTb
1807 251 M
0 63 V
0 1649 R
0 -63 V
LTa
2009 251 M
0 1712 V
LTb
2009 251 M
0 63 V
0 1649 R
0 -63 V
LTa
2210 251 M
0 1712 V
LTb
2210 251 M
0 63 V
0 1649 R
0 -63 V
LTa
2411 251 M
0 1712 V
LTb
2411 251 M
0 63 V
0 1649 R
0 -63 V
LTa
2612 251 M
0 1712 V
LTb
2612 251 M
0 63 V
0 1649 R
0 -63 V
LTa
2813 251 M
0 1712 V
LTb
2813 251 M
0 63 V
0 1649 R
0 -63 V
LTa
3015 251 M
0 1712 V
LTb
3015 251 M
0 63 V
0 1649 R
0 -63 V
LTa
3216 251 M
0 1712 V
LTb
3216 251 M
0 63 V
0 1649 R
0 -63 V
LTa
3417 251 M
0 1712 V
LTb
3417 251 M
0 63 V
0 1649 R
0 -63 V
600 251 M
2817 0 V
0 1712 V
-2817 0 V
600 251 L
LT0
801 964 M
201 -71 V
202 143 V
201 -29 V
201 29 V
1807 879 L
202 43 V
201 42 V
201 43 V
201 0 V
201 0 V
202 14 V
201 115 V
801 964 B
1002 893 B
1204 1036 B
1405 1007 B
1606 1036 B
1807 879 B
2009 922 B
2210 964 B
2411 1007 B
2612 1007 B
2813 1007 B
3015 1021 B
3216 1136 B
LT0
801 1078 M
201 43 V
202 100 V
201 -171 V
201 0 V
1807 864 L
202 172 V
2210 836 L
201 143 V
2612 864 L
201 157 V
3015 907 L
201 100 V
801 1078 D
1002 1121 D
1204 1221 D
1405 1050 D
1606 1050 D
1807 864 D
2009 1036 D
2210 836 D
2411 979 D
2612 864 D
2813 1021 D
3015 907 D
3216 1007 D
LT0
801 1464 M
201 185 V
202 -257 V
201 229 V
201 114 V
201 -143 V
202 143 V
201 -129 V
201 143 V
201 -214 V
201 200 V
202 -229 V
201 229 V
801 1464 T
1002 1649 T
1204 1392 T
1405 1621 T
1606 1735 T
1807 1592 T
2009 1735 T
2210 1606 T
2411 1749 T
2612 1535 T
2813 1735 T
3015 1506 T
3216 1735 T
LT0
801 608 M
201 -43 V
202 100 V
201 -15 V
201 29 V
201 0 V
202 -29 V
201 -14 V
201 57 V
201 -14 V
201 71 V
202 -71 V
201 29 V
801 608 BB
1002 565 BB
1204 665 BB
1405 650 BB
1606 679 BB
1807 679 BB
2009 650 BB
2210 636 BB
2411 693 BB
2612 679 BB
2813 750 BB
3015 679 BB
3216 708 BB
LT0
801 907 M
201 0 V
202 -85 V
201 -72 V
201 -71 V
201 29 V
202 42 V
2210 636 L
201 0 V
201 0 V
201 114 V
3015 636 L
201 114 V
801 907 DD
1002 907 DD
1204 822 DD
1405 750 DD
1606 679 DD
1807 708 DD
2009 750 DD
2210 636 DD
2411 636 DD
2612 636 DD
2813 750 DD
3015 636 DD
3216 750 DD
LT0
801 1435 M
201 214 V
202 -314 V
201 371 V
201 -43 V
201 43 V
202 14 V
201 0 V
201 72 V
201 -86 V
201 72 V
202 -172 V
201 172 V
801 1435 TT
1002 1649 TT
1204 1335 TT
1405 1706 TT
1606 1663 TT
1807 1706 TT
2009 1720 TT
2210 1720 TT
2411 1792 TT
2612 1706 TT
2813 1778 TT
3015 1606 TT
3216 1778 TT
stroke
grestore
end
showpage
}
\put(2008,-49){\makebox(0,0){{\large Cr Thickness (ML)}}}
\put(100,1107){%
\special{ps: gsave currentpoint currentpoint translate
270 rotate neg exch neg exch translate}%
\makebox(0,0)[b]{\shortstack{{\large Fermi Velocity}}}%
\special{ps: currentpoint grestore moveto}%
}
\put(3417,151){\makebox(0,0){14}}
\put(3015,151){\makebox(0,0){12}}
\put(2612,151){\makebox(0,0){10}}
\put(2210,151){\makebox(0,0){8}}
\put(1807,151){\makebox(0,0){6}}
\put(1405,151){\makebox(0,0){4}}
\put(1002,151){\makebox(0,0){2}}
\put(600,151){\makebox(0,0){0}}
\put(540,1963){\makebox(0,0)[r]{0.12}}
\put(540,1678){\makebox(0,0)[r]{0.10}}
\put(540,1392){\makebox(0,0)[r]{0.08}}
\put(540,1107){\makebox(0,0)[r]{0.06}}
\put(540,822){\makebox(0,0)[r]{0.04}}
\put(540,536){\makebox(0,0)[r]{0.02}}
\put(540,251){\makebox(0,0)[r]{0.00}}
\end{picture}
\caption{Fermi surface averages of the velocity components. Open symbols 
indicate $v_x$ and full symbols $v_z$. Diamonds mark the velocity of 
majority electrons, triangles the velocity of minority electrons and
squares the velocity in the anti-parallel aligned configuration 
\protect\cite{mertig95}.\label{fig17}}
\end{center}\end{figure}
%
%%%%%%%%%%%%%%%%%%%%%%%%%%%%%%%%%%%%%%%%%%%%%%%%%%%%%%%%%%%%%%%%%%%%%%%%%%%%%%%
%
The densities of states in the parallel and anti-parallel configuration
(Fig. \ref{fig16}) are of
the same order and they oscillate in phase with the $Cr$ layer thickness, that is,
they do not account for the $200 \%$ CIP-GMR and $700 \%$ in CPP-GMR
(Fig. \ref{fig14} and \ref{fig15}). Obviously, GMR is originated by
the differences in the Fermi velocities in the parallel and anti-parallel
configurations (Fig. \ref{fig17}).\par
%
The spin dependence of the relaxation time ($\beta\neq1$) due to scattering
centers in
the ferromagnetic layers ($Cr$ impurities in $Fe$ layers, for example)
leads to modifications. GMR is now to be calculated via
Eq. \ref{eq63}, \ref{eq64} and \ref{eq66}.
Under the assumption that the extension of the impurity is small
compared to the layer thickness,
the scattering properties are described by the spin-dependent
relaxation times calculated for the
$Cr$ impurities in $Fe$ \cite{mertig93,mertig93a}.
For this case $\beta=0.11$, that is, the majority electrons are
scattered strongly at a $Cr$ defect and the minority electrons
are just weakly scattered as they pass the defect.\par
The results assuming $Cr$ defects located in the $Fe$
layers lead to a GMR of about $600 \%$ in CIP (squares in Fig. \ref{fig14}) and
$2500 \%$ in CPP (squares in Fig. \ref{fig15}) which is larger than the experimental
results \cite{gijs92,gijs93,pratt91,gijs95}.
The anisotropy $\beta$ obtained in \cite{mertig93,mertig93a} for
a variety of defects scatter from $0.1$ to $10$.
Therefore, GMR is discussed as a function of $\beta$ in relaxation time
approximation.
If only one type of scatterer is included, GMR becomes a minimum for
\begin{equation}\label{eq74}
\beta_{min}\;=\;\frac{\tau^\uparrow}{\tau^\downarrow}=
\sqrt\frac{\sum_k\delta (E^\downarrow_{k}- E_F)\:{v^\downarrow_k}^2_\mu}
          {\sum_k\delta (E^\uparrow_{k}- E_F)\:{v^\uparrow_k}^2_\mu}.
\end{equation}
That is, GMR can be enhanced or
reduced by spin-de\-pen\-dent impurity scattering.
The results for $\beta=\beta_{min}$ are shown in Fig. \ref{fig14} and
\ref{fig15} (triangles).\par
Finally, we discuss the combination
of alternating $Fe$ layers with $Cr$ defects ($\beta=0.11$) and with
$Cu$ defects ($\beta=3.68$).
For the sake of simplicity, equal concentration is assumed.
The relaxation times are in parallel configuration
\begin{equation}
\frac{1}{\tau^{\sigma}}=\frac{1}{2}\;(\;\frac{1}{\tau^\sigma_{Cr}}\;+\;\frac{1}
{\tau^\sigma_{Cu}}\;),
\end{equation}
and in anti-parallel configuration
\begin{equation}
\frac{D^+_\uparrow+D^-_\uparrow}{\tau^{\uparrow}}=(\;\frac{D^+_\uparrow}
{\tau^{+}_{Cr}}\;+\;\frac{D^-_\uparrow}
{\tau^-_{Cu}}\;),
\end{equation}
and
\begin{equation}
\frac{D^+_\downarrow+D^-_\downarrow}{\tau^{\downarrow}}=(\;\frac{D^-_\downarrow}
{\tau^-_{Cr}}\;+\;\frac{D^+_\downarrow}
{\tau^+_{Cu}}\;),
\end{equation}
respectively where the superscripts $+,-$ correspond to majority and minority
electrons.
$D^+_\sigma$ and $D^-_\sigma$ is the local spin-dependent density of states in
a ferromagnetic layer where spin-$\sigma$ electrons are majority or minority
electrons, respectively. In the considered configuration the factors
have been set to unity.
The conductivity in the anti-parallel state including
both defects is given by
\begin{equation}
{\hat \sigma}^{AP} =  e^2
(\tau^{\uparrow}+\tau^{\downarrow}) \sum _{k} \delta (E_{k}^{AP} -
E_F)\: {\bf v}^{AP}_k\: {\bf v}^{AP}_k.
\end{equation}
\par
%
%%%%%%%%%%%%%%%%%%%%%%%%%%%%%%%%%%%%%%%%%%%%%%%%%%%%%%%%%%%%%%%%%%%%%%%%%%%%%%%
%%%   Abb. 7                                                                %%%
%%%%%%%%%%%%%%%%%%%%%%%%%%%%%%%%%%%%%%%%%%%%%%%%%%%%%%%%%%%%%%%%%%%%%%%%%%%%%%%
%
\begin{figure}\begin{center}
\setlength{\unitlength}{0.1bp}
\begin{picture}(3600,2014)(0,-100)
\special{"
%%Page: 1 1
gnudict begin
gsave
50 50 translate
0.100 0.100 scale
0 setgray
/Helvetica findfont 100 scalefont setfont
newpath
-500.000000 -500.000000 translate
LTa
600 562 M
2817 0 V
600 251 M
0 1712 V
LTb
LTa
600 251 M
2817 0 V
LTb
600 251 M
63 0 V
2754 0 R
-63 0 V
LTa
600 407 M
2817 0 V
LTb
600 407 M
63 0 V
2754 0 R
-63 0 V
LTa
600 562 M
2817 0 V
LTb
600 562 M
63 0 V
2754 0 R
-63 0 V
LTa
600 718 M
2817 0 V
LTb
600 718 M
63 0 V
2754 0 R
-63 0 V
LTa
600 874 M
2817 0 V
LTb
600 874 M
63 0 V
2754 0 R
-63 0 V
LTa
600 1029 M
2817 0 V
LTb
600 1029 M
63 0 V
2754 0 R
-63 0 V
LTa
600 1185 M
2817 0 V
LTb
600 1185 M
63 0 V
2754 0 R
-63 0 V
LTa
600 1340 M
2817 0 V
LTb
600 1340 M
63 0 V
2754 0 R
-63 0 V
LTa
600 1496 M
2817 0 V
LTb
600 1496 M
63 0 V
2754 0 R
-63 0 V
LTa
600 1652 M
2817 0 V
LTb
600 1652 M
63 0 V
2754 0 R
-63 0 V
LTa
600 1807 M
2817 0 V
LTb
600 1807 M
63 0 V
2754 0 R
-63 0 V
LTa
600 1963 M
2817 0 V
LTb
600 1963 M
63 0 V
2754 0 R
-63 0 V
LTa
600 251 M
0 1712 V
LTb
600 251 M
0 63 V
0 1649 R
0 -63 V
LTa
801 251 M
0 1712 V
LTb
801 251 M
0 63 V
0 1649 R
0 -63 V
LTa
1002 251 M
0 1712 V
LTb
1002 251 M
0 63 V
0 1649 R
0 -63 V
LTa
1204 251 M
0 1712 V
LTb
1204 251 M
0 63 V
0 1649 R
0 -63 V
LTa
1405 251 M
0 1712 V
LTb
1405 251 M
0 63 V
0 1649 R
0 -63 V
LTa
1606 251 M
0 1712 V
LTb
1606 251 M
0 63 V
0 1649 R
0 -63 V
LTa
1807 251 M
0 1712 V
LTb
1807 251 M
0 63 V
0 1649 R
0 -63 V
LTa
2009 251 M
0 1712 V
LTb
2009 251 M
0 63 V
0 1649 R
0 -63 V
LTa
2210 251 M
0 1712 V
LTb
2210 251 M
0 63 V
0 1649 R
0 -63 V
LTa
2411 251 M
0 1712 V
LTb
2411 251 M
0 63 V
0 1649 R
0 -63 V
LTa
2612 251 M
0 1712 V
LTb
2612 251 M
0 63 V
0 1649 R
0 -63 V
LTa
2813 251 M
0 1712 V
LTb
2813 251 M
0 63 V
0 1649 R
0 -63 V
LTa
3015 251 M
0 1712 V
LTb
3015 251 M
0 63 V
0 1649 R
0 -63 V
LTa
3216 251 M
0 1712 V
LTb
3216 251 M
0 63 V
0 1649 R
0 -63 V
LTa
3417 251 M
0 1712 V
LTb
3417 251 M
0 63 V
0 1649 R
0 -63 V
600 251 M
2817 0 V
0 1712 V
-2817 0 V
600 251 L
LT0
801 1445 M
201 368 V
202 -731 V
201 424 V
201 -87 V
201 -360 V
202 290 V
201 36 V
201 -175 V
201 17 V
2813 934 L
202 150 V
201 257 V
801 1445 B
1002 1813 B
1204 1082 B
1405 1506 B
1606 1419 B
1807 1059 B
2009 1349 B
2210 1385 B
2411 1210 B
2612 1227 B
2813 934 B
3015 1084 B
3216 1341 B
LT0
801 630 M
201 71 V
1204 587 L
201 67 V
201 60 V
201 -84 V
202 71 V
2210 568 L
201 36 V
201 -74 V
201 51 V
202 -77 V
201 61 V
801 630 BB
1002 701 BB
1204 587 BB
1405 654 BB
1606 714 BB
1807 630 BB
2009 701 BB
2210 568 BB
2411 604 BB
2612 530 BB
2813 581 BB
3015 504 BB
3216 565 BB
stroke
grestore
end
showpage
}
\put(2008,-49){\makebox(0,0){{\large Cr Thickness (ML)}}}
\put(200,1107){%
\special{ps: gsave currentpoint currentpoint translate
270 rotate neg exch neg exch translate}%
\makebox(0,0)[b]{\shortstack{{\large Giant Magnetoresistance}}}%
\special{ps: currentpoint grestore moveto}%
}
\put(3417,151){\makebox(0,0){14}}
\put(3015,151){\makebox(0,0){12}}
\put(2612,151){\makebox(0,0){10}}
\put(2210,151){\makebox(0,0){8}}
\put(1807,151){\makebox(0,0){6}}
\put(1405,151){\makebox(0,0){4}}
\put(1002,151){\makebox(0,0){2}}
\put(600,151){\makebox(0,0){0}}
\put(540,1807){\makebox(0,0)[r]{4.0}}
\put(540,1496){\makebox(0,0)[r]{3.0}}
\put(540,1185){\makebox(0,0)[r]{2.0}}
\put(540,874){\makebox(0,0)[r]{1.0}}
\put(540,562){\makebox(0,0)[r]{0.0}}
\put(540,251){\makebox(0,0)[r]{-1.0}}
\end{picture}
\caption{Calculated GMR for $Fe_3Cr_n$ with $Cr$ and $Cu$ defects in adjacent 
$Fe$ layers in CIP (closed squares) and CPP (open squares). For $n\geq 10$ 
the inverse GMR is obtained.\label{fig21}}
\end{center}\end{figure}
%
%%%%%%%%%%%%%%%%%%%%%%%%%%%%%%%%%%%%%%%%%%%%%%%%%%%%%%%%%%%%%%%%%%%%%%%%%%%%%%%
%
The results are shown in Fig. \ref{fig21}. The main message from this calculation
is that the GMR can be reduced drastically or even change sign,
if scatterers with different spin anisotropy are combined. (If they would be
located in the same layer they would merely act as an effective scattering
potential.) One should note that in Fig. \ref{fig21} the inverse
effect occurs in CIP but not in CPP.
Experimentally, the inverse effect was obtained for $Fe/Cr/Fe/Cu/Fe/Cu$
multilayers \cite{george94}, as well as in $Fe_{1-x}V_x/Au/Co$
multilayers \cite{renard95}, both in CIP geometry.\par
%
Generally, it should be noted that CPP-GMR for comparable scattering is always
larger than CIP by a factor of about $4$ in agreement with experimental results
\cite{gijs95}. This factor stems from the difference of the Fermi
velocity components in plane and perpendicular to the plane, and it
finds its natural explanation in the fact that the carriers
with large momentum components in direction of the current are more
influenced by the superstructure in CPP.\par
Moreover, GMR (Fig. \ref{fig14}, \ref{fig15}, \ref{fig21}) shows characteristic variations
with layer thicknesses.
On Fig. \ref{fig21} these variations are reminiscent of experimentally found
oscillations \cite{okuno94}.
%
\section*{Summary}
In conclusion,
we have shown that GMR, under the assumption of a coherent multilayer
system and a spin-independent impurity scattering, is determined
by the changes of the electronic structure as a function of the magnetization
direction.
Spin-dependent impurity scattering can enhance or reduce the effect.
The combination of impurities with different spin anisotropy can
in particular cause the inverse GMR.
%
\subsection*{Acknowledgement}
The authors would like to thank P. H. Dederichs and R. Zeller 
for valuable discussions. 
%
\begin{thebibliography}{m}\small

\bibitem{baibich88}
M.~N. Baibich, J.~M. Broto, A. Fert, F. {Nguyen van Dau}, F. Petroff, P.
  Etienne, G. Creuzet, A. Friederich, and J. Chazelas, Phys. Rev. Lett. {\bf
  61},  2472  (1988).

\bibitem{binash89}
G. Binash, P. Gr{\"u}nberg, F. Saurenbach, and W. Zinn, Phy. Rev. B {\bf 39},
  4828  (1989).

\bibitem{butler93}
W.~H. Butler, J.~M. MacLaren, and X.-G. Zhang, Mat. Res. Soc. Proc. {\bf 313},
  59  (1993).

\bibitem{camley89}
R.~E. Camley and J. Barna\'{s}, Phys. Rev. Lett. {\bf 63},  664  (1989).

\bibitem{george94}
J.~M. George, L.~G. Pereira, A. Barthelemy, F. Petroff, L. Steren, J.~L.
  Duvail, and A. Fert, Phys. Rev. Lett. {\bf 72},  408  (1994).

\bibitem{gijs93}
M.~A.~M. Gijs, S.~K. Lenczowski, and J.~B. Giesbers, Phys. Rev. Lett. {\bf 70},
   3343  (1993).

\bibitem{gijs95}
M.~A.~M. Gijs, S.~K.~J. Lenczowski, J.~B. Giesbers, R.~J.~M. {van de Veerdonk},
  M.~T. Johnson, R.~M. Jungblut, A. Reinders, and R.~M.~J. {van Gansewinkel}
  (unpublished).

\bibitem{gijs92}
M.~A.~M. Gijs and M. Okada, Phy. Rev. B {\bf 46},  2908  (1992).

\bibitem{hood92}
R.~Q. Hood and L.~M. Falicov, Phy. Rev. B {\bf 46},  8287  (1992).

\bibitem{inoue91}
J. Inoue, A. Oguri, and S. Maekawa, J. Phys. Soc. Jpn. {\bf 60},  376  (1991).

\bibitem{levy94}
P.~M. Levy, Sol. Stat. Phys. {\bf 47},  367  (1994).

\bibitem{levy90}
P.~M. Levy, S. Zhang, and A. Fert, Phys. Rev. Lett. {\bf 65},  1643  (1990).

\bibitem{mertig95}
I. Mertig, P. Zahn, M. Richter, H. Eschrig, R. Zeller, and P.~H. Dederichs, J.
  Mag. Mag. Mat.  (accepted)  (1995).

\bibitem{mertig93}
I. Mertig, R. Zeller, and P.~H. Dederichs, Phy. Rev. B {\bf 47},  16178
  (1993).

\bibitem{mertig93a}
I. Mertig, R. Zeller, and P.~H. Dederichs,  in {\em Metallic Alloys:
  Experimental and Theoretical Perspectives},  (Kluwer Academic Publishers,
  Dodrecht, Boston, London, 1994), p.\ 423.

\bibitem{oguchi93}
T. Oguchi, J. Mag. Mag. Mat. {\bf 126},  519  (1993).

\bibitem{okuno94}
S.~N. Okuno and K. Inomata, Phys. Rev. Lett. {\bf 72},  1553  (1994).

\bibitem{pratt91}
W.~P. {Pratt Jr.}, S.~F. Lee, J.~M. Slaughter, R. Loloee, P.~A. Schroeder, and
  J. Bass, Phys. Rev. Lett. {\bf 66},  3060  (1991).

\bibitem{renard95}
J.-P. Renard, P. Bruno, R. Megy, B. Bartenlian, P. Beauvillain, C. Chappert, C.
  Dupas, E. Kolb, M. Mulloy, P. Veillet, and E. Veu,   preprint  (1995).

\bibitem{schad94}
R. Schad, C.~D. Potter, P. Beli$\ddot{\mbox{e}}$n, G. Verbanck, V.~V.
  Moshchalkov, and Y. Bruynseraede, Appl. Phys. Lett. {\bf 64},  3500  (1994).

\bibitem{schep95}
K.~M. Schep, P.~J. Kelly, and G.~E.~W. Bauer, Phys. Rev. Lett. {\bf 74},  586
  (1995).

\bibitem{valet93}
T. Valet and A. Fert, Phy. Rev. B {\bf 48},  7099  (1993).

\bibitem{zahn95}
P. Zahn, I. Mertig, M. Richter, and H. Eschrig, Phys. Rev. Lett. {\bf 75},
  2996  (1995).

\end{thebibliography}
%

%%%%%%%%%%%%%%%%%%%%%%%%%%%%%%%%%%%%%%%%%%%%%%%%%%%%%%%%%%%%%%%%%%%%%%%%%%%%%
%                                                                           %
%      Bond Order Potentials Workshop   ............                        %
%                                                                           %
%%%%%%%%%%%%%%%%%%%%%%%%%%%%%%%%%%%%%%%%%%%%%%%%%%%%%%%%%%%%%%%%%%%%%%%%%%%%%
\newpage
\null
\begin{center}
{\Large{\it Announcement}}\\[10mm]

{\Large{\bf UNIVERSITY OF OXFORD--DEPARTMENT OF MATERIALS}} \\
{\Large{\bf MATERIALS MODELLING LABORATORY}}\\

\bigskip
{\Large{\bf BOND--ORDER POTENTIALS WORKSHOP}} \\
{\large{\it 18 AND 19 MARCH 1996}}\\
\end{center}

\vspace{1cm}
\noindent
The realistic atomistic simulation of materials requires the
development and application of O(N) methods that avoid the
crippling $N^{3}$ constraint of matrix diagonalisation for
finding the electronic structure and hence the energy of
a system.\\

\noindent
This two-day workshop presents the background theory and concepts
behind the novel Bond Order Potentials (BOPs) which allow the
Tight Binding (TB) energy to be expressed exactly as a rapidly
convergent many-atom expansion. The BOP approach has the 
advantage over other O(N) density matrix methods in that it is
equally applicable to both metals and semiconductors, and it is
naturally parallelisable.\\

\noindent
The nature of the BOP computer codes will be described and a
hands-on afternoon session will give workshop attendees experience
in simulationg a wide range of materials from hydrocarbon molecules
through to bulk semicondusctors and intermetallics. \\

\noindent
A.M. Bratkovsky \\
Department of Materials \\
University of Oxford \\
Parks Road \\
OXFORD OX1 3PH, U.K. \\
tel: 01865-283325 \\
fax: 01865-273764 \\
email: alex.bratkovsky@materials.ox.ac.uk \\
%%%%%%%%%%%%%%%%%%%%%%%%%%%%%%%%%%%%%%%%%%%%%%%%%%%%%%%%%%%%%%%%%%%%%%%%%%%%%
%                                                                           %
%      OXYGEN '96 Workshop Announcement ............                        %
%                                                                           %
%%%%%%%%%%%%%%%%%%%%%%%%%%%%%%%%%%%%%%%%%%%%%%%%%%%%%%%%%%%%%%%%%%%%%%%%%%%%%
\newpage
\null
\begin{center}
{\Large{\bf 			OXYGEN '96}}

{\large{\bf

                Early Stages of Oxygen Precipitation in Silicon\\
                        NATO Advanced Research Workshop\\
                             26th - 29th March 1996}}

{\large{\it                University of Exeter, U.K.

                         Co-sponsored by the University}}
\end{center}

{\centering{
------------------------------------------------------------------------------}}

{\large{\bf Objectives:}}

\begin{quotation} \noindent
     Topics to be covered include: Progress and problems involved with the
     early stage of oxygen precipitation in silicon; oxygen diffusion and
     clustering; oxygen-light element interactions especially H, C and N;
     oxygen-TM defects; oxygen-vacancy and interstitial complexes; thermal and
     new donors; theoretical modelling of oxygen defects and processes. A major
     aim is to bring theoretical modelling groups together with experimental
     ones to promote cross-fertilisation of ideas.
\end{quotation}

{\centering{
------------------------------------------------------------------------------}}

{\large{\bf Speakers Include:}}

\begin{quotation} \noindent
     C. A. J. Ammerlaan, P. De\'ak, S. Estreicher, C. Ewels,  U. Gosele,
     J. L. Lindstr\"om, V. P. Markevich, R. C. Newman, B. Pajot, J. M. Spaeth,
     M. Stavola, M. Suezawa, K. Sumino, G. D. Watkins, E. R. Weber, J. Weber
\end{quotation}

{\centering{
------------------------------------------------------------------------------}}

{\large{\bf Accomodation and Fees:}}

\begin{quotation} \noindent
     Full board is available from the evening of Monday March 25, 1996, through
     to Friday lunchtime on March 29th for 266 pounds sterling, including
     reception, conference dinner and an excursion. Some contribution to the
     travel and accommodation costs may be made subject to the regulations of
     NATO.
\end{quotation}

{\centering{
------------------------------------------------------------------------------}}

\newpage
{\large{\bf Proceedings:}}

\begin{quotation} \noindent
     Will be published in the NATO ASI series.
\end{quotation}

{\centering{
------------------------------------------------------------------------------}}

{\large{\bf Registration:}}

\begin{quotation} \noindent
     World Wide Web site: \verb+http://newton.ex.ac.uk/NATO/+
     It is possible to register directly using the World Wide Web, or else 
     complete the form enclosed below and return to \verb+ewels@excc.ex.ac.uk+
     or \verb+jones@excc.ex.ac.uk+
\end{quotation}

{\centering{
------------------------------------------------------------------------------}}

{\large{\bf Travel:}}

\begin{quotation} \noindent
     The University is located in Exeter, the beautiful county town of Devon in
     the South West of England. It can be reached by train from London
     (Paddington or Waterloo - 2:30 hrs). Delegates from London airport
     (Heathrow) should take the airport bus to Reading and then the train to
     Exeter (2 hrs). Alternatively coaches run from London (Victoria) or
     Heathrow directly to Exeter (3 hrs). The University is a few minutes
     taxi-ride from either the train or coach station.
\end{quotation}

{\centering{
------------------------------------------------------------------------------}}

{\large{\bf Organising Committee:}}

\begin{quotation} \noindent
     Dr. R. Jones - Department of Physics, University of Exeter, U.K.
     Prof. P. De\'ak - Department of Atomic Physics, Technical University,
     Budapest, Hungary.
     Prof. S. Estreicher - Max Planck Institute, Stuttgart, Germany.
     C. Ewels - Department of Physics, University of Exeter, U.K.
\end{quotation}

{\centering{
------------------------------------------------------------------------------}}

\begin{center}
{\large{\bf OXYGEN '96 Registration Form follows.}}
\end{center}

\newpage
\begin{verbatim}
                     OXYGEN '96 Registration Form

General Details:
     Title:  [ ]Prof [ ]Dr [ ]Mr [ ]Mrs [ ]Ms  Other ___________
     Gender: [ ]Male [ ]Female

     Your surname: _____________________________________________  
     Initials:     _______
     Institution:  _____________________________________________
                   _____________________________________________
     Your e-mail address: ______________________________________
     Telephone Number:    ______________________________________
     Fax Number:          ______________________________________

     Your postal address: ______________________________________
     ___________________________________________________________
     ___________________________________________________________
     ___________________________________________________________

     Do have any special requirements? (Diet, disability, etc.)
     ___________________________________________________________

     Areas of interest:
    ___________________________________________________________
     ___________________________________________________________
     ___________________________________________________________

     Plese quote the cost of economy travel between  London
     (Heathrow or Victoria) and  your home base:
     ___________________________________________________________

Presentation:
     I would like to submit a poster [ ] Yes [ ] No
     Title: ____________________________________________________
     ___________________________________________________________
     First Author (if different from above):
     ___________________________________________________________
     Affiliation:
     ___________________________________________________________
     ___________________________________________________________
     Co-Author(s):
     ___________________________________________________________
     ___________________________________________________________
     Affiliation(s):
     ___________________________________________________________
     ___________________________________________________________
     ___________________________________________________________
     ___________________________________________________________

Abstract: This can also be sent directly by e-mail, preferably in
          LaTeX/ReVTeX format, to ewels@excc.ex.ac.uk
\end{verbatim}

%%%%%%%%%%%%%%%%%%%%%%%%%%%%%%%%%%%%%%%%%%%%%%%%%%%%%%%%%%%%%%%%%%%%%%%%%%%%%
%                                                                           %
%      Network Conference 2nd Circular  ............                        %
%                                                                           %
%%%%%%%%%%%%%%%%%%%%%%%%%%%%%%%%%%%%%%%%%%%%%%%%%%%%%%%%%%%%%%%%%%%%%%%%%%%%%
\newpage
\null

% circular2.tex
%\documentstyle[a4]{article} 
%\documentstyle[epsf,a4]{article} 
%\topmargin -1.0cm
%\oddsidemargin 0.5cm
%\textwidth 15.4cm
%\textheight 24cm
%\headsep1.2cm
%\headheight14.2pt
%\renewcommand{\baselinestretch}{1.0}
%\renewcommand{\textfraction}{0}
%\renewcommand{\bottomfraction}{1}
%\renewcommand{\topfraction}{1}
%\pagenumbering{arabic}
%\parindent 0pt
%\begin{document}
%\vspace*{-45mm}
%\parbox[t]{4.5cm}{\epsfxsize=4.5cm \epsffile{logo3.epsi}}
\parbox[t]{15cm}{{\Huge\bf $\Psi_k$ Network Conference}\\[3mm]
{\bf  organized by the European Union HCM network}\\[3mm]
{\Large\bf"Ab initio (from electronic structure) calculation\\
of complex processes in materials"\\[6mm] {\large\bf Schw\"abisch Gm\"und,  September 18-21, 1996}}}\\[5mm]
\par
%\vspace{20mm}
{\huge\em Second Circular and call for suggestions}\\[5mm]
{\Large\bf Scientific Program}\\[5mm]
{\bf Methods}\\
\begin{itemize}
\item {\bf Quantum Monte Carlo Calculations }
\item {\bf New developments in Muffin-tin and Pseudopotential techniques }
\item {\bf Density Functional Molecular Dynamics simulation:\\ methods and applications}
\item {\bf Beyond the Local Density Approximation}
\item {\bf Quasiparticles}
\item {\bf Solving the Bogolubov-de Gennes Equations }
\item {\bf Approximate Methods for large systems}
\item {\bf Parallel Algorithms (Parallelisation)}\\[3mm]
\end{itemize}
{\bf Applications to all aspects of Condensed Matter and Materials Science including:}\\
\begin{itemize}
\item Electron spectroscopies
\item Novel superconductors
\item Magnetic multilayers
\item Oxide/metal interfaces
\item Reactions at surfaces
\item Semiconductor heterostructures and quantum wires
\item Carbon structures
\item Molecular materials
\end{itemize}
%\newpage
The conference is open to anyone interested. The expected attendance is 300.
All abstracts will be accepted, either as a contributed talk or as a
poster, subject to the limitation that each participant presents at most
one talk or poster.\\[3mm]   
{\large\bf Suggestion Form for invited speakers and symposia}\\[5mm]
Members of the network and external keypersons are asked to suggest a plenary speaker as well
as a topic, two speakers, and a chairperson for a symposium. We plan to have 8 plenary sessions
and 8-12 symposia running three in parallel.
\par
Accompaning this 2$^{nd }$ circular is a suggestion form either in print or as
electronic mail (\LaTeX ~file form.tex plus \LaTeX ~style psik.sty).
Only forms arriving by {\large\bf January 15th 1996} will be considered. We therefore recommend
the return of the completed form.tex file via electronic mail to psik@radix2.mpi-stuttgart.mpg.de.\\[5mm]
{\large\bf Timetable}\\[5mm]
\begin{tabular}{l l l }
Jan 15&	1996&	Deadline for suggestions.\\
Apr&	1996&	Third circular including details about symposia and invited speakers.\\
May&	1996&	Deadline for the submission of abstracts and registration.\\
Jun &	1996&	Final programme.\\
Sep&	1996&	Conference.\\[3mm]
\end{tabular}
\par
{\large\bf Location: Schw\"abisch Gm\"und (Germany)}\\[5mm]
Schw\"abisch Gm\"und is located in south-west Germany 50 kilometers east
of Stuttgart, the capital of the state of Baden-W\"urttemberg. It is easily
accessible by public transport from Stuttgart and from Stuttgart
international airport.
\par
The conference will be held at the Stadtgarten conference center which is
within walking distance from most of the hotels. Schw\"abisch Gm\"und is a lovely
old town with a population of 60,000 and with a history dating back to the
Roman Empire in the 2nd century. In the 12th century it became the first of
several imperial towns founded by the Hohenstaufen dynasty to protect their
empire. Many original old buildings show the transition of the former Free
Imperial City from Gothic through Renaissance to Baroque and the coexistence
of 19th century palatial grandeur with contemporary architecture.\\[3mm]
{\large\bf Local Organisation}\\[5mm]
O.K. Andersen, A. Burkhardt, O. Jepsen,\\ C. Irslinger, G. Schmidt,
Z. Szotek, W.M. Temmerman\\[3mm]
\par
{\large\bf Network Management Board and Program Committee}\\[5mm]
V. Heine	(NMB Chair, Cambridge)\\
O.K. Andersen	(Conference Chair, Stuttgart)\\[3mm]
\begin{tabular}{l l l l }
O. Bisi		& (Trento) &	P. Dederichs &	(J\"ulich)\\
P.J. Durham	& (Daresbury) &	M. Finnis &	(Belfast)\\
J. Ortega	& (Madrid) &	M. Gillan &	(Keele)\\
B.L. Gyorffy	& (Bristol) &	J. Hafner &	(Vienna)\\
J.E. Inglesfield& (Nijmegen) &	C. Koenig &	(Rennes)\\
J. Martins	& (Lissabon) &	R. Nieminen &	(Helsinki)\\
J. Norskov	& (Lyngby) &	C. Patterson &	(Dublin)\\
M. Scheffler	& (Berlin) &	N. Stefanou &	(Athens)\\
A. Svane	& (Aarhus) &	Z. Szotek &	(Daresbury)\\
W.M. Temmerman	& (Daresbury) &	V. Van Doren &	(Antwerpen)\\[3mm]
\end{tabular}

%\newpage
\noindent
{\large\bf Overseas Advisory Committee}\\[5mm]
\begin{tabular}{l l}
Peter J. Feibelman  &    (Albuquerque)\\
Arthur J. Freeman   &    (Evanston)\\
Sonia Frota-Pessoa  &    (Sao Paolo)\\
Takeo Fujiwara      &    (Tokyo)\\
Jisoon Ihm          &    (Seoul)\\
Steven G. Louie     &    (Berkeley)\\
Richard Martin      &    (Urbana)\\
Abhijit Mookerjee   &    (Calcutta)\\
Kiyoyuki Terakura   &    (Tsukuba)\\
David Vanderbilt    &    (Rutgers)\\[3mm]
\end{tabular}
\par
{\large\bf Further Information}\\[3mm]
Up to date information may be obtained from the World-Wide-Web Conference server at\\ 
{\it\bf http://radix2.mpi-stuttgart.mpg.de/hcm.html} .\\[3mm]
Correspondence should be addressed to:\\[2mm]
O. K. Andersen\\[2mm]
Max-Planck-Institut f\"ur  Festk\"orperforschung\\
Postfach 80 06 65\\
D-70506 Stuttgart\\[2mm]
FAX:    (++49)-711-6891632\\
E-mail: {\it\bf psik@radix2.mpi-stuttgart.mpg.de}

\end{document}

%%%%%%%%%%%%%%%%%%%%%%%%%%%%%%%%%%%%%%%%%%%%%%%%%%%%%%%%%%%%%%%%%%%%%%
%                                                                    %
%    Please save the two files below as form.tex and psik.sty        %
%                                                                    %
%    The first file is: form.tex                                     %
%    The second file is: psik.sty                                    %
%                                                                    %
%    The first file contains the Network Conference suggestion form. %
%    You should process this file separately by issuing command:     %
%                                                                    %
%                 latex form.tex                                     %
%                                                                    %
%    if you wish to print the suggestion form, fill it in and send   %
%    it back to the organisers via surface mail.                     %
%                                                                    %
%     This will call upon a formatting file psik.sty which is        %
%     attached just below form.tex file.                             %
%                                                                    %
%                                                                    %
%    If you prefer to send the form by electromic mail you just      %
%    need to extract the form.tex file below out of this document,   %
%    fill in the spaces in between the curly brackets and send the   %
%    form.tex back to psik@radix2.mpi-stuttgart.mpg.de.              %
%                                                                    %
%                                                                    %
%                                                                    %
%%%%%%%%%%%%%%%%%%%%%%%%%%%%%%%%%%%%%%%%%%%%%%%%%%%%%%%%%%%%%%%%%%%%%%
%                                                                    %
%%%%%%%%%%%%%%% Suggestion form for the Psi-k Conference %%%%%%%%%%%%%
%                                                                    %
%                            form.tex                                %
%                                                                    %
%%%%%%%%%%%%%%%%%%%%%%%%%%%%%%%%%%%%%%%%%%%%%%%%%%%%%%%%%%%%%%%%%%%%%%
\input psik.sty
\begintext
%
%%%%%%%%%%%%%%%%%%%%%%%%%%%%%%%%%%%%%%%%%%%%%%%%%%%%%%%%%%%%%%%%%%%%%%
% 
%
% ALL text goes INSIDE curly brackets ({}).
%
%
\no_of_speakers{3} %put number (1-3) of talks being nominated
%
\inputfirstspeakerinfo
%
% Title of the proposed symposium
%
\symposiumtitle{}
%
% See list of categories numbers on TeXed version of form.
\category{}
%
% provide the speaker for the plenary talk below
%
\speakername{}
\speakeremail{}
\speakeraddress{}
% Two references required: articles in journals (refone, etc.) or books.
% sequence of journal references: {author}{journal}{volume}{page no.}{year}
\refone{}{}{}{}{}
\reftwo{}{}{}{}{}
\refthree{}{}{}{}{}
% sequence of book references: {author, title (publisher, year) page nos.}
\bookref{}
\bookreftwo{}
\talktitle{}
% Inside of next curly brackets, include description, and
% supporting remarks (including what is new)
\text{}
%
%recommender sequence:{name}{E-mail}{institution/address}{phone}
%
\recommender{}{}{}{}
%
\printfirstspeaker
%
% chairman : {name}{address}{E-mail}{phone} in brackets,
%
\chairman{}{}{}{} 
%
% provide information about the speakers for the symposium below
%
\inputsecondspeakerinfo
\speakername{}
\speakeremail{}
\speakeraddress{}
% reference sequence:{author}{journal}{volume}{page no.}{year}
\refone{}{}{}{}{}
\reftwo{}{}{}{}{}
\refthree{}{}{}{}{}
% book sequence:{author, title, publisher, year} 
\bookref{}
\bookreftwo{}
\talktitle{}
\text{}
%
\printsecondspeaker

\inputthirdspeakerinfo
\speakername{}
\speakeremail{}
\speakeraddress{}
% sequence:{author}{journal}{volume}{page no.}{year}
\refone{}{}{}{}{}
\reftwo{}{}{}{}{}
\refthree{}{}{}{}{}
% sequence of book references: {author, title (publisher, year) page nos.} 
\bookref{}
\bookreftwo{}
\talktitle{}
\text{}
%
\printthirdspeaker
\printchair
\endtext
\bye


%%%%%%%%%%%%%%%%%%%%%%%%%%%%%%%%%%%%%%%%%%%%%%%%%%%%%%%%%%%%%%%%%%%%%%%%%%%
%                                                                         %
%  This is psik.sty file ..............                                   %
%                                                                         %
%%%%%%%%%%%%%%%%%%%%%%%%%%%%%%%%%%%%%%%%%%%%%%%%%%%%%%%%%%%%%%%%%%%%%%%%%%%
%                                                                         %
% psik.sty ...... formatting the Psi-k-Network Conference suggestion form %
%                 derived from the APS macro.sty                          %
%                                                                         %
%                 (version for electronic distribution)                   %
%  Nov 1995  Armin Burkhardt MPI/FKF burkhard@radix2.mpi-stuttgart.mpg.de %
%                                                                         %
%%%%%%%%%%%%%%%%%%%%%%%%%%%%%%%%%%%%%%%%%%%%%%%%%%%%%%%%%%%%%%%%%%%%%%%%%%%
\def\ifundefined#1{\expandafter\ifx\csname#1\endcsname\relax}
\def\iflatex#1{\ifundefined{LaTeX}\else#1\fi}
\def\iftex#1{\ifundefined{LaTeX}#1\else\fi}
\iflatex{\def\begintext{\def\endtext{\end{document}}
\documentstyle[psik]{article}\begin{document}}}
\iftex{\def\endtext{}\def\begintext{}}
\ifundefined{endtext}\endinput\fi
%----------------------ljmacro.sty----------------------------
\font\tenrm=cmr10  \font\teni=cmmi10
\font\tensy=cmsy10 \font\tenex=cmex10
\font\tenbf=cmbx10 \font\tensl=cmsl10 
\font\tentt=cmtt10 \font\tenit=cmti10

\font\sevenrm=cmr7 \font\seveni=cmmi7
\font\sevensy=cmsy7 \font\sevenbf=cmbx7

\font\fiverm=cmr5 \font\fivei=cmmi5
\font\fivesy=cmsy5 \font\fivebf=cmbx5

\def\tenpoint{\normalbaselineskip=12pt plus 0.1pt minus 0.1pt%
\abovedisplayskip 12pt plus 3pt minus 9pt%
\belowdisplayskip 12pt plus 3pt minus 9pt%
\abovedisplayshortskip 0pt plus 3pt%
\belowdisplayshortskip 7pt plus 3pt minus 4pt%
\smallskipamount=3pt plus1pt minus1pt%
\medskipamount=6pt plus2pt minus2pt%
\bigskipamount=12pt plus4pt minus4pt%
\def\rm{\fam0\tenrm}\def\it{\fam\itfam\tenit}%
\def\sl{\fam\slfam\tensl}\def\bf{\fam\bffam\tenbf}%
\def\smc{\tensmc}\def\mit{\fam 1}%
\def\cal{\fam 2}%
\textfont0=\tenrm\scriptfont0=\sevenrm\scriptscriptfont0=\fiverm%
\textfont1=\teni\scriptfont1=\seveni\scriptscriptfont1=\fivei%
\textfont2=\tensy\scriptfont2=\sevensy\scriptscriptfont2=\fivesy%
\textfont3=\tenex\scriptfont3=\tenex\scriptscriptfont3=\tenex%
\textfont\itfam=\tenit%
\textfont\slfam=\tensl%
\textfont\bffam=\tenbf\scriptfont\bffam=\sevenbf%
\scriptscriptfont\bffam=\fivebf%
\normalbaselines\rm}

%  Define a whole menagerie of pseudo-11pt fonts

\font\elevenrm=cmr10 scaled 1095    \font\eleveni=cmmi10 scaled 1095
\font\elevensy=cmsy10 scaled 1095   \font\elevenex=cmex10 scaled 1095
\font\elevenbf=cmbx10 scaled 1095   \font\elevensl=cmsl10 scaled 1095
\font\eleventt=cmtt10 scaled 1095   \font\elevenit=cmti10 scaled 1095

\font\ninerm=cmr5 scaled 1728    \font\ninei=cmmi5 scaled 1728
\font\ninesy=cmsy5 scaled 1728   \font\ninebf=cmbx5 scaled 1728

\font\sixrm=cmr5 scaled 1200    \font\sixi=cmmi5 scaled 1200
\font\sixsy=cmsy5 scaled 1200   \font\sixbf=cmbx5 scaled 1200

\def\elevenpoint{\normalbaselineskip=12.2pt plus 0.1pt minus 0.1pt%
\abovedisplayskip 12.2pt plus 3pt minus 9pt%
\belowdisplayskip 12.2pt plus 3pt minus 9pt%
\abovedisplayshortskip 0pt plus 3pt%
\belowdisplayshortskip 7.1pt plus 3pt minus 4pt%
\smallskipamount=3.3pt plus1.2pt minus1.2pt%
\medskipamount=6.6pt plus2.2pt minus2.2pt%
\bigskipamount=13.2pt plus4.4pt minus4.4pt%
\def\rm{\fam0\elevenrm}\def\it{\fam\itfam\elevenit}%
\def\sl{\fam\slfam\elevensl}\def\bf{\fam\bffam\elevenbf}%
\def\mit{\fam 1}\def\cal{\fam 2}%
\def\tt{\eleventt}%
\textfont0=\elevenrm\scriptfont0=\ninerm\scriptscriptfont0=\sixrm%
\textfont1=\eleveni\scriptfont1=\ninei\scriptscriptfont1=\sixi%
\textfont2=\elevensy\scriptfont2=\ninesy\scriptscriptfont2=\sixsy%
\textfont3=\elevenex\scriptfont3=\elevenex%
\scriptscriptfont3=\elevenex%
\textfont\itfam=\elevenit%
\textfont\slfam=\elevensl%
\textfont\bffam=\elevenbf\scriptfont\bffam=\ninebf%
\scriptscriptfont\bffam=\sixbf%
\normalbaselines\rm}%

\iftex{\tracingpages=1}
\font\helarge = cmr10 scaled 2488
\font\hebig = cmr10 scaled 1728

\font\heten = cmr10 scaled 1095
\font\hbten = cmbx10 scaled 1095
\font\hobten = cmr10 scaled 1095
\font\hbobten = cmbx10 scaled 1095

\font\henine = cmr7 scaled 1200
\font\hbnine = cmbx7 scaled 1200
\font\hobnine = cmr7 scaled 1200
\font\hbobnine = cmbx7 scaled 1200

\font\heeight = cmr7 scaled 1200
\font\hbeight = cmbx7 scaled 1095
\font\hobeight = cmr7 scaled 1200
\font\hbobeight = cmbx7 scaled 1095

\def\tenpt{\normalbaselineskip=11.2pt plus 0.1pt minus 0.1pt%
\def\rm{\fam0\heten}\def\it{\fam\itfam\hobten}%
\def\bf{\fam\bffam\hbten}\def\bit{\fam\slfam\hbobten}%
\normalbaselines\rm}%

\def\ninept{\normalbaselineskip=10.2pt plus 0.1pt minus 0.1pt%
\def\rm{\fam0\henine}\def\it{\fam\itfam\hobnine}%
\def\bf{\fam\bffam\hbnine}\def\bit{\fam\slfam\hbobnine}%
\normalbaselines\rm}%

\def\eightpt{\normalbaselineskip=9.2pt plus 0.1pt minus 0.1pt%
\def\rm{\fam0\heeight}\def\it{\fam\itfam\hobeight}%
\def\bf{\fam\bffam\hbeight}\def\bit{\fam\slfam\hbobeight}%
\normalbaselines\rm}%


\iftex{
\vsize=9.75truein	\voffset-.0truein
\hsize=7.5truein		\hoffset-.5truein
}

\iflatex{
\topmargin -.331truein
\textheight=9.819truein
%\advance\textheight by \topskip
\textwidth=7.5truein
\oddsidemargin=-0.5truein
\evensidemargin=\oddsidemargin
}

\parindent=0pt
\parskip=0pt
\baselineskip=0pt
\lineskip=0pt
\normallineskip=0pt
\topskip=4pt
\iflatex{
\headsep=6pt
}

\def\l{\vbox{\hsize=7.5in\baselineskip=0pt
\nointerlineskip
\hbox to 7.5in{\hrulefill}
\vskip3truept
\hbox to 7.5in{\hrulefill}
\nointerlineskip
}}

\def\indent#1{\vtop
        {\parindent=0pt\hangindent=15pt\hangafter=1\hsize=6.0truein
        \baselineskip 10pt
        \lineskip=0pt
        \lineskiplimit=2pt{#1\hfil}
        }}

\def\addbox#1{\vtop{\parindent=0pt\hsize=6.0truein\baselineskip=10pt
{#1}}}

\def\single{\eightpt\vtop{\hsize = 7.5in
          Suggest a single {\bit PLENARY SPEAKER} plus 2 speakers for one
	  {\bit SYMPOSIUM\/}.
           \hfil\break ALL INFORMATION REQUESTED MUST BE PROVIDED.
          }}
%\baselineskip 12pt
\iftex{\nopagenumbers}
\iflatex{\pagestyle{empty}}
\overfullrule=0pt
\iftex{\headline={\ifnum\pageno>1
\vtop{\hsize=7.4truein\eightpt%
{\bf SUGGESTION FOR SYMPOSIUM}\hfil\break
\hfil\break
\vbox{\symp}\hfil}
\else\hfil\fi}
}
\iflatex{\headheight16pt\def\@oddhead{\ifnum\thepage>1%
\vbox{\hsize=7.4truein\eightpt%
{\bf SUGGESTION FOR SYMPOSIUM}\hfil\break
\hfil\break
\vbox{\symp}\hfil}
\else\hfil\fi}
\def\@evenhead{\@oddhead}
}
%%


\newbox\thespeaker
\newbox\thephone
\newbox\theemail
\newbox\theaddress
\newbox\thetext
\newbox\therefone
\newbox\thereftwo
\newbox\therefthree
\newbox\thereffour
\newbox\thebook
\newbox\thebooktwo
\newbox\firstchairman
\newbox\thecategory
\newbox\thesymposiumtitle
\newbox\therecommender
\newbox\thetalktitle

\def\symp{\eightpt\vtop{\hsize=7.5in \noindent\hangindent=0pt\hangafter=1
   {Title of the Symposium:}
   \qquad\unhcopy\thesymposiumtitle\hfill}\vskip 5pt}
\def\no_of_speakers#1{\def\next{#1}\let\speakerno=\next}
\def\symposiumtitle#1{\setbox\thesymposiumtitle=
\hbox{\elevenpoint\vtop{\hsize=5.1truein\bf #1}}}
\def\category#1{\setbox\thecategory=\hbox{\elevenpoint #1}}
%
\def\speakername#1{\setbox\thespeaker=\hbox{\elevenpoint #1}}
\def\speakeremail#1{\setbox\theemail=\hbox{\tenpoint #1}}
\def\speakeraddress#1{\setbox\theaddress=\addbox{\tenpoint #1}}
\def\text#1{\setbox\thetext=\hbox to \hsize{\vtop{% to 1.75truein
\elevenpoint\noindent\unhcopy\thetalktitle\ #1 }}}
%	{\begingroup\parindent=0pt
%	{\unhcopy\thetalktitle} \reg #1 \hfil\endgroup}}}}

\def\checkbox#1{\vbox{\hsize=6.7truept\offinterlineskip%
\hrule\noindent\vrule height 5.6truept\hfill{\hfill#1\hfill}\hfill\vrule\par%
\hrule}} 


\def\pohead{\vbox
{\centerline{\helarge Invited Speaker/Symposium Suggestion Form}
\smallskip
\centerline{\hebig {\Large $\Psi_k$} Network Conference September18-21 1996 }
\vskip3pt
\l
\vskip4pt
\ninept
\halign{\hbox to 2in{\bf ##\hfil}&\hbox to 3.50in{\bf ##\hfil}%
&\hbox to 2.00in{\hfil \bf ##}\cr
Return completed form to: &O. K. Andersen
&DUE: 15 January 1996\cr
&Max-Planck-Institut f\"ur Festk\"orperforschung& NO FAX PLEASE\cr
&Postfach 80 06 65 \cr
&D-70506 Stuttgart\cr
&Germany\cr
&E-mail: psik@radix2.mpi-stuttgart.mpg.de\cr}
\vskip12pt
\tenpt
}}

\def\refone#1#2#3#4#5{\def\next{#1}\ifx\next\empty
	\setbox\therefone=\hbox to 0pt{\relax}
	\else \setbox\therefone=\hbox{\vtop{\noindent\elevenpoint
	\hangindent=36pt \hangafter=1 #1, {\sl #2} {\bf #3}, #4 (#5).\hfil
	}\hfil}\fi}
\def\reftwo#1#2#3#4#5{\def\next{#1}\ifx\next\empty
	\setbox\thereftwo=\hbox to 0pt{\relax}
	\else \setbox\thereftwo=\hbox{\vtop{\noindent\elevenpoint
	\hangindent=36pt \hangafter=1 #1, {\sl #2} {\bf #3}, #4 (#5).\hfil
	}\hfil}\fi}
\def\refthree#1#2#3#4#5{\def\next{#1}\ifx\next\empty
	\setbox\therefthree=\hbox to 0pt{\relax}
	\else \setbox\therefthree=\hbox{\vtop{\noindent\elevenpoint
	\hangindent=36pt \hangafter=1 #1, {\sl #2} {\bf #3}, #4 (#5).\hfil
	}\hfil}\fi}
\def\reffour#1#2#3#4#5{\def\next{#1}\ifx\next\empty
	\setbox\thereffour=\hbox to 0pt{\relax}
	\else \setbox\thereffour=\hbox{\vtop{\noindent\elevenpoint
	\hangindent=36pt \hangafter=1 #1, {\sl #2} {\bf #3}, #4 (#5).\hfil
	}\hfil}\fi}
 
\def\bookref#1{\if #1 \empty \setbox\thebook=\hbox to 0pt{\relax}
\else\setbox\thebook=\hbox{\vtop{\hangindent=36pt\hangafter=1
\elevenpoint #1\hfil}\hfil}\fi}
\def\bookreftwo#1{\if #1 \empty \setbox\thebooktwo=\hbox to 0pt{\relax}
\else\setbox\thebooktwo=\hbox{\vtop{\hangindent=36pt\hangafter=1
\elevenpoint #1\hfil}\hfil}\fi}
 
\iflatex{\let\pageno=\thepage}
\def\ref{\leftline{\eightpt Recent References%
\ifnum\pageno=1\ (give at least two)\fi:}
\vskip2pt
\unhcopy\therefone
\unhcopy\thereftwo
\unhcopy\therefthree
\unhcopy\thereffour
\unhcopy\thebook
\unhcopy\thebooktwo
}
 
 
\def\talktitle#1{\def\next{#1}\ifx\next\empty
	\setbox\thetalktitle=\hbox to 0pt{} \else
	\setbox\thetalktitle=\hbox{\elevenpoint\bf #1} \fi}

\def\talk#1{\vtop{\parindent=0pt\hangindent=0pt\hangafter=1\hsize=7.0truein
	\baselineskip 12pt
	\lineskip=0pt
	\lineskiplimit=2pt{\elevenpoint #1\hfil}\smallskip}}
 
\def\chairman#1#2#3#4{\setbox\firstchairman=
\vbox{\tenpt\baselineskip=11pt
\noindent Suggested Symposium Chairperson:\ \ {{\elevenpoint #1}}
\quad\hfill\hbox to 3.05truein{E-mail:\ \ {\tenpoint #3}\hfill}\par
\noindent Address:\ \ {\tenpoint #2}\hfill
\hbox to 2.0truein{Phone:\ \ {\tenpoint #4}\hfill}}
}


\def\recommender#1#2#3#4{\setbox\therecommender=
\vbox{\eightpt\hsize=7.2truein\parskip=2pt
\noindent Recommender's Name:\ \ {\elevenpoint #1}\hfill
\hbox to 3.0truein{E-mail:\ \ {\tenpoint #2}\hfill}\par
\noindent Institution, Address:\ \ \vtop{\hsize=4.0truein{\tenpoint #3}}\hfill
\hbox to 1.9truein{Phone:\ \ {\tenpoint #4}\hfill}}}

\def\frame#1{\vtop{\offinterlineskip\parskip=0pt\hsize=7.5in \hrule
\vrule height 2pt \hfill \vrule height 2pt\par
\vrule\hskip1ex #1 \hfill\hskip1ex\vrule\par
\vrule height 2pt \hfill \vrule height 2pt\par \hrule}}

\def\recom{\frame{\copy\therecommender\hfill}}

\def\ptalk{\eightpt\vtop{\hsize=7.5in \noindent\hangindent=0pt\hangafter=1
   {Title of the Plenary Talk:}
   \qquad\unhcopy\thetalktitle\hfill}\vskip 5pt}

\def\cat{\eightpt
   \leftline{Category {\it (see below)}:
   \qquad\copy\thecategory\hfil}
	\vskip 2pt}

\def\firstspeaker{\eightpt\frame{\vtop{\hsize=7.3in
\leftline{\hbox to 3.5truein{Speaker:\ \ \copy\thespeaker\hfil}\hfil
\hbox to 3.7truein{E-mail:\ \ \copy\theemail\hfil}}
\vskip 3pt
\leftline{Mailing Address:\ \ \copy\theaddress}}}
\vskip 4pt
\leftline{Title, Description, and Supporting Remarks (including what is
new):}
\vskip3pt
\unhcopy\thetext\hfil
}
\def\inputfirstspeakerinfo{}
\def\inputsecondspeakerinfo{}
\def\inputthirdspeakerinfo{}
%
\def\printfirstspeaker{
\iftex{
	\vbox{  \vskip-.35truein
        \vskip0pt
	\pohead
        \vskip6pt
	\single
 	\vskip6pt
	\ptalk
	\cat
	\firstspeaker}
	}
\iflatex{
\vbox{  \vskip-.40truein
	\pohead
        \vskip6pt
	\single
 	\vskip6pt
	\ptalk
	\cat
	\firstspeaker}
	}
\ifx\speakerno\empty\no_of_speakers{5}\fi
	\vskip 4pt
\filbreak
\vbox{\ref}
\vfil
\filbreak
\vskip3pt
\vbox{\categories}
\filbreak
 	\medskip
\vbox{\recom}
{\vskip1pt\rightline{\iflatex{\tenpoint \LaTeX}\iftex{\tenpoint \TeX}}}
\ifnum\pageno=1 \eject\fi
}

\def\printspeaker#1{
\ifnum\speakerno<#1 \else
\eightpt\frame{\vtop{\hsize=7.3in
\leftline{\hbox to 3.5truein{Speaker :\ \ \copy\thespeaker\hfil}\hfil
\hbox to 3.7truein{E-mail:\ \ \copy\theemail\hfil}}
\vskip 3pt
\leftline{Mailing Address:\ \ \copy\theaddress}}}
\vskip 4pt
\leftline{Title, Description, and Supporting Remarks (including what is new:)}
\vskip3pt
\unhcopy\thetext
\vskip 4pt
\vbox{\ref}
\filbreak
\fi
}
\def\printchair{
\vbox{\copy\firstchairman\par}
\filbreak
}

\def\printsecondspeaker{\printspeaker{2}}
\def\printthirdspeaker{\printspeaker{3}}
\def\categories{\baselineskip=0pt
\l
\vskip6truept
\centerline{\eightpt\bf CATEGORIES (INSERT ABOVE)}
\vskip5truept
\hbox{\vtop{\hsize=3.35in\baselineskip=7.58pt\parskip=0pt\eightpt
\noindent\phantom{1}1.\ Quantum Monte Carlo Calculations\par
\noindent\phantom{1}2.\ New developments in Muffin-tin and\par
\qquad Pseudopotential techniques\par
\noindent\phantom{1}3.\ Density Functional Molecular Dynamics\par
\qquad simulation: methods and applications\par
\noindent\phantom{1}4.\ Beyond the Local Density Approximation\par
\noindent\phantom{1}5.\ Quasiparticles\par
\noindent\phantom{1}6.\ Solving the Bogolubov-de Gennes Equations\par
\noindent\phantom{1}7.\ Approximate Methods for large Systems\par
\noindent\phantom{1}8.\ Parallel Algorithms}
\vtop{\hsize=6.55in\baselineskip=7.58pt\parskip=0pt\eightpt
9.\ Applications to Electron spectroscopies\par
10.\ Applications to Novel superconductors\par
11.\ Applications to Magnetic multilayers\par
12.\ Applications to Oxide/metal interfaces\par
13.\ Applications to Reactions at surfaces\par
14.\ Applications to Semiconductor heterostructures\par
15.\ Applications to Carbon Structures\par
16.\ Applications to Molecular Materials\par
17.\ Other applications}
}\par}


